\section{Специальные свойства бинарных отношений}

Пусть дано $A^2$ - бинарное отношение.

$\rho \subseteq A^2, A\neq \void$. Вместо $(x,y) \in \rho$ пишем
$x\rho y$.

Тогда его свойства:
\begin{itemize}
	\item Рефлексивность.

		Отношение $\rho$ называется рефлексивным, если  $(\forall x \in A)(x\rho x)$.
		То есть диагональ $id_A \subseteq \rho$.
	\item Иррефлексивность

		Отношение $\rho$ называется иррефлексивным, если $Id_A \cap A = \void$.
	\item Симметричность

		Отношение  $\rho$ называется симметричным, если \[
			(\forall ,y \in A)(x\rho y \implies y\rho x),
		\]
		то есть $\ rho = \rho^{-1}$
	\item Антисимметричность

		Отношение $\rho$ называется антисимметричным, если
		$$
			(\forall x,y \in A)(x\rho y \band y\rho x \implies x = y)
			$$
	\item Транзитивность

		Отношение $\rho$ называется транзитивным, если  \[
			(\forall x,y,z \in A)(x\rho y, y\rho z \implies x\rho z)
		\] 
\end{itemize}

\begin{theorem}
Отношение $\rho \subseteq A^2$ транзитивно тогда и только тогда, когда $\rho^2 \subseteq \rho$
\end{theorem}

\begin{myproof}
В прямую сторону.

Пусть $\rho \subseteq A^2$ транзитивно. Тогда если $x\rho^2 y$, то
$(\exists z)(x\rho z, z\rho y)$, то есть в силу транзитивности $x\rho y$.

\medskip

В обратную сторону.

Пусть  $\rho^2 \subseteq \rho$ и пусть $(\exists x,y,z \in \rho)(x\rho y, y\rho z)$.
Тогда мы имеем $x\rho^2 y \implies x\rho y$, то есть оно транзитивно.
\end{myproof}

\paragraph*{Замечание.}
Если отношение $\rho$ рефлексивно и транзитивно, то  $\rho^2  = \rho$


\medskip

Классы отношений:
\begin{itemize}
	\item[1)] Отношение эквивалентности
		\begin{itemize}
			\item Рефлексивность
			\item Симметричность
			\item Транзитивность
		\end{itemize}
	\item[2)] Отношение толерантности
		\begin{itemize}
			\item Рефлексивность
			\item Симметричность
		\end{itemize}
	\item[3)] Отношение порядка
		\begin{itemize}
			\item Рефлексивность
			\item Антисимметричность
			\item Транзитивность
		\end{itemize}
	\item[4)] Отношение предпорядка
		\begin{itemize}
			\item Рефлексивность
			\item Транзитивность
		\end{itemize}
\end{itemize}

