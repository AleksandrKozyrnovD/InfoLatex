\section{Неупорядоченная пара. Кортеж. Декартово произведение}

\begin{definition}
Пусть $A,B \neq \void, a \in A, b \in B$. Тогда $\{a,b\} $ - неупорядоченная пара
на множествах $A$ и  $B$.
\end{definition}

\medskip

При этом
\begin{itemize}
	\item Если $a=b$, то  $ \left| \{a,a\}  \right| = \left| \{a\}  \right| = 1$ 
	\item Если $a\neq b$, то $\left| \{a,b\}  \right| $ = 2
\end{itemize}

Равенство неупорядоченных пар:
\[
\{a,b\} = \{c,d\} \Leftrightarrow ((a=c)\band(b=d))\lor((a=d)\lor(b=c))   
\] 
\begin{definition}
Пусть $A,B\neq \void, a \in A, b \in B$. Тогда $(a,b)$ - упорядоченная пара на
множествах  $A$ и  $B$.
\end{definition}

Равенство упорядоченных пар: \[
	(a,b) = (c,d) \leftrightharpoons (a = c)\band(b=d)
\]

\medskip

Упорядоченная пара по определению не является множеством, но ее можно к нему свести: \[
	(a,b) \leftrightharpoons \{\{a\}, \{a,b\} \} 
\]
\medskip

\begin{definition}
Пусть даны множества $A_1,A_2,\ldots,A_{n}, n\ge 0$. Тогда
\[
	(a_1,a_2,\ldots,a_{n})\text{, где } a_1 \in A_1, a_2 \in A_2,\ldots,a_n \in A_n
\]
называется кортежем.
\end{definition}

\medskip

Можно задать через декартово умножение:
\[
A_1\times A_2\times \ldots\times A_n \leftrightharpoons \{(x_1,x_2,\ldots,x_{n}):
(\forall i=\overline{1,n})(x_{i} \in A_i)\} 
\] 

\medskip

По определению если $A_i = \void$, то все декартово произведение равно:
 \[
A_1\times A_2\times \ldots\times A_n = \void
\] 

\medksip

Если $A_1 = A_2 = \ldots = A_n$, то \[
A_1\times A_2\times \ldots\times A_n \leftrightharpoons A^{n}, n\ge 1
\] 

Также по определению $A^{0} \leftrightharpoons \{\lambda\} $, $\lambda$ - пустой кортеж,
а  $A \neq  \void$

\section{Дополнение к параграфу 1 и 2. Булеан}
\begin{definition}
Булеан множества $A$:  \[
2^{A} \leftrightharpoons \{x:x \subseteq A\} 
\].
То есть множество всех подмножеств.
\end{definition}

Пример:
\begin{align*}
	A &= \{a,b,c\} \\
	2^{A} &= \{\void, \{a\}, \{b\}, \{c\}, \{a,b\}, \{a,c\},\ldots     \} 
\end{align*}

\medskip

$2^{A} = \exp A=\Phi(A)=\beta(A)$

