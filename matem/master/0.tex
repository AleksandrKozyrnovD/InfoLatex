\chapter{Множества и отношения}
\section{Понятие множества. Операции над множествами}

Пусть дано $$A = \{x: P(x)\},$$ где $P(x)$ - предикат.

\medskip

Например:  \[
	A = \{x: x \text{ - четное число}\} 
\] 

$C = \{1,3,5\} = \{x:x=1 \lor x=2 \lor x=3\}$. $C$ - замкнутое множество.

\medskip

\begin{definition}
Равенство множеств можно задать так:
\[
A = B \leftrightharpoons (\forall x)(x \in A \Longleftrightarrow x \in B)
\] 
\end{definition}
Можно назвать это принципом экстенсиональности (extension).
При равенстве множеств важен только их состав.

\subsection{Подмножество}
\begin{definition}
Нестрогое включение:
$$A \subseteq B \leftrightharpoons (\forall x)(x \in A \implies x \in B)$$
\end{definition}

\medskip

Тогда равенство можно задать и через поднмножества:
\[
A = B \leftrightharpoons (A \subseteq B) \land (B \subseteq A)
\] 
\begin{definition}
Строгое включение: \[
A \subset B \leftrightharpoons (A\subseteq B) \land (A \neq B)
\] 
\end{definition}

Пара примеров:
\begin{gather*}
\{1,3,5\} = \{3,5,1\}  \neq \{\{1,3\},5 \} \\
\{1,3\} \subset \{1,3,5\} \\
\{1,3\} \in \{\{1,3\},5 \} \\
\{\{1,3\} \} \subset \{\{1,3\},5 \}  
\end{gather*}

\medskip

\begin{definition}
Пустое множество:
\[
\varnothing = \{x: F(x)\},
\]
где $F(x)$ - заведомо ложный предикат для всех  $x$.
По определению $\void \subseteq A$
\end{definition}


\begin{definition}
U - универсальное множество \[
	(\forall x)(x \in U)
\]

Также верно для всех других множеств: \[
A \subseteq U
\] 
\end{definition}

\subsection{Операции над множествами}
\begin{itemize}
\item Объединение \[
A \cup B \leftrightharpoons \{x: x \in A \lor x \in B\}
\]
\item Пересечение \[
A \cap B \leftrightharpoons \{x: x \in A \land x \in B\} 
\] 
\item Разность \[
A \setminus B \leftrightharpoons \{x: x \in A \land x \not\in B\} 
\]
\item Симметрическая разность \[
A \triangle B \leftrightharpoons (A\setminus B) \cup (B \setminus A) = (A \cup B) \setminus (A\cap B)
\] 
\item Дополнение \[
\overline{A} \leftrightharpoons \{x:x \not\in A\} = U \setminus A 
\] 
\end{itemize}

\medskip

Тождества операций над множествами:
\begin{itemize}
	\item \[
	A \cup (B \cup C) = (A \cup B) \cup C
	\]
	\[
	A \cap (B \cap C) = (A \cap B) \cap C
	\]
\item \[
A \cup B = B \cup A
\] 
\[
A \cap B = B \cap A
\]
\item \[
A \cap A = A \cup A = A
\] 
\item \[
A \cap (B \cup C) = (A \cap B) \cup (A \cap C)
\]
\item \[
A \cup \void = A
\] 
\[
A \cap \void = \void
\]
\item \[
\overline{A \cup B} = \overline{A} \cap	\overline{B}
\] 
\[
\overline{A \cap B} = \overline{A} \cup \overline{B}
\] 
\item \[
A \cup \overline{A} = U
\] 
\[
A \cap \overline{A} = \void
\] 
\item \[
A\trim B = B\trim A
\]
\item \[
A\trim (B \trim C) = (A\trim B)\trim C
\] 
\end{itemize}

\medskip

Метод доказательства тождеств с помощью двух включений. Пусть дано выражение \[
A \cap (B \cup C) = (A \cap B) \cup (A \cap C)
\]
Докажем его верность:
\begin{align*}
	x \in A \cap (B \cup C) &\implies \\
	\implies (x \in A)\band (x \in B \cup C) &\implies \\
	\implies (x \in A)\band ((x \in B)\lor(x \in C)) &\implies \\
	\implies ((x \in A)\band(x \in B))\lor((x \in A)\lor(x \in C)) &\implies \\
								       &\implies (x \in A \cap B) \cup (x \in A \cap C)
\end{align*}
Это верно и в обратную сторону (доказательство с конца в начало)

