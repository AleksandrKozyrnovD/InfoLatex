\documentclass{report}

\usepackage{amssymb, amsmath, amsthm, amscd}
\usepackage[utf8]{inputenc}
\usepackage[english, russian]{babel}
\usepackage{bookmark}

\usepackage{hyperref}
\hypersetup{
       colorlinks=true,
       linkcolor=blue,
}

\usepackage{geometry}
\geometry{papersize={15cm, 11in}, left=1.5cm, lmargin=1.5cm,right=2cm, top=2cm, bottom=3cm}

\usepackage{graphicx}

\usepackage{titlesec}
\titleformat{\chapter}[display]{\fontsize{17pt}{0pt}\bfseries}{}{-20pt}{}
\titleformat{\subsubsection}[display]{\fontsize{12pt}{0pt}\bfseries}{}{0pt}{}

\newcounter{mylabelcounter}

\makeatletter
\newcommand{\labelText}[2]{%
#1\refstepcounter{mylabelcounter}%
\immediate\write\@auxout{%
    \string\newlabel{#2}{{1}{\thepage}{{\unexpanded{#1}}}{mylabelcounter.\number\value{mylabelcounter}}{}}%
}%
}
\makeatother

\newcommand{\bslash}{\space \backslash \space}
\newcommand{\band}{\mbox{ } \& \mbox{ }}


\title{Дискретная математика}
\author{Козырнов Александр Дмитриевич, ИУ7-32Б}
\date{\today}

\begin{document}
\maketitle
\tableofcontents
\newpage

\chapter{Множества, отношения, алгебры}
\section{Опр. множества и операций над ним}

\subsection{Условие}
Множества, подмножества. Способы определения множеств. Равенство множеств.
Операции над множествами (объединение, пересечение, разность, симметрическая
разность, дополнение). Методы доказательства теоретико-множественных тождеств.

\subsection{Определение множества}
\textbf{Множества} мы будем обозначать большими буквами латинского алфавита $(X, Y, Z,...)$,
элементы множества малыми буквами $(x, y, z,...)$.
При­надлежность элемента х множеству X (неопределяемое понятие) будем обозначать $x \in X$.
Аналогично $x \notin X$ обозначает, что элемент х не принадлежит множеству X.

По определению считается $\varnothing \subseteq A$ и $A \subseteq U$

\subsection{Равенство множеств}
\textbf{$A = B \leftrightharpoons (\forall x)(x \in A \Leftrightarrow x \in B)$} \newline
Пример:\newline
$\{1,3,5\} = \{5,1,3\} = \{1,1,3,3,5,5\}$ \newline
Также возможно определить равенство через подмножество. \newline
$A \subseteq B \leftrightharpoons (\forall x)(x \in A \Rightarrow x \in B)$ - опр. нестрогого включения A в B.\newline
Тогда \newline
\textbf{$A = B \Leftrightarrow (A \subseteq B) \and (B \subseteq A)$}

\subsection{Операции над множествами}
\begin{itemize}
    \item объединение $A \cup B \leftrightharpoons \{x: x \in A \lor x \in B\}$
    \item
          пересечение $A \cap B \leftrightharpoons \{x: x \in A \space \& \space x \in B\}$
    \item
          разность
          $ A \bslash B \leftrightharpoons \{x:x \in A \band x \notin B\}$
    \item
          симметрическая разность
          $A \triangle B \leftrightharpoons (A \bslash B) \cup (B \bslash A) = (A \cup B) \bslash (A \cap B)$
    \item
          дополнение
          $ \overline{A} \leftrightharpoons \{x: x \notin A\} = U \bslash A$
\end{itemize}

\subsection{Методы доказательства теоретико-множественных тождеств}


\begin{itemize}
    \item Метод двух включений (На примере декартового умножения)\newline
          $A \times (B \cap C) = (A \times B) \cap (A \times C)$\newline
          Доказательство\newline
          $(x,y) \in A \times (B \cap C) \Leftrightarrow (x \in A) \band (y \in (B \cap C)) \Leftrightarrow (x \in A) \band (y \in B) \band (y \in C) \Leftrightarrow ((x,y) \in A \times B) \band ((x,y) \in A \times C) \Leftrightarrow (x,y) \in (A \times B) \cap (A \times C)$
    \item Методом Характеристических функций ($\chi_{A \times {B \cap C}}$)
    \item Методом эквивалетных преобазований ($\cap, \cup, \band, \bslash, ...$)
\end{itemize}
\newpage

\section{Неупорядоченная и упорядоченная пары, кортеж, Декартово произведение.}
\subsection{Условие}
Неупорядоченная пара, упорядоченная пара, кортеж. Декартово произведение множеств

\subsection{Виды вар}
\subsubsection{Неупорядоченная пара}
$A, B \neq 0, a \in A, b \in B$

Тогда $\{a,b\}$ - неупорядоченная пара на множествах A и B


$$
    \begin{cases}
        \left\{a,b\right\} = \{a\},|\{a\}| = 1 \mbox{, если $a = b$} \\
        |\left\{a,b\right\}| = 2 \mbox{, если $a \neq b$}            \\
    \end{cases}
$$

$\left\{a,b\right\} = \left\{c,d\right\} \leftrightharpoons ((a = c) \band (b = d)) \lor ((a = d) \band (b = c))$

То есть $\left\{1,2\right\} = \left\{2,1\right\}$

\subsubsection{Упорядоченная пара}
$A, B \neq 0, a \in A, b \in B$

Тогда $(a,b)$ - упорядоченная пара на множествах A и B

$(a,b) = (c,d) \leftrightharpoons (a = c) \band (b = d)$

То есть $(a,b) \neq (b,a)$


Ее можно свести к множеству: $(a,b) \leftrightharpoons \{\{a\}, \{a,b\}\}$

\subsection{Кортеж}
$A_1,A_2,\ldots,A_n, n \geq 1$

Тогда $(a_1,a_2,\ldots,a_n)$, где $a_1 \in A_1, a_2 \in A_2, \ldots, a_n \in A_n$ - кортеж

\begin{itemize}
    \item Равенство кортежей:\newline$(a_1,a_2,\ldots,a_n) = (b_1, b_2,\ldots,b_n) \leftrightharpoons (\forall i = \overline{1,n})(a_i = b_i)$
    \item Определение:\newline$A_1 \times A_2 \times \ldots \times A_n \leftrightharpoons \{(x_1, x_2, \ldots, x_n):(\forall i = \overline{1,n})(x_i \in A_i)\}$\newline
          По определению если $(\exists i = \overline{1,n})(A_i = \varnothing)$, то $A_1 \times A_1 \times \ldots \times A_n = \varnothing$
    \item Степень кортежа:\newlineЕсли $A_1 = A_2 = \ldots = A_n, n \geq 1$, то $A_1 \times A_2 \ldots \times A_n = A^n$\newline
          $A^0 \leftrightharpoons \{\lambda\}$, где $\lambda$ - пустой кортеж, $A \neq \varnothing$
\end{itemize}

Некоторые свойства декартового умножения:
\begin{itemize}
    \item $\overline{A \times B} = (\overline{A} \times \overline{B}) \cup (\overline{A} \times B) \cup (A \times \overline{B})$
    \item $A \times (B \bslash C) = (A \times B) \bslash (A \times C)$
    \item $A \times B \neq B \times A$
\end{itemize}
\newpage

\section{Отображение, частичное отображение}
\subsection{Условие}
Отображения: область определения, область значений. Инъективное, сюръективное и биективное отображения. Частичное отображение

\subsection{Отображение}
Отображение - это \underline{соответствие}, которое всюду определено и функционально по второй компоненте

\medskip

Пусть $\rho \subseteq A \times B$, тогда
\begin{itemize}
    \item Область определения: $D(\rho) \leftrightharpoons \{x:x \in A, (x,y) \in \rho\}$
    \item Область значений: $R(\rho) \leftrightharpoons \{y:y \in B, (x,y) \in \rho\}$
\end{itemize}

$f: A \underset{\bullet}{\rightarrow} B$ - частичное отображение, или отображение, где $D(\rho) \neq A$

Виды отобажений:
\begin{itemize}
    \item Инъективное отображение:\newline Если отображение функционально и по первой компоненте, то она называется инъекцией\newline
          $f: A \rightarrow B, (\forall x \in A)(\exists!\space y = f(x)) \band (\forall y \in R(f))(\exists!\space x \in A)(y = f(x))$
    \item Сюръективное отображение:\newline Это такое отображение, где $R(f) = B$, то есть\newline
          $(\forall y \in B)(\exists x \in A)(y = f(x))$
    \item Биективное отображение:\newline Это такое отображение, которое инъективно и сюръективно.\newline
          $(\forall x \in A)(\exists! y = f(x)) \band (\forall y \in B)(\exists! y = f(x))$
\end{itemize}
\newpage

\section{Соответствие}
\subsubsection{Условие}
Соответствия. График и граф соответствия, область определения, область значения.
Сечение соответствия. Сечение соответствия по множеству. Функциональность
соответствия по компоненте. Бинарные и n-арные отношения. Связь между
отношениями, соответствиями и отображениями.

\subsection{Соответствие}
$\rho \subseteq A \times B$ - соответствие из A в множество B, причем $A,B \neq \varnothing$


\begin{itemize}
    \item Область определения: $D(\rho) \leftrightharpoons \{x:x \in A, (x,y) \in \rho\}$
    \item Область значений: $R(\rho) \leftrightharpoons \{y:y \in B, (x,y) \in \rho\}$
    \item Сечение по $a \in A: \rho(a) \leftrightharpoons \{y: (a,y) \in \rho\}$
    \item Сечение по множеству $C \subseteq D(\rho): \rho(C) \leftrightharpoons \{(x,y):x \in C, (x,y) \in \rho\}$
\end{itemize}

Соответствие $\rho = A \times B$ называют функциональным по второй компоненте,
если $\forall (x,y)$ и $(x^\prime, y^\prime): x = x^\prime \Rightarrow y = y^\prime$

Соответствие $\rho = A \times B$ называют функциональным по первой компоненте,
если $\forall (x,y)$ и $(x^\prime, y^\prime): y = y^\prime \Rightarrow x = x^\prime$

\subsection{График и граф соответствия}
СДЕЛАТЬ

\subsection{Бинарные и н-арные (н-местные) отношения}

Н-местное (н-арное) отношение на множествах $A_1, A_2, \ldots, A_n, n \geq 1: \rho \subseteq A_1 \times A_2 \times \ldots \times A_n$

Бинарное отношение: $\rho = A^2, A \neq \varnothing$, иногда записывается как $x\rho y$

Свойства
\begin{itemize}
    \item Рефлексивность\newline
          $(\forall x \in A)(x\rho x)$, то есть диагональ $id_A \subseteq \rho$
    \item Иррефлексивность\newline
          $id_A \cap \rho = \varnothing$
    \item Симметричность\newline
          $\forall (x,y) \in A, x\rho y \Rightarrow y\rho x$, то есть $\rho = \rho^{-1}$
    \item  Антисимметричность\newline
          $\forall (x,y) \in A, x\rho y \band y\rho x \Rightarrow x = y$, например $x \leq y \band y \leq x \Rightarrow x = y$
    \item Транзитивность\newline
          $(\forall x,y,z \in A)(x\rho z \band z\rho y \Rightarrow x\rho y)$, например $x = z, z = y \Rightarrow x = y$
\end{itemize}

\subsection{Связь отношения, соответствия, отображения}
Соответствие - это бинарное отношения вида $\rho = A \times B$ или $\rho = A \times A$.
Отображение - это соответствие, которое всюду определено и функционально по второй компоненте.

\section{Композиция соответсвий, их свойства}
\subsection{Условие}
Композиция соответствий, обратное соответствие и их свойства (с доказательством)

\subsection{Композиция соответсвий}
$\rho \subseteq A \times B, \sigma \subseteq C \times D$

Тогда композиция:
\begin{itemize}
    \item $\rho \circ \sigma \rightleftharpoons \{(x,y): (x,z) \in \rho \band (z,y) \in \sigma\}$
    \item  $\rho = A \times B, \sigma = B \times C, \rho \circ \sigma = A \times C$
    \item $f: A \rightarrow B, g: B \rightarrow C, f \circ g: A \rightarrow C$
\end{itemize}
причем $R(\rho) \cap D(\sigma) \neq \varnothing$

\medskip

Покажем, что $f \circ g(x) = g(f(x))$

$f \circ g = \{(x,y): (\exists z)((x,z) \in f) \band ((z,y) \in g)\} = \{(x,y):(\exists z)(z = f(x), y = g(z))\} = \{(x,y): y = g(f(x))\}$

\subsection{Обратное соответствие}
$\rho^{-1} \rightleftharpoons \{(x,y): (y,x) \in \rho\}$ - обратное соответствие

Если $\rho = A \times B$, то $\rho^{-1} = B \times A$

\subsection{Свойства и их доказательства}
\begin{itemize}
    \item $\rho \circ (\sigma \circ \tau) = (\rho \circ \sigma) \circ \tau$\newline
          Рассмотрим $(\rho \circ (\sigma \circ \tau))(x)$: $(\rho \circ (\sigma \circ \tau))(x) = \rho((\sigma \circ \tau)(x)) = \rho(\sigma(\tau(x)))$\newline
          Рассмотрим $((\rho \circ \sigma) \circ \tau)(x)$: $((\rho \circ \sigma) \circ \tau)(x) = ((\rho \circ \sigma)(f(x)) = \rho(\sigma(\tau(x)))$\newline
          Как можем заметить, результат получен одинаковый
    \item $\rho \circ \sigma \neq \sigma \circ \rho$\newline
          Пусть $\rho = A \times B, f: A \rightarrow B$ и $\sigma = B \times C, g: B \rightarrow C$\newline
          Тогда $\rho \circ \sigma: A \rightarrow C$\newline
          Получаем $\sigma \circ \rho: B \rightarrow B$ при условии, что $B \cap A \neq \varnothing$, иначе $\sigma \circ \rho = \varnothing$\newline
          В обоих случаях $R(\rho \circ \sigma) \neq R(\sigma \circ \rho),
              D(\rho \circ \sigma) \neq D(\sigma \circ \rho) \Rightarrow
              \rho \circ \sigma \neq \sigma \circ \rho$ в общем случае
    \item $\rho \circ (\sigma \cup \tau) = (\rho \circ \sigma) \cup (\rho \circ \tau)$\newline
          $(x,y) \in \rho \circ (\sigma \circ \tau) \Rightarrow
              (\exists z)(((x,z) \in \rho) \land ((z,y) \in \sigma \cup \tau)) \Rightarrow
              (\exists z)((((x,z) \in \rho) \land ((z,y) \in \sigma) \lor ((z,y) \in \tau))) \Rightarrow
              (\exists u)(((x,u) \in \rho) \land ((u,y) \in \sigma) \lor (\exists v)((x,v) \in \rho \land (v,y) \in \tau))$
    \item  $\rho \circ (\sigma \cap \tau) \subseteq (\rho \circ \sigma) \cap (\rho \circ \tau)$\newline
          Доказательство Аналогично прошлому доказательству
    \item $(\rho^{-1})^{-1} = \rho$\newline
          $\rho^{-1} \rightleftharpoons \{(y,x): (x,y) \in \rho\}$\newline
          Тогда $(\rho^{-1})^{-1} \rightleftharpoons
              \{(x,y): (y,x) \in \rho^{-1}\} \rightleftharpoons
              \{(x,y): (x,y) \in \rho\} \rightleftharpoons \rho$
    \item $(\rho \circ \sigma)^{-1} = \sigma^{-1} \circ \rho^{-1}$\newline
          Пусть $\rho: A \rightarrow B, \sigma: B \rightarrow C$, тогда $\rho \circ \sigma: A \rightarrow C$\newline
          Тогда $(\rho \circ \sigma)^{-1}: C \rightarrow A$\newline
          $\rho^{-1}: B \rightarrow A,\mbox{ } \sigma^{-1}: C \rightarrow B$\newline
          Из этого следует $\sigma^{-1} \circ \rho^{-1}: C \rightarrow A$, что равно $(\rho \circ \sigma)^{-1}:C \rightarrow A$
    \item $\rho \subseteq A^{2}$

\end{itemize}

\end{document}