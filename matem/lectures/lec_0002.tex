\documentclass{report}

\usepackage{amssymb, amsmath, amsthm, amscd}
\usepackage[utf8]{inputenc}
\usepackage[russian]{babel}
\usepackage{bookmark}
\usepackage{wasysym}
\usepackage{import}
\usepackage{caption}
\usepackage{mathrsfs}

\usepackage[makeroom]{cancel}

\usepackage{tikz}
\usetikzlibrary{shapes,shapes.geometric,arrows,positioning,decorations.pathmorphing}

\usepackage{listings}

\usepackage{hyperref}
\hypersetup{
       colorlinks=true,
       linkcolor=blue,
}

\usepackage{geometry}
%\geometry{papersize={15cm, 11in}, left=1.5cm, lmargin=1.5cm,right=2cm, top=2cm, bottom=3cm}
\geometry{a4paper, left=1.5cm, lmargin=1.5cm,right=2cm, top=2cm, bottom=3cm}

\tolerance=1
\emergencystretch=\maxdimen
\hyphenpenalty=10000
\hbadness=10000

\usepackage{graphicx}

\usepackage{titlesec}
\titleformat{\chapter}[display]{\fontsize{17pt}{0pt}\bfseries}{}{-20pt}{}
\titleformat{\subsubsection}[display]{\fontsize{12pt}{0pt}\bfseries}{}{0pt}{}

\newcounter{mylabelcounter}

\makeatletter
\newcommand{\labelText}[2]{%
#1\refstepcounter{mylabelcounter}%
\immediate\write\@auxout{%
    \string\newlabel{#2}{{1}{\thepage}{{\unexpanded{#1}}}{mylabelcounter.\number\value{mylabelcounter}}{}}%
}%
}
\makeatother

\newcommand{\circlesign}[1]{
    \mathbin{
        \mathchoice
        {\buildcircledesign{\displaystyle}{#1}}
        {\buildcircledesign{\textstyle}{#1}}
        {\buildcircledesign{\scriptstyle}{#1}}
        {\buildcircledesign{\scriptscriptstyle}{#1}}
    }
}

\newcommand\buildcircledesign[2]{%
    \begin{tikzpicture}[baseline=(X.base), inner sep=0, outer sep=0]
        \node[draw,circle] (X) {\ensuremath{#1 #2}};
    \end{tikzpicture}%
}

\newcommand{\ozv}{\circlesign{*}}

\theoremstyle{plain}
\newtheorem{theorem}{Теорема}[chapter]
\theoremstyle{definition}
\newtheorem{definition}{Определение}
\newtheorem*{pruf}{Доказательство}

\newenvironment{myproof}[1][\textbf{\textup{Доказательство}}]
{\begin{proof}[#1]}
	%\renewcommand*{\qedsymbol}{\(\blacksquare\)}}
{\end{proof}}

\DeclareMathOperator{\band}{\&}
\DeclareMathOperator{\trim}{\triangle}
\DeclareMathOperator{\step}{\vdash}

\makeatletter
\newcommand*\bigcdot{\mathpalette\bigcdot@{.5}}
\newcommand*\bigcdot@[2]{\mathbin{\vcenter{\hbox{\scalebox{#2}{$\m@th#1\bullet$}}}}}
\makeatother


%\newcommand*\pathto[1]{ \Rightarrow^{*}_{#1} }

\newcommand{\pathto}[1][1]{ \Rightarrow^{*}_{#1} }

\newcommand{\dx}[2]{\frac{d#1}{d#2}}
\newcommand{\pdx}[2]{\frac{\partial#1}{\partial#2}}
\newcommand{\zap}[1]{\reflectbox{$3$}#1 3}

\newcommand{\incfig}[2]{%
    \def\svgwidth{#2\columnwidth}
    \import{./images/}{#1.pdf_tex}
}

\tikzstyle{startstop} = [rectangle, rounded corners, 
minimum width=3cm, 
minimum height=1cm,
text centered, 
draw=black, 
fill=red!30]

\tikzstyle{io} = [trapezium, 
trapezium stretches=true, % A later addition
trapezium left angle=70, 
trapezium right angle=110, 
minimum width=3cm, 
minimum height=1cm, text centered, 
draw=black, fill=blue!30]

\tikzstyle{process} = [rectangle, 
minimum width=3cm, 
minimum height=1cm, 
text centered, 
text width=3cm, 
draw=black, 
fill=orange!30]

\tikzstyle{decision} = [diamond, 
minimum width=3cm, 
minimum height=1cm, 
text centered, 
draw=black, 
fill=green!30]
\tikzstyle{arrow} = [thick,->,>=stealth]


%\newcommand{\void}{\varnothing}
\let\void\varnothing

\renewcommand{\phi}{\varphi}
\renewcommand{\epsilon}{\varepsilon}
\newcommand{\scr}[1]{\mathscr{#1}}



\title{}
\author{Козырнов Александр Дмитриевич, ИУ7-32Б}
\date{\today}

\begin{document}
\section{Неупорядоченная пара. Кортеж. Декартово произведение}

\begin{definition}
Пусть $A,B \neq \void, a \in A, b \in B$. Тогда $\{a,b\} $ - неупорядоченная пара
на множествах $A$ и  $B$.
\end{definition}

\medskip

При этом
\begin{itemize}
	\item Если $a=b$, то  $ \left| \{a,a\}  \right| = \left| \{a\}  \right| = 1$ 
	\item Если $a\neq b$, то $\left| \{a,b\}  \right| $ = 2
\end{itemize}

Равенство неупорядоченных пар:
\[
\{a,b\} = \{c,d\} \Leftrightarrow ((a=c)\band(b=d))\lor((a=d)\lor(b=c))   
\] 
\begin{definition}
Пусть $A,B\neq \void, a \in A, b \in B$. Тогда $(a,b)$ - упорядоченная пара на
множествах  $A$ и  $B$.
\end{definition}

Равенство упорядоченных пар: \[
	(a,b) = (c,d) \leftrightharpoons (a = c)\band(b=d)
\]

\medskip

Упорядоченная пара по определению не является множеством, но ее можно к нему свести: \[
	(a,b) \leftrightharpoons \{\{a\}, \{a,b\} \} 
\]
\medskip

\begin{definition}
Пусть даны множества $A_1,A_2,\ldots,A_{n}, n\ge 0$. Тогда
\[
	(a_1,a_2,\ldots,a_{n})\text{, где } a_1 \in A_1, a_2 \in A_2,\ldots,a_n \in A_n
\]
называется кортежем.
\end{definition}

\medskip

Можно задать через декартово умножение:
\[
A_1\times A_2\times \ldots\times A_n \leftrightharpoons \{(x_1,x_2,\ldots,x_{n}):
(\forall i=\overline{1,n})(x_{i} \in A_i)\} 
\] 

\medskip

По определению если $A_i = \void$, то все декартово произведение равно:
 \[
A_1\times A_2\times \ldots\times A_n = \void
\] 

\medksip

Если $A_1 = A_2 = \ldots = A_n$, то \[
A_1\times A_2\times \ldots\times A_n \leftrightharpoons A^{n}, n\ge 1
\] 

Также по определению $A^{0} \leftrightharpoons \{\lambda\} $, $\lambda$ - пустой кортеж,
а  $A \neq  \void$

\section{Дополнение к параграфу 1 и 2. Булеан}
\begin{definition}
Булеан множества $A$:  \[
2^{A} \leftrightharpoons \{x:x \subseteq A\} 
\].
То есть множество всех подмножеств.
\end{definition}

Пример:
\begin{align*}
	A &= \{a,b,c\} \\
	2^{A} &= \{\void, \{a\}, \{b\}, \{c\}, \{a,b\}, \{a,c\},\ldots     \} 
\end{align*}

\medskip

$2^{A} = \exp A=\Phi(A)=\beta(A)$

\end{document}
