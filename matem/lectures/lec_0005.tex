\documentclass{report}

\usepackage{amssymb, amsmath, amsthm, amscd}
\usepackage[utf8]{inputenc}
\usepackage[russian]{babel}
\usepackage{bookmark}
\usepackage{wasysym}
\usepackage{import}
\usepackage{caption}
\usepackage{mathrsfs}

\usepackage[makeroom]{cancel}

\usepackage{tikz}
\usetikzlibrary{shapes,shapes.geometric,arrows,positioning,decorations.pathmorphing}

\usepackage{listings}

\usepackage{hyperref}
\hypersetup{
       colorlinks=true,
       linkcolor=blue,
}

\usepackage{geometry}
%\geometry{papersize={15cm, 11in}, left=1.5cm, lmargin=1.5cm,right=2cm, top=2cm, bottom=3cm}
\geometry{a4paper, left=1.5cm, lmargin=1.5cm,right=2cm, top=2cm, bottom=3cm}

\tolerance=1
\emergencystretch=\maxdimen
\hyphenpenalty=10000
\hbadness=10000

\usepackage{graphicx}

\usepackage{titlesec}
\titleformat{\chapter}[display]{\fontsize{17pt}{0pt}\bfseries}{}{-20pt}{}
\titleformat{\subsubsection}[display]{\fontsize{12pt}{0pt}\bfseries}{}{0pt}{}

\newcounter{mylabelcounter}

\makeatletter
\newcommand{\labelText}[2]{%
#1\refstepcounter{mylabelcounter}%
\immediate\write\@auxout{%
    \string\newlabel{#2}{{1}{\thepage}{{\unexpanded{#1}}}{mylabelcounter.\number\value{mylabelcounter}}{}}%
}%
}
\makeatother

\newcommand{\circlesign}[1]{
    \mathbin{
        \mathchoice
        {\buildcircledesign{\displaystyle}{#1}}
        {\buildcircledesign{\textstyle}{#1}}
        {\buildcircledesign{\scriptstyle}{#1}}
        {\buildcircledesign{\scriptscriptstyle}{#1}}
    }
}

\newcommand\buildcircledesign[2]{%
    \begin{tikzpicture}[baseline=(X.base), inner sep=0, outer sep=0]
        \node[draw,circle] (X) {\ensuremath{#1 #2}};
    \end{tikzpicture}%
}

\newcommand{\ozv}{\circlesign{*}}

\theoremstyle{plain}
\newtheorem{theorem}{Теорема}[chapter]
\theoremstyle{definition}
\newtheorem{definition}{Определение}
\newtheorem*{pruf}{Доказательство}

\newenvironment{myproof}[1][\textbf{\textup{Доказательство}}]
{\begin{proof}[#1]}
	%\renewcommand*{\qedsymbol}{\(\blacksquare\)}}
{\end{proof}}

\DeclareMathOperator{\band}{\&}
\DeclareMathOperator{\trim}{\triangle}
\DeclareMathOperator{\step}{\vdash}

\makeatletter
\newcommand*\bigcdot{\mathpalette\bigcdot@{.5}}
\newcommand*\bigcdot@[2]{\mathbin{\vcenter{\hbox{\scalebox{#2}{$\m@th#1\bullet$}}}}}
\makeatother


%\newcommand*\pathto[1]{ \Rightarrow^{*}_{#1} }

\newcommand{\pathto}[1][1]{ \Rightarrow^{*}_{#1} }

\newcommand{\dx}[2]{\frac{d#1}{d#2}}
\newcommand{\pdx}[2]{\frac{\partial#1}{\partial#2}}
\newcommand{\zap}[1]{\reflectbox{$3$}#1 3}

\newcommand{\incfig}[2]{%
    \def\svgwidth{#2\columnwidth}
    \import{./images/}{#1.pdf_tex}
}

\tikzstyle{startstop} = [rectangle, rounded corners, 
minimum width=3cm, 
minimum height=1cm,
text centered, 
draw=black, 
fill=red!30]

\tikzstyle{io} = [trapezium, 
trapezium stretches=true, % A later addition
trapezium left angle=70, 
trapezium right angle=110, 
minimum width=3cm, 
minimum height=1cm, text centered, 
draw=black, fill=blue!30]

\tikzstyle{process} = [rectangle, 
minimum width=3cm, 
minimum height=1cm, 
text centered, 
text width=3cm, 
draw=black, 
fill=orange!30]

\tikzstyle{decision} = [diamond, 
minimum width=3cm, 
minimum height=1cm, 
text centered, 
draw=black, 
fill=green!30]
\tikzstyle{arrow} = [thick,->,>=stealth]


%\newcommand{\void}{\varnothing}
\let\void\varnothing

\renewcommand{\phi}{\varphi}
\renewcommand{\epsilon}{\varepsilon}
\newcommand{\scr}[1]{\mathscr{#1}}



\title{}
\author{Козырнов Александр Дмитриевич, ИУ7-32Б}
\date{\today}

\begin{document}
\section{Специальные свойства бинарных отношений}

Пусть дано $A^2$ - бинарное отношение.

$\rho \subseteq A^2, A\neq \void$. Вместо $(x,y) \in \rho$ пишем
$x\rho y$.

Тогда его свойства:
\begin{itemize}
	\item Рефлексивность.

		Отношение $\rho$ называется рефлексивным, если  $(\forall x \in A)(x\rho x)$.
		То есть диагональ $id_A \subseteq \rho$.
	\item Иррефлексивность

		Отношение $\rho$ называется иррефлексивным, если $Id_A \cap A = \void$.
	\item Симметричность

		Отношение  $\rho$ называется симметричным, если \[
			(\forall ,y \in A)(x\rho y \implies y\rho x),
		\]
		то есть $\ rho = \rho^{-1}$
	\item Антисимметричность

		Отношение $\rho$ называется антисимметричным, если
		$$
			(\forall x,y \in A)(x\rho y \band y\rho x \implies x = y)
			$$
	\item Транзитивность

		Отношение $\rho$ называется транзитивным, если  \[
			(\forall x,y,z \in A)(x\rho y, y\rho z \implies x\rho z)
		\] 
\end{itemize}

\begin{theorem}
Отношение $\rho \subseteq A^2$ транзитивно тогда и только тогда, когда $\rho^2 \subseteq \rho$
\end{theorem}

\begin{myproof}
В прямую сторону.

Пусть $\rho \subseteq A^2$ транзитивно. Тогда если $x\rho^2 y$, то
$(\exists z)(x\rho z, z\rho y)$, то есть в силу транзитивности $x\rho y$.

\medskip

В обратную сторону.

Пусть  $\rho^2 \subseteq \rho$ и пусть $(\exists x,y,z \in \rho)(x\rho y, y\rho z)$.
Тогда мы имеем $x\rho^2 y \implies x\rho y$, то есть оно транзитивно.
\end{myproof}

\paragraph*{Замечание.}
Если отношение $\rho$ рефлексивно и транзитивно, то  $\rho^2  = \rho$


\medskip

Классы отношений:
\begin{itemize}
	\item[1)] Отношение эквивалентности
		\begin{itemize}
			\item Рефлексивность
			\item Симметричность
			\item Транзитивность
		\end{itemize}
	\item[2)] Отношение толерантности
		\begin{itemize}
			\item Рефлексивность
			\item Симметричность
		\end{itemize}
	\item[3)] Отношение порядка
		\begin{itemize}
			\item Рефлексивность
			\item Антисимметричность
			\item Транзитивность
		\end{itemize}
	\item[4)] Отношение предпорядка
		\begin{itemize}
			\item Рефлексивность
			\item Транзитивность
		\end{itemize}
\end{itemize}

\section{Отношение эквивалентности и фактор-множества.}
$x \equiv k|(x-y)$ (k делит (x-y));  $x,y,k \in Z, k > 0$

Докажем, что $x \equiv ky \Longleftrightarrow mod(x,k)=mod(y,k)$.

\begin{myproof}
Пусть $x \equiv ky$, то есть
\begin{align*}
	k | (x-y) \implies& \\
	\implies k | [x]_k + mod(x,k) - [y]_k - mod(y,k) \implies& \\
	\implies k | mod(x,k) - mod(y,k) \implies& \\
	\implies mod(x,k)-mod(y,k) =& \\
	=& 0
\end{align*}

Если $mod(x,k) = mod(y,k)$, то  $x - y = [x]_k + \cancel{mod(x,k)} - [y]_k - \cancel{mod(y,k)}
= [x]_k - [y]_k \vdots k$
\end{myproof}

\begin{definition}
Пусть $\rho \subseteq A^2$ - отношение эквивалентности. Тогда \[
	[x]_{\rho} \leftrightharpoons \{y:y\rho x\} 
\]
называется классом эквивалентности элемента $x$ по отношению  $\rho$. \[
	(\forall x \in A)(x \in [x]_{\rho})
\] 
\end{definition}

\paragraph*{Пример.}
Пусть дано $\rho = \{(x,y):x^2+y^2=a^2\}$. Найдем такой класс эквивалентности: $[(x_0,y_0)]_{\rho}$ :
\[
	[(x_0,y_0)]_{\rho} = \{(x,y): x^2+y^2=x_0^2+y_0^2\},
\]
где $x_0,y_0$ не изменяются. Графиком такой функции будет окружность.

\begin{theorem}
Классы эквивалентности для любого произвольно заданного отношения эквивалентности
попарно не пересекаются.
\end{theorem}
\begin{myproof}
	Пусть $[x]_{\rho} \neq [y]_{\rho}$ и пусть $[x]_{\rho} \cap [y]_{\rho} \neq \void$.

	Так как $[x]_{\rho} \neq [y]_{\rho}$, то \[
		(\exists u)(u \in [x]_{\rho}\setminus [y]_{\rho}) \lor (\exists v)(v \in [y]_{\rho}
		\setminus [x]_{\rho})
	\] 

	Тогда \[
		u\rho x, z\rho x, z\rho y \implies u\rho x, x\rho z, z\rho y \implies x \in [y]_{\rho}
	\]
	Получено противоречие, так как он принадлежит сразу двум классам эквивалентности!
\end{myproof}

\end{document}
