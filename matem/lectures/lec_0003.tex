\documentclass{report}

\usepackage{amssymb, amsmath, amsthm, amscd}
\usepackage[utf8]{inputenc}
\usepackage[russian]{babel}
\usepackage{bookmark}

\usepackage{tikz}
\usetikzlibrary{shapes,arrows,positioning,decorations.pathmorphing}

\usepackage{listings}

\usepackage{hyperref}
\hypersetup{
       colorlinks=true,
       linkcolor=blue,
}

\usepackage{geometry}
\geometry{papersize={15cm, 11in}, left=1.5cm, lmargin=1.5cm,right=2cm, top=2cm, bottom=3cm}

\usepackage{graphicx}

\usepackage{titlesec}
\titleformat{\chapter}[display]{\fontsize{17pt}{0pt}\bfseries}{}{-20pt}{}
\titleformat{\subsubsection}[display]{\fontsize{12pt}{0pt}\bfseries}{}{0pt}{}

\newcounter{mylabelcounter}

\makeatletter
\newcommand{\labelText}[2]{%
#1\refstepcounter{mylabelcounter}%
\immediate\write\@auxout{%
    \string\newlabel{#2}{{1}{\thepage}{{\unexpanded{#1}}}{mylabelcounter.\number\value{mylabelcounter}}{}}%
}%
}
\makeatother

\newcommand{\bslash}{\mbox{ } \backslash \mbox{ }}
\newcommand{\band}{\mbox{ } \& \mbox{ }}

%\newcommand*\pathto[1]{ \Rightarrow^{*}_{#1} }

\newcommand{\pathto}[1][1]{ \Rightarrow^{*}_{#1} }

\renewcommand{\phi}{\varphi}


\title{}
\author{Козырнов Александр Дмитриевич, ИУ7-32Б}
\date{\today}

\begin{document}
\section{Отношение. Соответствие. Отображение.}
Рассуждения будут происходить от самого абстрактного к самому точному примеру.

\begin{definition}
Н-местное (н-арное) отношение на множествах $A_1,A_2,\ldots,A_{n}$: \[
\rho \subseteq A_1\times A_2\times \ldots\times A_n
\] 
\end{definition}

Пример:
\begin{gather*}
	A_1=A_2=\mathbb{R} \\
	\rho \leftrightharpoons \{(x,y): x^2+y^2=a^2, a\ge 0\} \\
\end{gather*}

\medskip

\begin{definition}
Если $A_1=A_2=\ldots=A_n$, то $\rho \subseteq A^{n}, n\ge 1$, то это называется
н-арным отношением на множестве $A$.
\end{definition}

\begin{definition}
Область определения: \[
D(\rho) \leftrightharpoons \{x: x \in A, (x,y) \in \rho\} 
\] 
\end{definition}

\begin{definition}
Область значений:
\[
R(\rho) \leftrightharpoons \{y: y \in B, (x,y) \in \rho\} 
\] 
\end{definition}

\begin{definition}
Область сечения соответствия $\rho \subseteq A\times B$ по $a \in A$: \[
\rho(a) \leftrightharpoons \{y: (a,y) \in \rho\} 
\] 
\end{definition}

\begin{definition}
Область сечения соответствия $\rho \subseteq A\times B$ по множеству $C\subseteq D(\rho)$:
 \[
\rho(C) \leftrightharpoons \{(x,y): x \in C, (x,y) \in \rho\} 
\] 
\end{definition}

\begin{definition}
Если у соотношения $\rho \subseteq A\times B$ будет верно $D(\rho) = A$, то оно называется
соответствием.
\end{definition}

\begin{definition}
Соответствие $\rho \subseteq A\times B$ называют функциональным по 2-й компоненте, если
$\forall (x,y)$ и $\forall (x',y'): x=x' \implies y=y'$
\end{definition}

\begin{definition}
Соответствие $\rho \subseteq A\times B$ называют функциональным по 1-й компоненте, если
для $\forall (x,y)$ и $(x',y'): y=y' \implies x = x'$
\end{definition}

\begin{definition}
Соответствие, которое всюду определено и функционально по 2-й компоненте, называют отображением. \[
f: A \to B 
.\]
Причем $D(f) = A$.
\end{definition}

\medskip

Также $(x,y) \in f$ превращается в $y = f(x)$. y - образ элемента х
при отображении в y. То есть $x=x'\implies f(x)=f(x')$

\begin{definition}
Частичное отображение - это такое отображение, где $D(f) \neq A$. \[
	f: A \underset{\bullet}{\to }B
\] 
\end{definition}

\begin{definition}
Если отображение функционально и по первой компоненте, то оно называется инъекцией. \[
	(\forall x \in A)(\exists! y=f(x))\band(\forall y \in R(f))(\exists !x \in A)(y=f(x))
\] 
\end{definition}

\begin{definition}
Сюръекция - это отображение, где $R(f) = B$.  \[
	(\forall y \in B)(\exists x \in A)(y = f(x))
\] 
\end{definition}


\begin{definition}
Биекция - это инъекция и сюръекция одновременно.
\end{definition}

\begin{definition}
Два множества называют эквивалентными, если между ними можно установить
взаимооднозначное соответствие.
\begin{align*}
	&1) A \sim A (f: A\to A\text{, где }(\forall x \in A)(f(x)=x)) \\
	&2) A \sim B, B \sim C \implies A \sim C \\
	&3) A\sim B \implies B\sim A
\end{align*}
\end{definition}

\paragraph*{Заметка.} Принимается, как аксиома, что между 2-мя конечными множествами может
быть установлено однозначно взаимное соответствие, если они состоят из одинакового числа
элементов.

\paragraph*{Способы доказательства на отображения.} Докажем, что \[
f: X \to Y; A,B \subseteq X \implies f(A \cup B) = f(A) \cup f(B)

\begin{myproof}
\begin{align*}
	y \in f(A \cup B) &\implies\\
	\implies (\exists x \in A \cup B)(y=f(x)) &\implies \\
	\implies (\exists x)(x \in A, y=f(x)) \lor (x \in B, y=f(x)) &\implies \\
								     &\implies y \in f(A) \cup f(B) 
\end{align*}
Докажем в обратную сторону:
\begin{align*}
	y \in f(A) \cup f(B) &\implies \\
	\implies y \in f(A) \lor y \in f(B) &\implies \\
	\implies (\exists x \in A)(y = f(x)) \lor (\exists x' \in B)(y=f(x')) &\implies \\
	\implies (\exists t \in A \cup B)(y = f(t)) &\implies \\
						    &\implies y \in f(A \cup B)
\end{align*}
\end{myproof}
\] 
\end{document}
