\documentclass{report}

\usepackage{amssymb, amsmath, amsthm, amscd}
\usepackage[utf8]{inputenc}
\usepackage[russian]{babel}
\usepackage{bookmark}
\usepackage{wasysym}
\usepackage{import}
\usepackage{caption}
\usepackage{mathrsfs}

\usepackage[makeroom]{cancel}

\usepackage{tikz}
\usetikzlibrary{shapes,shapes.geometric,arrows,positioning,decorations.pathmorphing}

\usepackage{listings}

\usepackage{hyperref}
\hypersetup{
       colorlinks=true,
       linkcolor=blue,
}

\usepackage{geometry}
%\geometry{papersize={15cm, 11in}, left=1.5cm, lmargin=1.5cm,right=2cm, top=2cm, bottom=3cm}
\geometry{a4paper, left=1.5cm, lmargin=1.5cm,right=2cm, top=2cm, bottom=3cm}

\tolerance=1
\emergencystretch=\maxdimen
\hyphenpenalty=10000
\hbadness=10000

\usepackage{graphicx}

\usepackage{titlesec}
\titleformat{\chapter}[display]{\fontsize{17pt}{0pt}\bfseries}{}{-20pt}{}
\titleformat{\subsubsection}[display]{\fontsize{12pt}{0pt}\bfseries}{}{0pt}{}

\newcounter{mylabelcounter}

\makeatletter
\newcommand{\labelText}[2]{%
#1\refstepcounter{mylabelcounter}%
\immediate\write\@auxout{%
    \string\newlabel{#2}{{1}{\thepage}{{\unexpanded{#1}}}{mylabelcounter.\number\value{mylabelcounter}}{}}%
}%
}
\makeatother

\newcommand{\circlesign}[1]{
    \mathbin{
        \mathchoice
        {\buildcircledesign{\displaystyle}{#1}}
        {\buildcircledesign{\textstyle}{#1}}
        {\buildcircledesign{\scriptstyle}{#1}}
        {\buildcircledesign{\scriptscriptstyle}{#1}}
    }
}

\newcommand\buildcircledesign[2]{%
    \begin{tikzpicture}[baseline=(X.base), inner sep=0, outer sep=0]
        \node[draw,circle] (X) {\ensuremath{#1 #2}};
    \end{tikzpicture}%
}

\newcommand{\ozv}{\circlesign{*}}

\theoremstyle{plain}
\newtheorem{theorem}{Теорема}[chapter]
\theoremstyle{definition}
\newtheorem{definition}{Определение}
\newtheorem*{pruf}{Доказательство}

\newenvironment{myproof}[1][\textbf{\textup{Доказательство}}]
{\begin{proof}[#1]}
	%\renewcommand*{\qedsymbol}{\(\blacksquare\)}}
{\end{proof}}

\DeclareMathOperator{\band}{\&}
\DeclareMathOperator{\trim}{\triangle}
\DeclareMathOperator{\step}{\vdash}

\makeatletter
\newcommand*\bigcdot{\mathpalette\bigcdot@{.5}}
\newcommand*\bigcdot@[2]{\mathbin{\vcenter{\hbox{\scalebox{#2}{$\m@th#1\bullet$}}}}}
\makeatother


%\newcommand*\pathto[1]{ \Rightarrow^{*}_{#1} }

\newcommand{\pathto}[1][1]{ \Rightarrow^{*}_{#1} }

\newcommand{\dx}[2]{\frac{d#1}{d#2}}
\newcommand{\pdx}[2]{\frac{\partial#1}{\partial#2}}
\newcommand{\zap}[1]{\reflectbox{$3$}#1 3}

\newcommand{\incfig}[2]{%
    \def\svgwidth{#2\columnwidth}
    \import{./images/}{#1.pdf_tex}
}

\tikzstyle{startstop} = [rectangle, rounded corners, 
minimum width=3cm, 
minimum height=1cm,
text centered, 
draw=black, 
fill=red!30]

\tikzstyle{io} = [trapezium, 
trapezium stretches=true, % A later addition
trapezium left angle=70, 
trapezium right angle=110, 
minimum width=3cm, 
minimum height=1cm, text centered, 
draw=black, fill=blue!30]

\tikzstyle{process} = [rectangle, 
minimum width=3cm, 
minimum height=1cm, 
text centered, 
text width=3cm, 
draw=black, 
fill=orange!30]

\tikzstyle{decision} = [diamond, 
minimum width=3cm, 
minimum height=1cm, 
text centered, 
draw=black, 
fill=green!30]
\tikzstyle{arrow} = [thick,->,>=stealth]


%\newcommand{\void}{\varnothing}
\let\void\varnothing

\renewcommand{\phi}{\varphi}
\renewcommand{\epsilon}{\varepsilon}
\newcommand{\scr}[1]{\mathscr{#1}}



\title{Домашнее задание №1\\ Логика и Теория Алгоритмов}
\author{Козырнов Александр Дмитриевич\\ ИУ7-42Б\\Вариант 6}
\date{\today}

\begin{document}
\maketitle
\tableofcontents
\newpage

\section{Условие задачи}
Построить НА, который аннулирует все слова вида $x\$x$, где
$x \in \{a,b\}^{*}$, а $\$ \not\in \{a,b\} $

\section{Решение задачи}


Назовем наш НА $DoubleDel$
\[
DoubleDel:
\begin{cases}
    \alpha\xi \to \xi\beta\xi\alpha \text{ //}\xi,\eta \in V \cup \{\$\} & (1)\\
    \beta\xi\eta \to \eta\beta\xi \text{ //}\alpha,\beta,\# \not\in V \cup \{\$\} & (2)\\
    \alpha \to \#& (3)\\
    \beta\xi\# \to \#\xi& (4)\\
    
    \zeta\chi\gamma\zeta \to \gamma\zeta\chi\zeta & (5)\\
    \chi\gamma\zeta \to \zeta\chi\gamma\zeta & (6)\\
    \zeta\chi\gamma \to \gamma\zeta\chi \text{ //}\chi,\gamma \in V; \zeta, \rho \not\in V& (7)\\
    \zeta\chi \to \rho\chi & (8)\\
    \rho\chi\rho \to \rho\chi & (9)\\
    \chi\rho \to \rho\chi & (10)\\
    \zeta \to \square& (11)\\
    \rho \to \square & (12)\\

    \square\$ \to \triangle& (13)\\
    \square\chi \to \chi\square& (14)\\
    \chi\square \to \sigma& (15)\\ 
    \chi\triangle\chi \to \triangle& (16)\\
    \chi\triangle\gamma \to \sigma \text{ //Ошибка сравнения}& (17) \\
    \chi\triangle\# \to \sigma\# \text{ //Ошибка сравнения, левое слово больше правого}& (18)\\
    \triangle\chi \to \sigma \text{ //Ошибка сравнения, правое слово больше левого}& (19)\\
    \triangle\# \to \delta \text{ //Сравнение прошло удачно}& (20)\\
    \xi\sigma \to \sigma& (21)\\
    \sigma\xi \to \sigma& (22)\\
    \sigma\# \to \bigcdot& (23)\\
    \xi\delta \to \delta& (24)\\
    \delta\xi \to \delta& (25)\\
    \delta\# \to \delta& (26)\\
    \delta \to \bigcdot& (27)\\
    \zeta\alpha \to \bigcdot& (28)\\
    \to \zeta\alpha& (29)\\
\end{cases}
\] 

Служебные символы - это $\alpha, \beta, \#, \$, \zeta, \rho, \triangle, \square$.
Их можно объединить в группы.

Например, символы  $\alpha$ и  $\beta$ учавствуют в копировании исходного слова. Причем
копируют они символы вида  $\xi \in V \cup \{\$\} $, то есть с разделителем. Команды,
копирующие слова, будут с номерами $(1) - (4)$. Копирование происходит с
разделителем  $\#$, который отделяет две разные пары. С первой парой мы будем работать и
изменять, когда вторую оставляем для неудачного варианта. При выполнении данных комманд
будет получено слово $x\$x\#x\$x$

Команды  $(5) - (12)$ и символы  $\zeta, \rho$ работают с переворачиванием первого
слова до разделителя  $\$$, то есть работают с символами  $\chi$ (Хи) и  $\gamma$, относящиеся
к алфавиту $V$ и не более. Он является модификацией алгорифма переворота слов, где
используется множественное количество раз команда типа $ \to \zeta$. В моей модификации
потребуется только одна модификация. Это возможно из-за команд $(5)$ и  $(6)$.
При работе данной части алгорифма будет получено что-то вроде  $\square x^{R}\$x\#x\$x$.

Далее командами $(13) - (15)$ ищется первый разделитель  $\$$. В случае, если он будет
не найден, то есть  $\square$ дошел до разделителя  $\#$, будет получен негативный вариант 
ластика. Если разделитель будет найден, то поставится значок сравнения $\triangle$. В итоге будет
получено слово $x^{R}\triangle x\#x\$x$.

Со значкном сравнения  $\triangle$ будут сравниваться первые два слова. В случае успеха
будет поставлен ластик, стирающий все символы. В случае неуспеха будет поставлен
ластик, стирающий все символы до  $\#$, включая ее саму. Команды работы
со сравнителем  $(16) - (20)$.

Далее идет часть НА, которая отвечает за конечный ответ. В случае успеха
будет получен символ  $\delta$, стирающий все символы, в том числе  $\#$. В
случае  $\sigma$ - неуспешного сравнения - будут стерты все символы, а далее
завершающе стерт символ  $\#$ (команда $(23)$).

\medskip

Логические части не могут пересекаться, потому что у них всех 
единственная инициализация челноков ($\to \zeta\alpha$), которая
превращается по ходу программы в другие типы челноков, ответственные
за свою работу. Так, символы $\alpha, \beta$ полностью исчезают и
дают символ  $\#$. Оставшийся символ  $\zeta$ превращается в итоге
в  $\square$.  $\square$ в свою очередь тоже челнок и ищет
первый разделитель  $\$$ и при его нахождении превращается
в  $\triangle$. После сравнения слов треугольником он может
превратиться в два типа ластиков: стирающий все ($\delta$) 
или стирающий только левую пару слов ($\sigma$).

<<<<<<< HEAD
  При подаче пустого слова сработает формула (12) $\to$ (11). При подаче слова без вхождения
  слова $u$ алгорифм завершает работу на \newline (12) $\to$ (10) формулах. В случае
  вхождения какого-то слова $\omega$ до слова  $u$ завершаем НА формулой (12) $\to $ (10).

\newpage
=======
>>>>>>> _alg
\section{Прогонка}

\subsection{Положительный результат работы НА}

В случае успеха слово должно быть аннулировано.

\paragraph*{Пример 1.} ${}$ \newline
Пусть подаем слово $\lambda\$\lambda$, то есть слова равны и нас интересует результат
вида  $\lambda$.

\begin{align*}
<<<<<<< HEAD
    DoubleDel(a\$a)
    = a\$a \underset{(12)}{\step} \square a\$a 
                      \underset{(9)}{\step} a\#\$a \underset{(6)}{\step} a\$ a\nabla
    \underset{(2)}{\step} a\$a\gamma \underset{(3),(4)}{\models}^{3} \gamma 
    \underset{(5)}{\step}\bigcdot \lambda
=======
    DoubleDel(\lambda\$\lambda) &= \$ \underset{(29)}{\step} \zeta\alpha\$
    \underset{(1)}{\step} \zeta\$\beta\$\alpha \underset{(3)}{\step} \zeta\$\beta\$\#
    \underset{(4)}{\step} \zeta\$\#\$ \underset{(11)}{\step} \square\$\#\$
    \underset{(13)}{\step} \triangle\#\$ \underset{(20)}{\step} \delta\$
    \underset{(25)}{\step} \delta \underset{(27)}{\step}\bigcdot \lambda
>>>>>>> _alg
\end{align*}


\paragraph*{Пример 2.} ${}$ \newline
Подаем на вход слово $a\$a$

 \begin{align*}
     DoubleDel(a\$a) &= a\$a \underset{(29)}{\step} \zeta\alpha a\$a
     \underset{(1)}{\step} \zeta a\beta a\alpha\$a \underset{(1)}{\step}
     \zeta a\beta a \$\beta\$ \alpha a \underset{(1)}{\step}
     \zeta a\beta a \$\beta\$ a\beta a\alpha \underset{(2)}{\step}\\
                     &\underset{(2)}{\step} \zeta a\$\beta a\beta\$a\beta a\alpha \underset{(2)}{\step}
                     \zeta a\$\beta aa\beta\$\beta a\alpha \underset{(2)}{\step}
                     \zeta a\$a\beta a\beta\$\beta a\alpha \underset{(3)}{\step}\\
                     &\underset{(3)}{\step} \zeta a\$a\beta a\beta\$\beta a\#
                     \underset{(4)}{\models}^{3} \zeta a\$a\#a\$a
                     \underset{(8)}{\step} \rho a\$a\#a\$a \underset{(12)}{\step}\\
                     &\underset{(12)}{\step} \square a\$a\#a\$a \underset{(14)}{\step}
                     a\square\$a\#a\$a \underset{(13)}{\step}
                     a\triangle a\#a\$a \underset{(16)}{\step}\\
                     &\underset{(16)}{\step} \triangle\#a\$a \underset{(20)}{\step}
                     \delta a\$a \underset{(25)}{\models}^{3} \delta
                     \underset{(27)}{\step}\bigcdot\\
                     &\underset{(27)}{\step}\bigcdot \lambda
\end{align*}

\newpage

\paragraph*{Пример 3.} ${}$ \newline
Подадим на вход слово 123\$123 (дабы убедится, что на больших словах тоже работает)

\begin{align*}
<<<<<<< HEAD
    DoubleDel(abba\$abba)
    = abba\$abba \underset{(12)}{\step} \square abba\$abba \underset{(9)}{\step} abba\#\$abba
    \underset{(6)}{\step}
     & abba\$ abba\nabla \underset{(2)}{\step} abba\$abba\gamma
    \underset{(3),(4)}{\models}^{9} \gamma \underset{(5)}{\step}\bigcdot \lambda
=======
    DoubleDel(123\$123) &= 123\$123 \underset{(29)}{\step} \zeta\alpha 123\$123
    \underset{(1)}{\step} \zeta 1\beta 1\alpha 23\$123 \underset{(1)}{\step} 
    \zeta 1\beta 1 2\beta 2\alpha 3\$123 \models\\
                        &\models \zeta 123\$123\#123\$123 \underset{(7)}{\step}
                        2\zeta 13\$ 123\#123\#123
                        \underset{(7)}{\step} 23\zeta 1\$123\#123\$123 \underset{(6)}{\step}\\
                        &\underset{(6)}{\step} \zeta 23\zeta 1\$123\#123\$123 \underset{(5)}{\step}
                        3\zeta 2\zeta 1\$123\#123\$123 \underset{(8)}{\step}\\
                        &\underset{(8)}{\step} 3\rho 2\zeta 1\$123\#123\$123
                        \underset{(8)}{\step} 3\rho 2\rho 1\$123\#123\$123 \underset{(9)}{\step}\\
                        &\underset{(9)}{\step}
                        3\rho 21\$123\#123\$123 \underset{(10)}{\step} \rho 321\$123\#123\$123
                        \underset{(12)}{\step}\\
                        &\underset{(12)}{\step} \square 321\$123\#123\$123 \underset{(14)}{\models}^{3}
                        321\square\$123\#123\$123 \underset{(13)}{\step}\\
                        &\underset{(13)}{\step} 321\triangle 123\#
                        123\$123 \underset{(16)}{\models}^{3} \triangle\#123\$123
                        \underset{(20)}{\step}\\
                        &\underset{(20)}{\step} \delta 123\$123 \underset{(25)}{\models}^{7} \delta
                        \underset{(27)}{\step}\bigcdot\\
                        &\underset{(27)}{\step}\bigcdot \lambda
>>>>>>> _alg
\end{align*}

\subsection{Отрицательный результат работы НА}

В случае неуспеха нормальный алгорифм вычислит тождественную функцию

\paragraph*{Пример 1.} ${}$\newline
Пусть подаем слово $\lambda$

\begin{align*}
     &DoubleDel(\lambda) = \lambda \underset{(29)}{\step}
     \zeta\alpha \underset{(28)}{\step}\bigcdot \lambda
\end{align*}


\paragraph*{Пример 2.} ${}$ \newline
Подадим на вход неравные слова, например ab\$a

\begin{align*}
    DoubleDel(ab\$a) &= ab\$a \underset{(29)}{\step} \zeta\alpha ab\$a \underset{(1)}{\step}
    \zeta a\beta a\alpha b\$a \models \zeta ab\$a\#ab\$a \underset{(7)}{\step}\\
                     &\underset{(7)}{\step} b\zeta a\$a\#ab\$a \underset{(8)}{\step}
                     b\rho a\$a\#ab\$a \underset{(10)}{\step} \rho ba\$a\#ab\$a
                     \underset{(12)}{\step}\\
                     &\underset{(12)}{\step} \square ba\$a\#ab\$a
                     \underset{(14)}{\models}^{2} ba\square\$a\#ab\$a
                     \underset{(13)}{\step} ba\triangle a\#ab\$a 
                     \underset{(16)}{\step}\\
                     &\underset{(16)}{\step} b\triangle\#ab\$a
                     \underset{(18)}{\step} \sigma\#ab\$a
                     \underset{(23)}{\step} \bigcdot\\
                     &\underset{(23)}{\step} \bigcdot ab\$a
\end{align*}


\end{document}
