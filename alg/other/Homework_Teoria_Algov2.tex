\documentclass{report}

\usepackage{amssymb, amsmath, amsthm, amscd}
\usepackage[utf8]{inputenc}
\usepackage[russian]{babel}
\usepackage{bookmark}

\usepackage{tikz}
\usetikzlibrary{shapes,arrows,positioning,decorations.pathmorphing}

\usepackage{listings}

\usepackage{hyperref}
\hypersetup{
       colorlinks=true,
       linkcolor=blue,
}

\usepackage{geometry}
\geometry{papersize={15cm, 11in}, left=1.5cm, lmargin=1.5cm,right=2cm, top=2cm, bottom=3cm}

\usepackage{graphicx}

\usepackage{titlesec}
\titleformat{\chapter}[display]{\fontsize{17pt}{0pt}\bfseries}{}{-20pt}{}
\titleformat{\subsubsection}[display]{\fontsize{12pt}{0pt}\bfseries}{}{0pt}{}

\newcounter{mylabelcounter}

\makeatletter
\newcommand{\labelText}[2]{%
#1\refstepcounter{mylabelcounter}%
\immediate\write\@auxout{%
    \string\newlabel{#2}{{1}{\thepage}{{\unexpanded{#1}}}{mylabelcounter.\number\value{mylabelcounter}}{}}%
}%
}
\makeatother

\newcommand{\bslash}{\mbox{ } \backslash \mbox{ }}
\newcommand{\band}{\mbox{ } \& \mbox{ }}

%\newcommand*\pathto[1]{ \Rightarrow^{*}_{#1} }

\newcommand{\pathto}[1][1]{ \Rightarrow^{*}_{#1} }

\renewcommand{\phi}{\varphi}


\usepackage{karnaugh-map}

\title{Домашнее задание №2\\ Логика и Теория Алгоритмов}
\author{Козырнов Александр Дмитриевич\\ ИУ7-42Б\\Вариант 6}
\date{\today}

\begin{document}
\maketitle
\tableofcontents
\newpage

\chapter{Задача 1}

\section{Условие}

Для булевой функции f, заданной в таблице 1:

а) найти сокращенную ДНФ; б) найти ядро функции;

в) получить все тупиковые ДНФ и указать, какие из них являются минимальными;

г) на картах Карно указать ядро и покрытия, соответствующие минимальным ДНФ.

\medskip

Сама функция $f$, заданная в виде вектора значений:
 \[
     \boxed{f(1100\text{ }1101\text{ }1101\text{ }1001)}
\] 

\section{Решение}


\subsection{Карты Карно}

%Uncolored Variant. For colored variant make \begin{karnaught-map}[4][4]][1][..][..]
\begin{karnaugh-map}[4][4][1][$X_3X_4$][$X_1X_2$]
    \minterms{0,1,
                4,5,7,
                8,9,11,
                12,15}
    \implicant{0}{8}
    \implicant{0}{5}
    \implicant{5}{7}
    \implicant{7}{15}
    \implicant{15}{11}
    \implicant{9}{11}
    \implicantedge{0}{1}{8}{9}
\end{karnaugh-map}

\subsection{Ядро функции и сокращенная ДНФ}

Ядром функции будет являться $\bar{x}_3 \bar{x}_4$ 

\medskip

Сокращенная ДНФ:
\[
    \boxed{
\bar{x}_3\bar{x}_4 \lor \bar{x}_2 \bar{x}_3 \lor \bar{x}_1 \bar{x}_3 \lor x_2x_3x_4 \lor x_1x_3x_4 \lor
x_1 \bar{x}_2 x_4 \lor \bar{x}_1 x_2x_4
}
\] 

\subsection{Поиск Тупиковых ДНФ. Функция Патрика}

$
\begin{matrix}
    K_1 = \bar{x}_3 \bar{x}_4 && K_2 = \bar{x}_2 \bar{x}_3\\ \\
    K_3 = \bar{x}_1 \bar{x}_3 && K_4 = x_2x_3x_4\\ \\
    K_5 = x_1x_3x_4 && K_6 = x_1 \bar{x}_2 x_4\\ \\
    K_7 = \bar{x}_1x_2x_4\\
\end{matrix}
$

\medskip

Тогда изначальная функция Патрика будет выглядеть так:
\[
    (K_2 \lor K_3) \land (K_3 \lor K_7) \land (K_7 \lor K_4) \land (K_4 \lor K_5) \land (K_5 \lor K_6) \land (K_6 \lor K_2)
\] 


Вычислим (упростим) найденную функцию Патрика:
\begin{align*}
    &(K_2 \lor K_3) \land (K_3 \lor K_7) \land (K_7 \lor K_4) \land (K_4 \lor K_5) \land (K_5 \lor K_6) \land (K_6 \lor K_2) =\\
    &= (K_3 \lor K_3K_7 \lor K_2K_3 \lor K_2K_7) \land (K_7K_4 \lor K_7K_5 \lor K_4 \lor K_4K_5) 
    \land (K_5K_6 \lor K_2K_5 \lor K_6 \lor K_6K_2) = \\
    &= (K_3 \lor K_2K_7) \land (K_4 \lor K_5K_7) \land (K_6 \lor K_2K_5) = \\
    &= (K_3K_4 \lor K_3K_5K_7 \lor K_2K_4K_7 \lor K_2K_5K_7) \land (K_6 \lor K_2K_5) =\\
    &= K_3K_4K_6 \lor K_3K_4K_5 \lor K_2K_3K_4 \lor \bcancel{K_3K_5K_6K_7} \lor K_3K_5K_7 \bcancel{\lor K_2K_3K_5L_7} \lor\\
    &\lor \bcancel{K_2K_4K_6K_7} \lor \bcancel{K_2K_4K_5K_7} \lor K_2K_4K_7 \lor \bcancel{K_4K_5K_6K_7} \lor
    K_2K_5K_7 \lor \bcancel{K_2K_5K_7} = \\
    & = K_3K_4K_6 \lor K_3K_4K_5 \lor K_2K_3K_4 \lor K_3K_5K_7 \lor K_2K_4K_7 \lor K_2K_5K_7
\end{align*}


Получаем из вышенайденного
\[
    \underbrace{\bar{x}_3 \bar{x}_4}_{\text{Ядро}} \lor
    \underbrace{
        \begin{cases}
            \bar{x}_1 \bar{x}_3 \lor x_2x_3x_4 \lor x_1\bar{x}_2x_4\\
            \bar{x}_1 \bar{x}_3 \lor x_2x_3x_4 \lor x_1x_3x_4\\
            \bar{x}_1 \bar{x}_3 \lor x_2x_3x_4 \lor \bar{x}_2 \bar{x}_3\\
            \bar{x}_1 \bar{x}_3 \lor x_1x_3x_4 \lor \bar{x}_1x_2x_4\\
            \bar{x}_2\bar{x}_3 \lor x_2x_3x_4 \lor \bar{x}_1x_2x_4\\
            \bar{x}_2\bar{x}_3 \lor x_1x_3x_4 \lor \bar{x}_1x_2x_4
        \end{cases}}_{\text{Тупиковые ДНФ}}
\] 


\subsection{Минимальная ДНФ}
В итоге можем получить минимальную ДНФ:
\[
    \boxed{ \underbrace{\bar{x}_3\bar{x}_4}_{\text{Ядро}}
    \lor \underbrace{\bar{x}_2\bar{x}_3}_{K_2} \lor \underbrace{\bar{x}_1 \bar{x}_3}_{K_3} \lor \underbrace{x_2x_3x_4}_{K_4}}
\]

\chapter{Задача 2}
\section{Условие}

Даны функции $f$ (таблица 2) и $w$ (таблица 3).

а) Вычислить таблицу значений функции $f$. б) Найти минимальные ДНФ функций $f$ и $w$.

в) Выяснить полноту системы $\{f, w\}$. Если система не полна, дополнить систему функцией
$g$ до полной системы.

\emph{Указание.} Запрещается дополнять систему константами, отрицанием и базовыми функциями
двух переменных ($\oplus, \lor, \land, |, \downarrow$ и т.д.)
Не допускается дополнение функцией, образующей
с $f$ или $w$ полную подсистему, кроме случаев, когда иное невозможно.

г) Из функциональных элементов, реализующих функции полной системы $\{f, w\}$ или $\{f, w, g\}$,
построить функциональные элементы, реализующие базовые функции $(\lor, \land, \overline{\phantom{A}},
0, 1)$.

\medskip

\begin{center}
    
\begin{matrix}
    \text{Функция $f$:} && \text{Вектор значений функции $w$:}\\
     \boxed{(x_3 \Rightarrow (x_2 \sim \bar{x}_3)) \lor (x_1 \oplus \bar{x}_2) \oplus x_1x_2}
                        && \boxed{w(1101\text{  }0100)}
\end{matrix}
\end{center}

\section{Решение}
\subsection{Нахождение таблицы значений функции $f$}
$
     f = \underbrace{(x_3 \Rightarrow (x_2 \sim \bar{x}_3))}_{B} \lor
     \underbrace{(x_1 \oplus \bar{x}_2) \oplus x_1x_2}_{A}
$


\begin{center}
    \begin{tabular}{||c | c | c | c | c | c | c | c | c||}
        \hline
        x_1 & x_2 & x_3 & x_1x_2 & x_1\oplus \bar{x}_2 & A & x_2\sim \bar{x}_3 & B & f\\
        \hline\hline
    \end{tabular}
\end{center}


\subsection{Нахождение минимальных ДНФ}
\subsubsection{Минимальная ДНФ функции $f$}

\begin{karnaugh-map}[4][2][1][$X_2X_3$][$X_1$]
\end{karnaugh-map}

\subsubsection{Минимальная ДНФ функции $w$}

\subsection{Выяснение полноты системы}

\end{document}
