\documentclass{report}

\usepackage{amssymb, amsmath, amsthm, amscd}
\usepackage[utf8]{inputenc}
\usepackage[russian]{babel}
\usepackage{bookmark}
\usepackage{wasysym}
\usepackage{import}
\usepackage{caption}
\usepackage{mathrsfs}

\usepackage[makeroom]{cancel}

\usepackage{tikz}
\usetikzlibrary{shapes,shapes.geometric,arrows,positioning,decorations.pathmorphing}

\usepackage{listings}

\usepackage{hyperref}
\hypersetup{
       colorlinks=true,
       linkcolor=blue,
}

\usepackage{geometry}
%\geometry{papersize={15cm, 11in}, left=1.5cm, lmargin=1.5cm,right=2cm, top=2cm, bottom=3cm}
\geometry{a4paper, left=1.5cm, lmargin=1.5cm,right=2cm, top=2cm, bottom=3cm}

\tolerance=1
\emergencystretch=\maxdimen
\hyphenpenalty=10000
\hbadness=10000

\usepackage{graphicx}

\usepackage{titlesec}
\titleformat{\chapter}[display]{\fontsize{17pt}{0pt}\bfseries}{}{-20pt}{}
\titleformat{\subsubsection}[display]{\fontsize{12pt}{0pt}\bfseries}{}{0pt}{}

\newcounter{mylabelcounter}

\makeatletter
\newcommand{\labelText}[2]{%
#1\refstepcounter{mylabelcounter}%
\immediate\write\@auxout{%
    \string\newlabel{#2}{{1}{\thepage}{{\unexpanded{#1}}}{mylabelcounter.\number\value{mylabelcounter}}{}}%
}%
}
\makeatother

\newcommand{\circlesign}[1]{
    \mathbin{
        \mathchoice
        {\buildcircledesign{\displaystyle}{#1}}
        {\buildcircledesign{\textstyle}{#1}}
        {\buildcircledesign{\scriptstyle}{#1}}
        {\buildcircledesign{\scriptscriptstyle}{#1}}
    }
}

\newcommand\buildcircledesign[2]{%
    \begin{tikzpicture}[baseline=(X.base), inner sep=0, outer sep=0]
        \node[draw,circle] (X) {\ensuremath{#1 #2}};
    \end{tikzpicture}%
}

\newcommand{\ozv}{\circlesign{*}}

\theoremstyle{plain}
\newtheorem{theorem}{Теорема}[chapter]
\theoremstyle{definition}
\newtheorem{definition}{Определение}
\newtheorem*{pruf}{Доказательство}

\newenvironment{myproof}[1][\textbf{\textup{Доказательство}}]
{\begin{proof}[#1]}
	%\renewcommand*{\qedsymbol}{\(\blacksquare\)}}
{\end{proof}}

\DeclareMathOperator{\band}{\&}
\DeclareMathOperator{\trim}{\triangle}
\DeclareMathOperator{\step}{\vdash}

\makeatletter
\newcommand*\bigcdot{\mathpalette\bigcdot@{.5}}
\newcommand*\bigcdot@[2]{\mathbin{\vcenter{\hbox{\scalebox{#2}{$\m@th#1\bullet$}}}}}
\makeatother


%\newcommand*\pathto[1]{ \Rightarrow^{*}_{#1} }

\newcommand{\pathto}[1][1]{ \Rightarrow^{*}_{#1} }

\newcommand{\dx}[2]{\frac{d#1}{d#2}}
\newcommand{\pdx}[2]{\frac{\partial#1}{\partial#2}}
\newcommand{\zap}[1]{\reflectbox{$3$}#1 3}

\newcommand{\incfig}[2]{%
    \def\svgwidth{#2\columnwidth}
    \import{./images/}{#1.pdf_tex}
}

\tikzstyle{startstop} = [rectangle, rounded corners, 
minimum width=3cm, 
minimum height=1cm,
text centered, 
draw=black, 
fill=red!30]

\tikzstyle{io} = [trapezium, 
trapezium stretches=true, % A later addition
trapezium left angle=70, 
trapezium right angle=110, 
minimum width=3cm, 
minimum height=1cm, text centered, 
draw=black, fill=blue!30]

\tikzstyle{process} = [rectangle, 
minimum width=3cm, 
minimum height=1cm, 
text centered, 
text width=3cm, 
draw=black, 
fill=orange!30]

\tikzstyle{decision} = [diamond, 
minimum width=3cm, 
minimum height=1cm, 
text centered, 
draw=black, 
fill=green!30]
\tikzstyle{arrow} = [thick,->,>=stealth]


%\newcommand{\void}{\varnothing}
\let\void\varnothing

\renewcommand{\phi}{\varphi}
\renewcommand{\epsilon}{\varepsilon}
\newcommand{\scr}[1]{\mathscr{#1}}



\usepackage{karnaugh-map}

\title{Домашнее задание №2\\ Логика и Теория Алгоритмов}
\author{Козырнов Александр Дмитриевич\\ ИУ7-42Б\\Вариант 6}
\date{\today}

\begin{document}
\maketitle
\tableofcontents
\newpage

\chapter{Задача 1}

\section{Условие}

Для булевой функции f, заданной в таблице 1:

а) найти сокращенную ДНФ; б) найти ядро функции;

в) получить все тупиковые ДНФ и указать, какие из них являются минимальными;

г) на картах Карно указать ядро и покрытия, соответствующие минимальным ДНФ.

\medskip

Сама функция $f$, заданная в виде вектора значений:
 \[
     \boxed{f(1100\text{ }1101\text{ }1101\text{ }1001)}
\] 

\section{Решение}


\subsection{Карты Карно}

%Uncolored Variant. For colored variant make \begin{karnaught-map}[4][4]][1][..][..]
\begin{karnaugh-map}*[4][4][1][$X_3X_4$][$X_1X_2$]
    \minterms{0,1,
                4,5,7,
                8,9,11,
                12,15}
    \implicant{0}{8}
    \implicant{0}{5}
    \implicant{5}{7}
    \implicant{7}{15}
    \implicant{15}{11}
    \implicant{9}{11}
    \implicantedge{0}{1}{8}{9}
\end{karnaugh-map}

\subsection{Ядро функции и сокращенная ДНФ}

Ядром функции будет являться $\bar{x}_3 \bar{x}_4$ 

\medskip

Сокращенная ДНФ:
\[
    \boxed{
\bar{x}_3\bar{x}_4 \lor \bar{x}_2 \bar{x}_3 \lor \bar{x}_1 \bar{x}_3 \lor x_2x_3x_4 \lor x_1x_3x_4 \lor
x_1 \bar{x}_2 x_4 \lor \bar{x}_1 x_2x_4
}
\] 

\subsection{Поиск Тупиковых ДНФ. Функция Патрика}

$
\begin{matrix}
    K_1 = \bar{x}_3 \bar{x}_4 && K_2 = \bar{x}_2 \bar{x}_3\\ \\
    K_3 = \bar{x}_1 \bar{x}_3 && K_4 = x_2x_3x_4\\ \\
    K_5 = x_1x_3x_4 && K_6 = x_1 \bar{x}_2 x_4\\ \\
    K_7 = \bar{x}_1x_2x_4\\
\end{matrix}
$

\medskip

Тогда изначальная функция Патрика будет выглядеть так:
\[
    (K_2 \lor K_3) \land (K_3 \lor K_7) \land (K_7 \lor K_4) \land (K_4 \lor K_5) \land (K_5 \lor K_6) \land (K_6 \lor K_2)
\] 


Вычислим (упростим) найденную функцию Патрика:
\begin{align*}
    &(K_2 \lor K_3) \land (K_3 \lor K_7) \land (K_7 \lor K_4) \land (K_4 \lor K_5) \land (K_5 \lor K_6) \land (K_6 \lor K_2) =\\
    &= (K_3 \lor K_3K_7 \lor K_2K_3 \lor K_2K_7) \land (K_7K_4 \lor K_7K_5 \lor K_4 \lor K_4K_5) 
    \land (K_5K_6 \lor K_2K_5 \lor K_6 \lor K_6K_2) = \\
    &= (K_3 \lor K_2K_7) \land (K_4 \lor K_5K_7) \land (K_6 \lor K_2K_5) = \\
    &= (K_3K_4 \lor K_3K_5K_7 \lor K_2K_4K_7 \lor K_2K_5K_7) \land (K_6 \lor K_2K_5) =\\
    &= K_3K_4K_6 \lor K_3K_4K_5 \lor K_2K_3K_4 \lor \bcancel{K_3K_5K_6K_7} \lor K_3K_5K_7 \bcancel{\lor K_2K_3K_5L_7} \lor\\
    &\lor \bcancel{K_2K_4K_6K_7} \lor \bcancel{K_2K_4K_5K_7} \lor K_2K_4K_7 \lor \bcancel{K_4K_5K_6K_7} \lor
    K_2K_5K_7 \lor \bcancel{K_2K_5K_7} = \\
    & = K_3K_4K_6 \lor K_3K_4K_5 \lor K_2K_3K_4 \lor K_3K_5K_7 \lor K_2K_4K_7 \lor K_2K_5K_7
\end{align*}


Получаем из вышенайденного
\[
    \underbrace{\bar{x}_3 \bar{x}_4}_{\text{Ядро}} \lor
    \underbrace{
        \begin{cases}
            \bar{x}_1 \bar{x}_3 \lor x_2x_3x_4 \lor x_1\bar{x}_2x_4\\
            \bar{x}_1 \bar{x}_3 \lor x_2x_3x_4 \lor x_1x_3x_4\\
            \bar{x}_1 \bar{x}_3 \lor x_2x_3x_4 \lor \bar{x}_2 \bar{x}_3\\
            \bar{x}_1 \bar{x}_3 \lor x_1x_3x_4 \lor \bar{x}_1x_2x_4\\
            \bar{x}_2\bar{x}_3 \lor x_2x_3x_4 \lor \bar{x}_1x_2x_4\\
            \bar{x}_2\bar{x}_3 \lor x_1x_3x_4 \lor \bar{x}_1x_2x_4
        \end{cases}}_{\text{Тупиковые ДНФ}}
\] 


\subsection{Минимальная ДНФ}
В итоге можем получить минимальную ДНФ:
\[
    \boxed{ \underbrace{\bar{x}_3\bar{x}_4}_{\text{Ядро}}
    \lor \underbrace{\bar{x}_2\bar{x}_3}_{K_2} \lor \underbrace{\bar{x}_1 \bar{x}_3}_{K_3} \lor \underbrace{x_2x_3x_4}_{K_4}}
\]

\chapter{Задача 2}
\section{Условие}

Даны функции $f$ (таблица 2) и $w$ (таблица 3).

а) Вычислить таблицу значений функции $f$. б) Найти минимальные ДНФ функций $f$ и $w$.

в) Выяснить полноту системы $\{f, w\}$. Если система не полна, дополнить систему функцией
$g$ до полной системы.

\emph{Указание.} Запрещается дополнять систему константами, отрицанием и базовыми функциями
двух переменных ($\oplus, \lor, \land, |, \downarrow$ и т.д.)
Не допускается дополнение функцией, образующей
с $f$ или $w$ полную подсистему, кроме случаев, когда иное невозможно.

г) Из функциональных элементов, реализующих функции полной системы $\{f, w\}$ или $\{f, w, g\}$,
построить функциональные элементы, реализующие базовые функции $(\lor, \land, \overline{\phantom{A}},
0, 1)$.

\medskip

\begin{center}
    
\begin{matrix}
    \text{Функция $f$:} && \text{Вектор значений функции $w$:}\\
     \boxed{(x_3 \Rightarrow (x_2 \sim \bar{x}_3)) \lor (x_1 \oplus \bar{x}_2) \oplus x_1x_2}
                        && \boxed{w(1101\text{  }0100)}
\end{matrix}
\end{center}

\section{Решение}
\subsection{Нахождение таблицы значений функции $f$}
$$
\boxed{
     f = (x_3 \Rightarrow (x_2 \sim \bar{x}_3)) \lor
 (x_1 \oplus \bar{x}_2) \oplus x_1x_2}
$$


\begin{center}
    \begin{tabular}{||c | c | c | c||}
        \hline
        x_{1} & x_{2} & x_{3} & f \\
        \hline\hline
        0 & 0 & 0 & 1 \\
        \hline
        0 & 0 & 1 & 1 \\
        \hline
        0 & 1 & 0 & 1 \\
        \hline
        0 & 1 & 1 & 0 \\
        \hline
        1 & 0 & 0 & 1 \\
        \hline
        1 & 0 & 1 & 1 \\
        \hline
        1 & 1 & 0 & 0 \\
        \hline
        1 & 1 & 1 & 0\\
        \hline
    \end{tabular}
\end{center}


\subsection{Нахождение минимальных ДНФ}
\subsubsection{Минимальная ДНФ функции $f$}

\begin{karnaugh-map}*[4][2][1][$X_2X_3$][$X_1$]
    \minterms{0,1,2,
                4,5}

    \implicant{0}{5}
    \implicantedge{0}{0}{2}{2}
\end{karnaugh-map}

Так как все склейки находятся в ядре функции, то найденная ДНФ будет минимальной:
\[
    \boxed{\bar{x}_2 \lor \bar{x}_1 \bar{x}_3}
\] 

\subsubsection{Минимальная ДНФ функции $w$}


\begin{karnaugh-map}*[4][2][1][$X_2X_3$][$X_1$]
    \minterms{0,1,3,
                5}

    \implicant{0}{1}
    \implicant{1}{3}
    \implicant{1}{5}
\end{karnaugh-map}

Аналогично функции $f$ функция  $w$ имеет все склейки в ядре, отчего минимальная ДНФ:
 \[
     \boxed{
         \bar{x}_{1} \bar{x}_{2} \lor \bar{x}_{2} x_{3} \lor \bar{x}_{1} x_{3}
     }
\]

\subsection{Выяснение полноты системы}


\begin{center}
    \begin{tabular}{|c | c | c | c | c | c|}
        \hline
        & $T_0$ & $T_1$ & S & M & L \\
        \hline\hline
        $f$ & - & - & - & - & -\\
        \hline
        $w$ & - & - & + & - & -\\
        \hline
    \end{tabular}
\end{center}


Легко понять, что обе функции не сохраняют констант $1$ и  $0$.  $f \not\in S$, то есть
не является самодвойственной, потому что ее набор значений как минимум является несимметричным.
У функции $w$ вектор значений является симметричным (1101 - 0100). То есть $Revers(0100) = 0010$, а
$\overline{0010} = 1101$, что равно левой части. Функция $f$ немонотонна, например есть значение
$010 = 1$ и  $011 = 0$. Обе функции нелинейны, так как их полиномы Жегалкина нелинейны.

\subsubsection{Полином Жегалкина функции f}
$$
\begin{align}
 & f(0,0,0) = a_{0} = 1 & \implies a_{0} = 1 \\
 & f(1,0,0) = a_{0} \oplus a_{1} = 1 & \implies a_{1} = 0  \\
 & f(0,1,0) = a_{0} \oplus a_{2} = 1  & \implies a_{2} = 0 \\
 & f(0,0,1) = a_{0} \oplus a_{3} = 1 & \implies a_{3} = 0 \\
 & f(1,1,0) = a_{0} \oplus\cancel{  a_{1} } \oplus \cancel{ a_{2} } \oplus a_{12} = 0 & \implies a_{12} = 1 \\
 & f(1,0,1) = a_{0} \oplus \cancel{ a_{1} } \oplus \cancel{ a_{3} } \oplus a_{13} = 1 & \implies a_{13} = 0 \\
 & f(0,1,1) = a_{0} \oplus a_{2} \oplus a_{3} \oplus a_{23} = 0  & \implies a_{23} = 1 \\ 
 & f(1,1,1) = a_{0} \oplus \cancel{ a_{1} } \oplus \cancel{ a_{2} } \oplus \cancel{ a_{3} } \oplus a_{12} \oplus \cancel{ a_{13} } \oplus a_{23} \oplus a_{123} = 0  & \implies a_{123} = 1
\end{align}
$$

Полином Жегалкина:
$
\boxed{
x_{1}x_{2} \oplus x_{2}x_{3} \oplus x_{1}x_{2}x_{3} \oplus 1
}
$


\subsubsection{Полином Жегалкина функции w}

$$
\begin{align}
 & w(0,0,0) = a_{0} = 1  & \implies a_{0} = 1 \\
 & w(1,0,0) = a_{0} \oplus a_{1} = 0 & \implies a_{1} = 1 \\
 & w(0,1,0) = a_{0} \oplus a_{2} = 0 & \implies a_{2} = 1 \\
 & w(0,0,1) = a_{0} \oplus a_{3} = 1 & \implies a_{3} = 0 \\
 & w(1,1,0) = a_{0} \oplus a_{1} \oplus a_{2} \oplus a_{12} = 0 & \implies a_{12} = 1 \\
 & w(1,0,1) = a_{0} \oplus a_{1} \oplus \cancel{ a_{3} } \oplus a_{13} =1  & \implies a_{13} = 1 \\
 & w(0,1,1) = a_{0} \oplus a_{2} \oplus \cancel{ a_{3} } \oplus a_{23} = 1  & \implies a_{23} = 1 \\
 & w(1,1,1) = a_{0} \oplus a_{1} \oplus a_{2} \oplus \cancel{ a_{3} } \oplus a_{12} \oplus a_{13} \oplus a_{23} \oplus a_{123} = 0 & \implies a_{123} = 0
\end{align}
$$

Полином Жегалкина:
$
\boxed{
x_{1} \oplus x_{2} \oplus x_{1}x_{2} \oplus x_{1}x_{3} \oplus x_{2}x_{3} \oplus 1
}
$

\subsection{Построение функциональных элементов, образующих базовые функции}

\paragraph*{Конъюнкция} ${}$ \newline

$f(x_1,x_2,0) = x_1x_2 \oplus 1$ - штрих Шеффера (отрицание конъюнкции)

\medskip

Отсюда конъюнкция - это $x_1x_2 = \overline{f(x_1,x_2,0)}$ 


\paragraph*{Дизъюнкция} ${}$ \newline

$x_1 \lor x_2 = \overline{x_1x_2} = f(\bar{x}_1, \bar{x}_2, 0)$ 

\paragraph*{Константа 0 и 1. Отрицание} ${}$ \newline

$
1 = f(1,0,x) \quad
0 = w(1,1,x) \quad
\overline{x} = w(x,x,x)
$

\chapter{Задача 3}

\section{Условие}

Доказать в исчислении высказываний (буквы обозначают произвольные
формулы):
\[
    \boxed{
        \lnot((X\band Y)\band \lnot Z) \step (\lnot(\lnot X \to \lnot Y) \lor (Y \to Z))
    }
\] 


\section{Решение}

Нам известно, что
\[
\begin{matrix}
    A\band B = \lnot(A \to \lnot B) && A \lor B = \lnot A \to B
\end{matrix}
\] 
Перепишем формулу:
\[
\lnot\lnot(\lnot(X \to \lnot Y) \to \lnot\lnot Z) \step \lnot\lnot(\lnot X \to \lnot Y) \to (Y \to Z)
\] 


Доказательство:

1) $\lnot\lnot(\lnot(X \to \lnot Y) \to \lnot\lnot Z)$ - Гипотеза

2) $\lnot\lnot(\lnot(X \to \lnot Y) \to \lnot\lnot Z) \to (\lnot(\lnot X \to \lnot Y) \to \lnot\lnot Z)$ -
секвенция 3 при $A := \lnot(X \to \lnot Y) \to \lnot\lnot Z$

3) $\lnot(\lnot X \to \lnot Y) \to \lnot\lnot Z$ - modus ponens, (1) и (2)

4) $\lnot\lnot(\lnot X \to \lnot Y)$ - Гипотеза

5) $\lnot\lnot(\lnot X \to \lnot Y) \to (\lnot X \to \lnot Y)$ - секвенция 3 при $A := 
(\lnot X \to \lnot Y)$

6) $\lnot X \to \lnot Y$ - modus ponens, (4) и (5)

7) $Y$ - Гипотеза

8) $(\lnot X \to \lnot Y) \to (Y \to X)$ - секвенция 6 при $ A:= Y, B := X$

9)  $Y \to X$ - modus pomems, (6) и (8)

10) $X$ - modus ponens, (7) и (9)

11)  $X \to (\lnot \lnot Y\to \lnot(X \to \lnot Y))$ - секвенция 9 при $A:=X, B:=\lnot Y$

12) $\lnot \lnot Y\to \lnot(X \to \lnot Y)$ - modus ponens, (10) и (11)

13)  $Y \to \lnot\lnot Y$ - секвенция 4 при $A:=Y$

14)  $Y \to \lnot(X \to \lnot Y)$ - секвенция 1, (12) и (13) при $A:=Y,B:=\lnot\lnot Y,
C:=\lnot(X \to \lnot Y)$

15) $\lnot(X \to \lnot Y)$ - modus ponens, (7) и (14)

16) $\lnot\lnot Z$ - modus ponens, (3) и (15)

17)  $\lnot\lnot Z \to Z$ - секвенция 3 при $A := Z$

18)  $Z$ - modus ponens, (16) и (17)

\end{document}
