\documentclass{report}

\usepackage{amssymb, amsmath, amsthm, amscd}
\usepackage[utf8]{inputenc}
\usepackage[russian]{babel}
\usepackage{bookmark}

\usepackage{tikz}
\usetikzlibrary{shapes,arrows,positioning,decorations.pathmorphing}

\usepackage{listings}

\usepackage{hyperref}
\hypersetup{
       colorlinks=true,
       linkcolor=blue,
}

\usepackage{geometry}
\geometry{papersize={15cm, 11in}, left=1.5cm, lmargin=1.5cm,right=2cm, top=2cm, bottom=3cm}

\usepackage{graphicx}

\usepackage{titlesec}
\titleformat{\chapter}[display]{\fontsize{17pt}{0pt}\bfseries}{}{-20pt}{}
\titleformat{\subsubsection}[display]{\fontsize{12pt}{0pt}\bfseries}{}{0pt}{}

\newcounter{mylabelcounter}

\makeatletter
\newcommand{\labelText}[2]{%
#1\refstepcounter{mylabelcounter}%
\immediate\write\@auxout{%
    \string\newlabel{#2}{{1}{\thepage}{{\unexpanded{#1}}}{mylabelcounter.\number\value{mylabelcounter}}{}}%
}%
}
\makeatother

\newcommand{\bslash}{\mbox{ } \backslash \mbox{ }}
\newcommand{\band}{\mbox{ } \& \mbox{ }}

%\newcommand*\pathto[1]{ \Rightarrow^{*}_{#1} }

\newcommand{\pathto}[1][1]{ \Rightarrow^{*}_{#1} }

\renewcommand{\phi}{\varphi}


\title{}
\author{Козырнов Александр Дмитриевич, ИУ7-32Б}
\date{\today}

\begin{document}
Гедаль, а не бедаль!!!

\subsection{Теорема Дедукции для ИП1}

\begin{theorem}
    (Дедукции для ИП1). Если $\Gamma, A\step B$, причем существует такой вывод формулы  $B$ из
    множества формул  $\Gamma \cup \{A\} $, в котором ни при каком применении правила
    $Gen$ к формулам, зависящем в этом
    выводе от формулы  $A$, не связывается квантором никакая свободная переменная формулы  $A$,
    то  $\Gamma \step A \to B$.
\end{theorem}


В ИП эквивалентность отличается отэквивалентности в исчислении выражений:
$$\Phi \equiv \Psi \leftrightharpoons \step(\Phi \to \Psi)\band(\Psi\to \Phi)$$

\subsection{Некоторые дополнительные правила}

Одно мы уже знаем (A4):
\[
    \frac{(\forall x_{i})A(x_{i})}{A(t)} \text{ при } Free(t,x_{i},A)
\] 
Вот еще одно схожее (E4):
\[
    \frac{A(t)}{(\exists x_{i})A(x_{i})} \text{ при } Free(t,x_{i},A)
\] 
Правило выбора (C):
\[
\frac{(\exists x)A(x)}{A(b)}
\] 

\section{Теории первого порядка}

Аксиомы теории первого порядка имеет две части:
\begin{enumerate}
    \item Логически общезначимые формулы (ИП1)
    \item Нелогические аксиомы (это такие, к-рые не являются общезначимыми, но верны в широком классе
        интерпретации)
\end{enumerate}

\begin{definition}
Любая интерпретация, в которой верна нелогическая аксиома, называется моделью.
\end{definition}


\end{document}
