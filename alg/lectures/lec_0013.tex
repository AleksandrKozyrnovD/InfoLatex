\documentclass{report}

\usepackage{amssymb, amsmath, amsthm, amscd}
\usepackage[utf8]{inputenc}
\usepackage[russian]{babel}
\usepackage{bookmark}

\usepackage{tikz}
\usetikzlibrary{shapes,arrows,positioning,decorations.pathmorphing}

\usepackage{listings}

\usepackage{hyperref}
\hypersetup{
       colorlinks=true,
       linkcolor=blue,
}

\usepackage{geometry}
\geometry{papersize={15cm, 11in}, left=1.5cm, lmargin=1.5cm,right=2cm, top=2cm, bottom=3cm}

\usepackage{graphicx}

\usepackage{titlesec}
\titleformat{\chapter}[display]{\fontsize{17pt}{0pt}\bfseries}{}{-20pt}{}
\titleformat{\subsubsection}[display]{\fontsize{12pt}{0pt}\bfseries}{}{0pt}{}

\newcounter{mylabelcounter}

\makeatletter
\newcommand{\labelText}[2]{%
#1\refstepcounter{mylabelcounter}%
\immediate\write\@auxout{%
    \string\newlabel{#2}{{1}{\thepage}{{\unexpanded{#1}}}{mylabelcounter.\number\value{mylabelcounter}}{}}%
}%
}
\makeatother

\newcommand{\bslash}{\mbox{ } \backslash \mbox{ }}
\newcommand{\band}{\mbox{ } \& \mbox{ }}

%\newcommand*\pathto[1]{ \Rightarrow^{*}_{#1} }

\newcommand{\pathto}[1][1]{ \Rightarrow^{*}_{#1} }

\renewcommand{\phi}{\varphi}


\title{}
\author{Козырнов Александр Дмитриевич, ИУ7-32Б}
\date{\today}

\begin{document}
\begin{theorem}
Каждый класс Поста замкнут.
\end{theorem}

\section{Теорема поста}

\begin{theorem}
Множество булевых функций полно тогда и только тогда, когда оно не содержится (целиком)
ни в одном из классов Поста.
\end{theorem}

\begin{myproof}
Необходимость. Полагая, что множество булевых функций содержится в каком-то классе Поста,
получим, в силу замкнутости каждого класса Поста, что формулами над этим множеством могут быть
представлены только функции этого класса, а, стало быть, не может быть представлена ни одна функция,
не содержащаяся ни в одном из классов Поста, например, штрих Шеффера. Значит, такое множество не может
быть полным.

Достаточность. Достаточно показать, что формулами над множеством ${\cal F}$, удовлетворяющем
условию теоремы, могут быть представлены функции какого-то уже известного полного множества. В
качестве такого множества можно взять такое, состоящее из конъюнкции и дизъюнкции.

Так как множество $\{*, \overline{\phantom{A}}\} $ является полным, достаточно указать способ
построения формул для конъюнкции и отрицания над базисом ${\cal F}$, который удовлетворяет
условию теоремы Поста, то есть не содержится ни в одном из классов Поста, что
можно выразить следующим образом:
\[
    (\forall C \in \{T_0,T_1,S,M,L\} )(\exists f_c \in F\setminus C)
\]
1 случай)
Представим константу 1:
\[
1 = f_0(x,\ldots,x),
\]
а константу 0 представим с использованием какой-нибудь функции $g_1 \in F\setminus T_1:$ 
\[
0 = g(1,\ldots,1) = g(f(x,\ldots,x), \ldots, f(x,\ldots,x))
\]
Имея формулы для обеих констант, отрицание представим формулой, используя немонотонную
функцию.

\medskip

2 случай) Всякая функция $f_0 \in F \setminus T_0$ не сохраняет и константу 1, а всякая
функция $f_1 \in F \setminus T_1$ не сохраняет и константу 0. В этом случае
сразу получаем формулу для отрицания.
\[
\overline{x} = f_0(x,\ldots,x)
\] 

Тут используется лемма о несамодвойственной функции.
\end{myproof}

\chapter{Элементы математической логики}
\section{Предпосылки возникновения математической логики}

Пример Гиберта.

$
Y = \{x: |x| \ge 3\} \quad \text{x - множество}
$

То есть возьмем такие примеры и получим:
\[
\{1,2,3\} \in Y, \{1,2,3,4\} \in Y, \{1,2,3,4,5\} \in Y \implies Y \in Y   
\] 

\begin{definition}
Нормальные множества - это такие множества, которые не содержат самих себя.
\end{definition}

Пусть мы хотим найти все Нормальные множества:
$Z = \{x: x \not\in x\} \quad Z \not\in Z \implies Z \in Z \implies Z \not\in Z$.

Это называется парадокс Рассела.

\section{Понятие формальной аксиоматической теории}
\begin{definition}
${\cal T} = ( \underbrace{V}_{\text{алфавит}}, \underbrace{{\cal F}}_{\text{формулы}}, 
\underbrace{{\cal A} \subseteq {\cal F}}_{\text{Мн. аксиом}},
\underbrace{{\cal P}}_{\text{Мн. правил вывода}})$ называется теорией.
\end{definition}

\begin{definition}
Фиксируется некоторое множество $\Gamma \subseteq {\cal F}$ - гипотеза.
Среди гипотез нет ни одной аксиомы: $\Gamma \cap {\cal A} = \void$.
\end{definition}

\begin{definition}
Вывод теории ${\cal T}$ из множества гипотез $\Gamma$ - это последовательность формул (конечная или
бесконечная):  $\theta_0,\theta_1,\ldots,\theta_n,\ldots, \quad n\ge 0$, где для каждого
 $\forall i \ge 0:$ 1) $\theta_i \in \Gamma$, 2) $\theta_i \in {\cal A}$, 3)
 существует правило вывода в ${\cal P}$ : $\frac{\theta_{j 1}\ldots \theta_{jm}}{\theta_i}$, где
 $j_1,\ldots,j_m < j$.
\end{definition}

Если $\Phi = \theta_i$, то  $\Gamma \step_{{\cal T}} \Phi$. Если $\Gamma = \void$, то
пишем  $\step_{{\cal T}} \Phi$.

\begin{theorem}
Если формула $\Phi$ выводима из гипотезы ($\Gamma \step_{{\cal T}} \Phi$), то для
любого $\Gamma' \supset \Gamma$ верно $\Gamma' \step_{{\cal T}} \Phi$.
\end{theorem}

\paragraph*{Следствие.}
Если $\step_{{\cal T}}\Phi$, то для любого $\Gamma: \Gamma \step_{{\cal T}} \Phi$.


\end{document}
