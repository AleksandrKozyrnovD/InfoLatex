\documentclass{report}

\usepackage{amssymb, amsmath, amsthm, amscd}
\usepackage[utf8]{inputenc}
\usepackage[russian]{babel}
\usepackage{bookmark}
\usepackage{wasysym}
\usepackage{import}
\usepackage{caption}
\usepackage{mathrsfs}

\usepackage[makeroom]{cancel}

\usepackage{tikz}
\usetikzlibrary{shapes,shapes.geometric,arrows,positioning,decorations.pathmorphing}

\usepackage{listings}

\usepackage{hyperref}
\hypersetup{
       colorlinks=true,
       linkcolor=blue,
}

\usepackage{geometry}
%\geometry{papersize={15cm, 11in}, left=1.5cm, lmargin=1.5cm,right=2cm, top=2cm, bottom=3cm}
\geometry{a4paper, left=1.5cm, lmargin=1.5cm,right=2cm, top=2cm, bottom=3cm}

\tolerance=1
\emergencystretch=\maxdimen
\hyphenpenalty=10000
\hbadness=10000

\usepackage{graphicx}

\usepackage{titlesec}
\titleformat{\chapter}[display]{\fontsize{17pt}{0pt}\bfseries}{}{-20pt}{}
\titleformat{\subsubsection}[display]{\fontsize{12pt}{0pt}\bfseries}{}{0pt}{}

\newcounter{mylabelcounter}

\makeatletter
\newcommand{\labelText}[2]{%
#1\refstepcounter{mylabelcounter}%
\immediate\write\@auxout{%
    \string\newlabel{#2}{{1}{\thepage}{{\unexpanded{#1}}}{mylabelcounter.\number\value{mylabelcounter}}{}}%
}%
}
\makeatother

\newcommand{\circlesign}[1]{
    \mathbin{
        \mathchoice
        {\buildcircledesign{\displaystyle}{#1}}
        {\buildcircledesign{\textstyle}{#1}}
        {\buildcircledesign{\scriptstyle}{#1}}
        {\buildcircledesign{\scriptscriptstyle}{#1}}
    }
}

\newcommand\buildcircledesign[2]{%
    \begin{tikzpicture}[baseline=(X.base), inner sep=0, outer sep=0]
        \node[draw,circle] (X) {\ensuremath{#1 #2}};
    \end{tikzpicture}%
}

\newcommand{\ozv}{\circlesign{*}}

\theoremstyle{plain}
\newtheorem{theorem}{Теорема}[chapter]
\theoremstyle{definition}
\newtheorem{definition}{Определение}
\newtheorem*{pruf}{Доказательство}

\newenvironment{myproof}[1][\textbf{\textup{Доказательство}}]
{\begin{proof}[#1]}
	%\renewcommand*{\qedsymbol}{\(\blacksquare\)}}
{\end{proof}}

\DeclareMathOperator{\band}{\&}
\DeclareMathOperator{\trim}{\triangle}
\DeclareMathOperator{\step}{\vdash}

\makeatletter
\newcommand*\bigcdot{\mathpalette\bigcdot@{.5}}
\newcommand*\bigcdot@[2]{\mathbin{\vcenter{\hbox{\scalebox{#2}{$\m@th#1\bullet$}}}}}
\makeatother


%\newcommand*\pathto[1]{ \Rightarrow^{*}_{#1} }

\newcommand{\pathto}[1][1]{ \Rightarrow^{*}_{#1} }

\newcommand{\dx}[2]{\frac{d#1}{d#2}}
\newcommand{\pdx}[2]{\frac{\partial#1}{\partial#2}}
\newcommand{\zap}[1]{\reflectbox{$3$}#1 3}

\newcommand{\incfig}[2]{%
    \def\svgwidth{#2\columnwidth}
    \import{./images/}{#1.pdf_tex}
}

\tikzstyle{startstop} = [rectangle, rounded corners, 
minimum width=3cm, 
minimum height=1cm,
text centered, 
draw=black, 
fill=red!30]

\tikzstyle{io} = [trapezium, 
trapezium stretches=true, % A later addition
trapezium left angle=70, 
trapezium right angle=110, 
minimum width=3cm, 
minimum height=1cm, text centered, 
draw=black, fill=blue!30]

\tikzstyle{process} = [rectangle, 
minimum width=3cm, 
minimum height=1cm, 
text centered, 
text width=3cm, 
draw=black, 
fill=orange!30]

\tikzstyle{decision} = [diamond, 
minimum width=3cm, 
minimum height=1cm, 
text centered, 
draw=black, 
fill=green!30]
\tikzstyle{arrow} = [thick,->,>=stealth]


%\newcommand{\void}{\varnothing}
\let\void\varnothing

\renewcommand{\phi}{\varphi}
\renewcommand{\epsilon}{\varepsilon}
\newcommand{\scr}[1]{\mathscr{#1}}



\title{}
\author{Козырнов Александр Дмитриевич, ИУ7-32Б}
\date{\today}

\begin{document}

\begin{theorem}
Язык является контекстно свободным тогда и только тогда, когда он допускается некоторым
МП-автоматом.
\end{theorem}

Дано: КС-грамматика ${\cal J} = (V,N,S,\scr{P})$ 

Строим: МП-автомат ${\cal M} = (Q,V,\Gamma, q_0, F, z_0, \delta)$

\boxed{$L({\cal M}) = L(\scr{J})$}

\medskip

${\cal M} = (\{q\}, V, V \cup N, q, \{q\}, S, \delta_{\scr{P}})$

Причем
$q\lambda A \to q\alpha \in \delta_{\scr{P}} \leftrightharpoons A \to \alpha \in \scr{P}$

\medskip

$(\forall a \in V)(qaa \to q\lambda \in \delta_{\scr{P}})$

\paragraph*{Пример 1.} ${}$ \newline

${\cal J}:\quad S \to aSa \big| bSb \big| aa \big| bb \big| a \big| b \big|$ 

То есть $L({\cal J}) = \{x: x = x^{R}, x \neq \lambda\} $

То есть система комманд такая:
\[
\delta_{\scr{P}}: \begin{cases}
    q\lambdaS \to qaSa \big| qbSb \big| qaa \big| qbb \big|qa \big|qb \\
    qaa \to q\lambda\\
    qbb \to q\lambda
\end{cases}
\] 

${\cal J}: S \step aSa \step abSba \step ababa$ 

Для автомата:

$
(q, ababa, S) \step (q, ababa, aSa) \step (q, baba, Sa) \step (q, baba, bSba) \step
(q,aba,Sba) \step (q, aba, aba) \models^{3} (q,\lambda,\lambda)
$ - допуск

\paragraph*{Пример 2.} ${}$ \newline

$S \to ab \big| aSb \big| SS$

\[
\delta: \begin{cases}
    qaS \to qb\big| qsb\\
    q\lambda S \to qSS\\
    qaa \to q\lambda\\
    qbb \to q\lambda
\end{cases}
\] 

$S \step SS \step aSbS \step aabbS \step aabbab$

Как автомат ее разберет:

$
(q,aabbab,S) \step (q, aabbab, SS) \step (q, abbab, SbS) \step (q, bbab, bbS) \models^{2}
(q,ab,S) \step (q, b, b) \step (q,\lambda,\lambda)
$ - допуск

\chapter{Булевы функции}

\section{Булева алгебра}

Свойства симметричного полукольца:
\begin{itemize}
    \item a + (b + c) = (a + b) + c
    \item a + b = b + a
    \item a + a = a
    \item a + 0 = a
    \item a * (b * c) = (a * b) * c
    \item a * 1 = 1 * a =a
    \item a*(b+c) = ab + ac
    \item a*0=0*b=0
    \item ab = ba
    \item aa = a
    \item a + 1 = 1
    \item a + bc = (a + b)(a + c)
\end{itemize}

Симметричное полукольцо:
$\scr{S} = (S, +, \cdot , 0 , 1)$ 

Симметричное ему полукольцо:
$\scr{S}^{*} = (S, \cdot , +, 1, 0)$

$(\forall a)(a^{*} = 1)$

\paragraph*{Принцип двойственности симметрического полукольца.}

Любое тождество, доказанное для симметрического полукольца, останется справедливым, если
в нем произвести взаимные замены операции сложения и умножения, а также взаимные замены
нуля и единицы.

\medskip

\paragraph*{Пример.} ${}$ \newline

$(a + b)(a + c) = a^2 + ac + ab + bc = a + ac + ab + bc = a \underbrace{(1 + c + b)}_{1}
+ bc = a + bc$ 

\paragraph*{Свойство 1.} $a + ab = a(a + b) = a$
\begin{myproof}
$a(a+b) = a^2 + ab = a + ab = a(1+b) = a*1 = a$
\end{myproof}

\paragraph*{Свойство 2.} $a \le b \Longleftrightarrow ab = a$
\begin{myproof} ${}$\newline 
$a \le b \implies a + b = b \implies ab = a(a + b) = a$
\newline
$ab = a \implies a + b = ab + b = ab + 1*b = (a+1)b = 1*b = b$
\end{myproof}

\paragraph*{Свойство 3.} $(\forall a)(a \le 1)$, то есть $(\forall a)(0 \le a \le 1)$


\begin{definition}
Дополнение элемента $a$: $\overline{a} * a = 0$ и $\overline{a} + a = 1$
\end{definition}

\begin{theorem}
Если дополнение элемента симметрического полукольца определено, то оно определено
однозначно.
\end{theorem}
\begin{myproof}
Пусть $(\exists x)(a + x = 1, ax = 0)$

Тогда

$$
x = x + a*\overline{a} = (x + a)(x + \overline{a}) = 1(x + \overline{a}) = 
(a + \overline{a})(x + \overline{a}) = ax + \overline{a} = 0 + \overline{a} = \overline{a}
$$
\end{myproof}

\paragraph*{Следствие.}
$\overline{\overline{a}} = a$

\begin{definition}
Булева алгебра - это симметричное полукольцо, в котором каждый элемент имеет дополнение.
\end{definition}

\paragraph*{Примеры.} ${}$ \newline

${\cal B} = (\{0,1\}, +, *, 0, 1 )$ 

${\cal S}_M = (2^{M}, \cup, \cap, \void, M)$

\medskip

Булева алгебра обозначается так:
\[
    \scr{D} = (B, \lor, \land, \Theta, I, bar sdelat)
\] 

\begin{theorem}
В любой булевой алгебре имеет место:
\[
\overline{a \lor b} = \overline{a} \land \overline{b}; \quad
\overline{a \land b} = \overline{a} \lor \overline{b}
\]
\end{theorem}
\begin{myproof}
    \[
        (a \lor b) \lor (\overline{a} \land \overline{b}) = (a \lor b \lor \overline{a}) \land
        (a \lor b \lor \overline{b}) = I
    \]
    \[
        (a \lor b) \land (\overline{a} \land \overline{b}) = (\overline{a} \land \overline{b}
        \land a) \lor (\overline{a} \land \overline{b} \land b) = \Theta \lor \Theta = \Theta
    \] 

    Отсюда $\overline{a \lor b} = \overline{a} \lor \overline{b}$
\end{myproof}


\end{document}
