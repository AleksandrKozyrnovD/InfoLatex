\documentclass{report}

\usepackage{amssymb, amsmath, amsthm, amscd}
\usepackage[utf8]{inputenc}
\usepackage[russian]{babel}
\usepackage{bookmark}
\usepackage{wasysym}
\usepackage{import}
\usepackage{caption}
\usepackage{mathrsfs}

\usepackage[makeroom]{cancel}

\usepackage{tikz}
\usetikzlibrary{shapes,shapes.geometric,arrows,positioning,decorations.pathmorphing}

\usepackage{listings}

\usepackage{hyperref}
\hypersetup{
       colorlinks=true,
       linkcolor=blue,
}

\usepackage{geometry}
%\geometry{papersize={15cm, 11in}, left=1.5cm, lmargin=1.5cm,right=2cm, top=2cm, bottom=3cm}
\geometry{a4paper, left=1.5cm, lmargin=1.5cm,right=2cm, top=2cm, bottom=3cm}

\tolerance=1
\emergencystretch=\maxdimen
\hyphenpenalty=10000
\hbadness=10000

\usepackage{graphicx}

\usepackage{titlesec}
\titleformat{\chapter}[display]{\fontsize{17pt}{0pt}\bfseries}{}{-20pt}{}
\titleformat{\subsubsection}[display]{\fontsize{12pt}{0pt}\bfseries}{}{0pt}{}

\newcounter{mylabelcounter}

\makeatletter
\newcommand{\labelText}[2]{%
#1\refstepcounter{mylabelcounter}%
\immediate\write\@auxout{%
    \string\newlabel{#2}{{1}{\thepage}{{\unexpanded{#1}}}{mylabelcounter.\number\value{mylabelcounter}}{}}%
}%
}
\makeatother

\newcommand{\circlesign}[1]{
    \mathbin{
        \mathchoice
        {\buildcircledesign{\displaystyle}{#1}}
        {\buildcircledesign{\textstyle}{#1}}
        {\buildcircledesign{\scriptstyle}{#1}}
        {\buildcircledesign{\scriptscriptstyle}{#1}}
    }
}

\newcommand\buildcircledesign[2]{%
    \begin{tikzpicture}[baseline=(X.base), inner sep=0, outer sep=0]
        \node[draw,circle] (X) {\ensuremath{#1 #2}};
    \end{tikzpicture}%
}

\newcommand{\ozv}{\circlesign{*}}

\theoremstyle{plain}
\newtheorem{theorem}{Теорема}[chapter]
\theoremstyle{definition}
\newtheorem{definition}{Определение}
\newtheorem*{pruf}{Доказательство}

\newenvironment{myproof}[1][\textbf{\textup{Доказательство}}]
{\begin{proof}[#1]}
	%\renewcommand*{\qedsymbol}{\(\blacksquare\)}}
{\end{proof}}

\DeclareMathOperator{\band}{\&}
\DeclareMathOperator{\trim}{\triangle}
\DeclareMathOperator{\step}{\vdash}

\makeatletter
\newcommand*\bigcdot{\mathpalette\bigcdot@{.5}}
\newcommand*\bigcdot@[2]{\mathbin{\vcenter{\hbox{\scalebox{#2}{$\m@th#1\bullet$}}}}}
\makeatother


%\newcommand*\pathto[1]{ \Rightarrow^{*}_{#1} }

\newcommand{\pathto}[1][1]{ \Rightarrow^{*}_{#1} }

\newcommand{\dx}[2]{\frac{d#1}{d#2}}
\newcommand{\pdx}[2]{\frac{\partial#1}{\partial#2}}
\newcommand{\zap}[1]{\reflectbox{$3$}#1 3}

\newcommand{\incfig}[2]{%
    \def\svgwidth{#2\columnwidth}
    \import{./images/}{#1.pdf_tex}
}

\tikzstyle{startstop} = [rectangle, rounded corners, 
minimum width=3cm, 
minimum height=1cm,
text centered, 
draw=black, 
fill=red!30]

\tikzstyle{io} = [trapezium, 
trapezium stretches=true, % A later addition
trapezium left angle=70, 
trapezium right angle=110, 
minimum width=3cm, 
minimum height=1cm, text centered, 
draw=black, fill=blue!30]

\tikzstyle{process} = [rectangle, 
minimum width=3cm, 
minimum height=1cm, 
text centered, 
text width=3cm, 
draw=black, 
fill=orange!30]

\tikzstyle{decision} = [diamond, 
minimum width=3cm, 
minimum height=1cm, 
text centered, 
draw=black, 
fill=green!30]
\tikzstyle{arrow} = [thick,->,>=stealth]


%\newcommand{\void}{\varnothing}
\let\void\varnothing

\renewcommand{\phi}{\varphi}
\renewcommand{\epsilon}{\varepsilon}
\newcommand{\scr}[1]{\mathscr{#1}}



\title{}
\author{Козырнов Александр Дмитриевич, ИУ7-42Б}
\date{\today}

\begin{document}
\paragraph*{Примеры использования теоремы сочетания.}

\begin{itemize}
    \item[1)] Проекцирующие НА

        Дано $V, \$ \not\in V$. Векторное слово в алфавите
        $V: x_1\$x_2\$\ldots\$x_{n}, n\ge 1$, где
        $(\forall i = \overline{1,n})(x_{i} \in V^{*})$ 

        Нужен алгоритм, который вычисляет его $x_{i}$

        $\prod_{i}(x_1\$x_2\$\ldots\$x_{n}) = x_{i},\quad i = 1\ldots n$

        \[
        \scr{P}_{1}: \begin{cases}
            \$\eta \to $\text{ //}\eta \in V\\
            \$ \to \\
            \to \bigcdot\\
        \end{cases}
        
        Результат работы $\scr{P}_{1}(x_1\$x_2\$\ldots\$x_{n}) = x_1$ 
        \]

        \[
        \scr{P}_{2}: \begin{cases}
            \eta\3 \to \#\text{ //}\eta \in V,\# \not\in V\\
            \# \to \bigcdot\\
            \$ \to \#\\
        \end{cases}
        \] 
        
        То есть $\scr{P}_{2}(x_1\$x_2\$\ldots\$x_{n}) = x_2\$\ldots\$x_{n}$ 

        Получаем $\prod_{i} = \scr{P}_{1} \circ \scr{P}_{2}^{i-1}, \quad 1 \le i \le n$ 

        i = 1: $\scr{P}_{2}^{i-1} = \scr{P}_{2}^{0} = {\cal J}\alpha$ 

        i = n: $\scr{P}_{2}^{n-1}(x_1\$x_2\$\ldots\$x_{n}) = x_{w};\quad \scr{P}_{1}(x_{n}) = x_{n}$ 
        
        \item[2)] НА распознавания равенства слов
            
            $EQ(x\$y) = \lambda \Longleftrightarrow x = y; \quad x,y \in V^{*}, \$ \not\in V$ 

            $EQ(x\$y) \simeq Comp({\cal J}\alpha\${\cal I}nv(y))$ 

            ${\cal I}nv(y) = y^{R}$ 

            \[
            Comp: \begin{cases}
                \eta\$\eta \to \$\text{ //}\eta \in V\\
                \$ \to \bigcdot\\
            \end{cases}
            \]

            $x^{R} = (x(1)x(2)\ldots x(k))^{R} = x(k)\ldots x(2)x(1)$
            

        \item[3)] НА определения центра слова

            $\scr{C}(x) = x_1\$x_2$, где $x_1x_2 = x,  \quad \left| |x_1| - |x_2| \right| \le 1$,
            $x \in V^{*}; \quad \$ \not\in V$

            $\scr{C} = \scr{B} \circ {}_{\scr{A}}\langle L \circ R \rangle$ 

            \[
            L: \begin{cases}
                \alpha\beta \to \bigcdot\alpha\beta\\
                \alpha\xi \to \bigcdot \xi\alpha\text{ //}\xi \in V, \alpha \not\in V\\
                \to \alpha\\
            \end{cases}
            \] 
            \[
            R: \begin{cases}
                \gamma\xi \to \xi\gamma\text{ //}\xi \in V; \beta,\gamma \not\in V\\
                \xi\gamma \to \bigcdot\beta\xi\\
                \xi\beta \to \bigcdot\beta\xi\\
                \to \gamma\\
            \end{cases}
            \]
            \[
            \scr{A}: \begin{cases}
                \alpha\beta\xi \to \alpha\beta\\
                \xi\alpha\beta \to \alpha\beta\\
                \alpha\beta \to \bigcdot\\
                \to \bigcdot\\
            \end{cases}
            \] 
            \[
            \scr{B}: \begin{cases}
                \alpha\beta \to \bigcdot\$\\
                \to \bigcdot\$
            \end{cases}
            \]

            Пример 1. $\lambda,\quad \scr{B}(\lambda) = \$$

            $\scr{A}(\lambda) = \lambda \implies $ тело цикла не выполнилось
            
            \medskip

            Пример 2. $x = a \in V$

            $\scr{A}(a) = a \neq \lambda$ 

            $R: a \step \gamma a\step a\gamma \step\bigcdot \beta a$

            $L: \beta a \step \alpha\beta a \step\bigcdot \alpha\beta a$

            $\scr{A}(\alpha\beta a) = \lambda$ 

            $\scr{B}(\alpha\beta a) = \$ a$
            
            \medskip

            Пример 3. $x = ab$

             $\scr{A}(ab) = ab \neq \lambda$ 

             $R: ab \step \gamma ab \models^{2} ab\gamma \step\bigcdot \alpha\beta b$ 

             $L: \alpha\beta B \step \alpha a\beta b \step\bigcdot a\alpha\beta b$

              $\scr{A}(a\alpha\beta b) = \lambda$ 

              $\scr{B}(a\alpha\beta b) = a\$b$

              \medskip

              Пример 4. $x = abcde$

               $\scr{A}(x) = x \neq \lambda$ 
                
               1 Итерация:

               $R: abcde \step \gamma abcde \models^{5} abcde\gamma \step\bigcdot abcde\beta e$

               $L: abcd\beta e \step \alpha abcd\beta e \step\bigcdot a\alpha bcd \beta e$

               2 Итерация:

               $R: a\alpha bcd\beta e \step\bigcdot a\alpha bc\beta de$

               $L: a\alpha bc\beta de \step\bigcdot ab\alpha c\beta de$

               3 Итерация:

               $R: ab\alpha c\beta de \step\bigcdot ab\alpha\beta cde$

               $L: ab\alpha\beta cde \step\bigcdot ab\alpha\beta cde$

               $\scr{A}(ab\alpha\beta cde) = \lambda$ 

               $\scr{B}(ab\alpha\beta cde) = ab\$cde$
\end{itemize}

\section{Универсальный нормальный алгорифм.}

Пусть дан НА:
\[
\scr{A}: \begin{cases}
    u_1 \to [\bigcdot]v_1\\
    \vdots\\
    u_n \to [\bigcdot]v_n\\
\end{cases}
\] 

\[
    A^{\text{И}} \leftrightharpoons u_1\alpha[\beta]v_1\gamma u_2\alpha[\beta]v_2\gamma \ldots 
    \gamma u_{n}\alpha[\beta]v_{n}\text{, где } \alpha,\beta,\gamma \not\in V
\] 

Пусть
\[
\scr{A}_0: \begin{cases}
    \# a \to a\# \\
    \# b \to b\#\\
    \# \to \bigcdot aba
    \to \#\\
\end{cases}
\]

Отсюда
\[
    A_{0}^{\text{И}} = \# a \alpha a \# \gamma\# b\alpha b\#\gamma\# \alpha\beta aba\gamma\alpha\#
\]

Красивые скобки!!

\[
\{A_0\} = \underbrace{01110}_{\#} \underbrace{010}_{a} \underbrace{011110}_{\alpha}
\underbrace{010}_{a} \underbrace{01110}_{\#} \underbrace{01111110}_{\gamma}
\] 
это запись НА

\begin{matrix}
    a & b & \# & \alpha & \beta & \gamma\\
    1 & 2 & 3 & 4 & 5 & 6\\
\end{matrix}

\begin{theorem}
    (Об универсальном НА). Пусть $V$ - произвольный алфавит. Может быть построен НА  $U$ над алфавитом
     $V \cup V_0$ такой, что для любых НА  ${\cal A}$ в алфавите $V$ и слова $x \in V^{*}$ имеет место
     $U(\{{\cal A}\}\$x) \simeq {\cal A}(x)$, где $\$ \not\in V \cup V_0$
\end{theorem}


\section{Разрешимые и перечислимые языки.}

\begin{definition}
Язык $L \subseteq V^{*}$ называется алгоритмически разрешимым, если может быть построен
НА ${\cal A}_{L}$ над алфавитом $V$ такой, что
 \[
     (\forall x \in V^{*})(!{\cal A}_{L})\text{ и } {\cal A}_L(x) = \lambda
     \Longleftrightarrow x \in L
\] 
\end{definition}

\paragraph*{Пример.} Пусть $L = \{\omega\omega: \omega \in V^{*}\} $ 

рис1

\begin{definition}
НА $\widetilde{\scr{A}_L}$ называется полуразрешимым для языка $L \subseteq V^{*}$, если
\[
!\widetilde{\scr{A}_L}(x) \Longleftrightarrow x \in L\\
\] 
\end{definition}

\begin{theorem}
Если для языка $L$ невозможен полуразрешающий НА, то невозможен и разрешающий.
\end{theorem}

\end{document}
