\documentclass{report}

\usepackage{amssymb, amsmath, amsthm, amscd}
\usepackage[utf8]{inputenc}
\usepackage[russian]{babel}
\usepackage{bookmark}

\usepackage{tikz}
\usetikzlibrary{shapes,arrows,positioning,decorations.pathmorphing}

\usepackage{listings}

\usepackage{hyperref}
\hypersetup{
       colorlinks=true,
       linkcolor=blue,
}

\usepackage{geometry}
\geometry{papersize={15cm, 11in}, left=1.5cm, lmargin=1.5cm,right=2cm, top=2cm, bottom=3cm}

\usepackage{graphicx}

\usepackage{titlesec}
\titleformat{\chapter}[display]{\fontsize{17pt}{0pt}\bfseries}{}{-20pt}{}
\titleformat{\subsubsection}[display]{\fontsize{12pt}{0pt}\bfseries}{}{0pt}{}

\newcounter{mylabelcounter}

\makeatletter
\newcommand{\labelText}[2]{%
#1\refstepcounter{mylabelcounter}%
\immediate\write\@auxout{%
    \string\newlabel{#2}{{1}{\thepage}{{\unexpanded{#1}}}{mylabelcounter.\number\value{mylabelcounter}}{}}%
}%
}
\makeatother

\newcommand{\bslash}{\mbox{ } \backslash \mbox{ }}
\newcommand{\band}{\mbox{ } \& \mbox{ }}

%\newcommand*\pathto[1]{ \Rightarrow^{*}_{#1} }

\newcommand{\pathto}[1][1]{ \Rightarrow^{*}_{#1} }

\renewcommand{\phi}{\varphi}


\title{}
\author{Козырнов Александр Дмитриевич, ИУ7-32Б}
\date{\today}

\begin{document}

\paragraph*{Следствие 3.}
(Свойства конъюнкции). 

1) $A,B \step A\band B$

2)  $A\band B \step A,B$

3)  $A\band B \step B\band A$

\begin{myproof}
    ${}$

\paragraph*{1 пункт.} ${}$ \newline

1) $A$ - гипотеза

2) $B$ - гипотеза

3) $\lnot\lnot B$ - R4, (2)

4)  $\lnot(A \to \lnot B)$ - R8, (1) и (3)

\paragraph*{2 пункт} ${}$ \newline

1) $\lnot A$ - гипотеза

2)  $\lnot A \to (A \to \lnot B)$ - секвенция 5

3) $A \to \lnot B$ - MP, (1) и (2)

4) $\lnot\lnot(A \to \lnot B)$ - R4, (3)

\paragraph*{3 пункт.} ${}$ \newline

1) $B \to \lnot A$ - гипотеза

2) $\lnot\lnot A \to \lnot B$ - R7, (1)

3) $A \to \lnot\lnot A$ - секвенция 4

4) $A \to \lnot B$ - R1, (3) и (2)

\end{myproof}

\section{Непротиворечивость и полнота теории L}

\begin{theorem}
Любая теорема теории $L$ есть тавтология.
\end{theorem}
\begin{myproof}
Легко проверить, что каждая формула, получаемая из схемы аксиомы, будет тавтологией.

$\Phi$ - тавтология,  $\Phi \to \Psi$ - тавтология.

Пусть $\Psi$ - не есть тавтология.

 \[
     (\forall \widetilde{\alpha})\Phi(\widetile{\alpha}) = T,\quad 
     (\Phi \to \Psi)(\widetile{\alpha}) = \Phi(\widetilde{\alpha}) \to \Psi(\widetilde{\alpha}) = T
\] 

То есть
 \[
     \Phi(\widetilde{\alpha}) \to \Psi(\widetilde{\alpha}) = T \to F
\]
есть противоречие.
\end{myproof}

\paragraph*{Следствие.}
В теории $L$ нельзя доказать формулу и ее отрицание.


\begin{theorem}
Любая тавтология доказуема в теории $L$.
\end{theorem}

\begin{myproof}
Будем считать, что
\[
    \Phi = \Phi(x_1,\ldots,x_{n});\quad \widetilde{\alpha} = (\alpha_1,\ldots,\alpha_n);
    \quad \Phi^{\widetilde{\alpha}} \leftrightharpoons
    \begin{cases}
        \Phi,\text{ если } \Phi(\widetilde{\alpha}) = T\\
        \lnot\Phi\text{, если } \Phi(\widetilde{\alpha}) = F
    \end{cases}
\]

\paragraph*{Лемма (Кальмара)}.
$x^{\alpha_1}_1, \ldots,x_{n}^{\alpha_n} \step \Phi^{\widetilde{\alpha}}$ 
\begin{myproof}
    (Док-во леммы).
    Индукция по числу $l(\Phi)$ логических связок в формуле  $\Phi$.

    \medskip

    Базис: $l(\Phi) = 0$, значит формула  $\Phi$ есть переменная.
     $\Phi = x_{i}$ - переменная.

     Тогда очевидна такая секвенция
     $x_{i}^{\alpha_i} \step x_{i}^{\alpha_i}$, то есть
     $x_{i}\step x_{i}$ или $\lnot x_{i} \step \lnot x_{i}$ - очевидно
     В силу $\step(A \to A)$.
    
     \medskip

     Предположение: Пусть утверждение леммы справедливо при любом значении $l(\Phi) \le n-1,
     n \ge 1$

     \medskip

     Переход: Полагаем, что $l(\Phi) = n$.

     \paragraph*{1 случай.} ${}$ \newline

      $\Phi = \lnot \Psi$, где  $l(\Psi) = n - 1$

      1.1  $\Psi(\widetilde{\alpha}) = F$

      $\Phi(\widetilde{\alpha}) = n, \Phi^{\widetilde{\alpha}} = \Phi, \Psi^{\widetilde{\alpha}} =
      \lnot \Psi$ 

      По предположению индукции $x_1^{\alpha_1}, \ldots, x_{n}^{\alpha_n} \step
      \Psi^{\widetilde{\alpha}} = \lnot\Psi = \Phi = \Phi^{\widetilde{\alpha}}$ 

      \medskip

      1.2 $\Psi(\widetilde{\alpha}) = T$

      $\Phi(\widetilde{\alpha}) = F, \quad \Phi^{\widetilde{\alpha}} = \Phi, \quad
      \Psi^{\widetilde{\alpha}} = \Psi$ 
        
      $x_1^{\alpha_1},\ldots,x_{n}^{\alpha_n} \step \Psi^{\widetilde{\alpha}} \step \lnot\lnot\Psi =
      \lnot\Phi = \Phi^{\widetilde{\alpha}}$
        
      \paragraph*{2 случай.} ${}$ \newline
      
      $\Phi = q \to \psi$, где $l(Q) + l(\Psi) = n - 1, \quad l(Q), l(\Psi) < n$.

      \medskip

      2.1  $Q(\widetilde{\alpha}) = \Psi(\widetilde{\alpha}) = F$

      $Q^{\widetilde{\alpha}} = \lnot Q, \Psi^{\widetilde{\alpha}} = \lnot\Psi,
      \Phi(\widetilde{\alpha}) = F \to F = T$

      По предположению индукции:

      $x_1^{\alpha_1},\ldots,x_{n}^{\alpha_n} \step \lnot Q, \lnot \Psi; \quad \lnot Q \to 
      (Q \to \Psi)$ - секвенция 5; $Q \to \Psi$ - MP

      $x_1^{\alpha_1},\ldots, x_{n}^{\alpha_n} \step Q \to \Psi = \Phi = \Phi^{\widetilde{\alpha}}$

      \medskip

      2.2  $Q(\widetilde{\alpha}) = F \quad \Psi(\widetilde{\alpha}) = T$

      $Q^{\widetilde{\alpha}} = \lnot Q, \Psi^{\widetilde{\alpha}} = \Psi,
      \Phi(\widetilde{\alpha}) = F \to T = T$

      По предположению индукции:

      $x_1^{\alpha_1},\ldots,x_{n}^{\alpha_n} \step \lnot Q, \Psi; \quad \lnot Q \to 
      (Q \to \Psi)$ - секвенция 5; $Q \to \Psi$ - MP

      $x_1^{\alpha_1},\ldots, x_{n}^{\alpha_n} \step Q \to \Psi = \Phi = \Phi^{\widetilde{\alpha}}$

      \medskip

      2.3  $Q(\widetilde{\alpha}) = T \quad \Psi(\widetilde{\alpha}) = F$

      $Q^{\widetilde{\alpha}} = Q, \Psi^{\widetilde{\alpha}} = \lnot\Psi,
      \Phi(\widetilde{\alpha}) = T \to F = F$

      По предположению индукции:

      $x_1^{\alpha_1},\ldots,x_{n}^{\alpha_n} \step Q, \lnot \Psi; \lnot(Q \to \Psi)$ - по R8 

      $x_1^{\alpha_1},\ldots, x_{n}^{\alpha_n} \step \lnot(Q \to \Psi) = \lnot \Phi
      = \Phi^{\widetilde{\alpha}}$

      \medskip

      2.4  $Q(\widetilde{\alpha}) = T \quad \Psi(\widetilde{\alpha}) = T$

      $Q^{\widetilde{\alpha}} = Q, \Psi^{\widetilde{\alpha}} = \Psi,
      \Phi(\widetilde{\alpha}) = T \to T = T$

      По предположению индукции:

      $x_1^{\alpha_1},\ldots,x_{n}^{\alpha_n} \step Q, \Psi; \Psi \to (\Phi \to \Psi),
      Q \to \Psi$ - MP 

      $x_1^{\alpha_1},\ldots, x_{n}^{\alpha_n} \step (Q \to \Psi) = \Phi
      = \Phi^{\widetilde{\alpha}}$
     
\end{myproof}

Продолжаем доказательство теоремы.

Пусть $\Phi$ - тавтология, то есть  $(\forall \widetilde{\alpha})(\Phi(\widetilde{\alpha}) = T)$.

В силу леммы: $x_1^{\alpha_1}, \ldots, x_{n}^{\alpha_n} \step \Phi \left[ 
(\forall \widetilde{\alpha}) Phi^{\widetilde{\alpha}} = \Phi\right]$ 

$$\widetilde{\alpha}_1 = (\alpha_1,\ldots,\alpha_{n-1}, \lnot \alpha_n) \quad
x_1^{\alpha_1},\ldots,x_{n}^{\alpha_n-1}, x_{n-1}^{\alpha_n} \step \Phi$$

То есть 
\[
x_1^{\alpha_1},\ldots,x_{n-1}^{\alpha_{n-1}} \step \Phi,
\]
где стало на 1 меньше. Так отсчипываем, пока не получим:
\[
x_1^{\alpha_1} \step \Phi_1, \lnot x_1^{\alpha_1} \step \Phi
\]
\[
\step \Phi
\] 

\end{myproof}

\paragraph*{Следствие.} Формула является тавтологией тогда и только тогда, когда она доказуема
в теории $L.$

\section{Эквивалентные формулы}

\begin{definition}
Две формулы называют эквивалентными, если они выводимы друг из друга
\[
\Phi \equiv \Psi \leftrightharpoons \Phi \step \Psi \quad \Psi \step \Phi
\]
\[
\Phi \equiv \Psi \Longleftrightarrow \step (\Phi \to \Psi)\band (\Psi \to \Phi)
\]
\end{definition}

Также $\Phi \equiv \Psi \Longleftrightarrow \lnot \Phi \equiv \lnot \Psi$

\paragraph*{Утверждение.} Если $\Phi \equiv \Psi$, то  $(\forall \widetilde{\alpha})
(\Phi(\widetilde{\alpha}) = \Psi(\widetilde{\alpha}))$


\paragraph*{Примеры эквивалентности.} ${}$ \newline

1) $\lnot\lnot A \equiv A$

2)  $(A \to B) \equiv (\lnot B \to \lnot A)$

3) $\lnot ( A \lor B) \equiv \lnot A \band \lnot B \quad 
\lnot(A\band B) \equiv \lnot A \lor \lnot B$

\end{document}
