\documentclass{report}

\usepackage{amssymb, amsmath, amsthm, amscd}
\usepackage[utf8]{inputenc}
\usepackage[russian]{babel}
\usepackage{bookmark}
\usepackage{wasysym}
\usepackage{import}
\usepackage{caption}
\usepackage{mathrsfs}

\usepackage[makeroom]{cancel}

\usepackage{tikz}
\usetikzlibrary{shapes,shapes.geometric,arrows,positioning,decorations.pathmorphing}

\usepackage{listings}

\usepackage{hyperref}
\hypersetup{
       colorlinks=true,
       linkcolor=blue,
}

\usepackage{geometry}
%\geometry{papersize={15cm, 11in}, left=1.5cm, lmargin=1.5cm,right=2cm, top=2cm, bottom=3cm}
\geometry{a4paper, left=1.5cm, lmargin=1.5cm,right=2cm, top=2cm, bottom=3cm}

\tolerance=1
\emergencystretch=\maxdimen
\hyphenpenalty=10000
\hbadness=10000

\usepackage{graphicx}

\usepackage{titlesec}
\titleformat{\chapter}[display]{\fontsize{17pt}{0pt}\bfseries}{}{-20pt}{}
\titleformat{\subsubsection}[display]{\fontsize{12pt}{0pt}\bfseries}{}{0pt}{}

\newcounter{mylabelcounter}

\makeatletter
\newcommand{\labelText}[2]{%
#1\refstepcounter{mylabelcounter}%
\immediate\write\@auxout{%
    \string\newlabel{#2}{{1}{\thepage}{{\unexpanded{#1}}}{mylabelcounter.\number\value{mylabelcounter}}{}}%
}%
}
\makeatother

\newcommand{\circlesign}[1]{
    \mathbin{
        \mathchoice
        {\buildcircledesign{\displaystyle}{#1}}
        {\buildcircledesign{\textstyle}{#1}}
        {\buildcircledesign{\scriptstyle}{#1}}
        {\buildcircledesign{\scriptscriptstyle}{#1}}
    }
}

\newcommand\buildcircledesign[2]{%
    \begin{tikzpicture}[baseline=(X.base), inner sep=0, outer sep=0]
        \node[draw,circle] (X) {\ensuremath{#1 #2}};
    \end{tikzpicture}%
}

\newcommand{\ozv}{\circlesign{*}}

\theoremstyle{plain}
\newtheorem{theorem}{Теорема}[chapter]
\theoremstyle{definition}
\newtheorem{definition}{Определение}
\newtheorem*{pruf}{Доказательство}

\newenvironment{myproof}[1][\textbf{\textup{Доказательство}}]
{\begin{proof}[#1]}
	%\renewcommand*{\qedsymbol}{\(\blacksquare\)}}
{\end{proof}}

\DeclareMathOperator{\band}{\&}
\DeclareMathOperator{\trim}{\triangle}
\DeclareMathOperator{\step}{\vdash}

\makeatletter
\newcommand*\bigcdot{\mathpalette\bigcdot@{.5}}
\newcommand*\bigcdot@[2]{\mathbin{\vcenter{\hbox{\scalebox{#2}{$\m@th#1\bullet$}}}}}
\makeatother


%\newcommand*\pathto[1]{ \Rightarrow^{*}_{#1} }

\newcommand{\pathto}[1][1]{ \Rightarrow^{*}_{#1} }

\newcommand{\dx}[2]{\frac{d#1}{d#2}}
\newcommand{\pdx}[2]{\frac{\partial#1}{\partial#2}}
\newcommand{\zap}[1]{\reflectbox{$3$}#1 3}

\newcommand{\incfig}[2]{%
    \def\svgwidth{#2\columnwidth}
    \import{./images/}{#1.pdf_tex}
}

\tikzstyle{startstop} = [rectangle, rounded corners, 
minimum width=3cm, 
minimum height=1cm,
text centered, 
draw=black, 
fill=red!30]

\tikzstyle{io} = [trapezium, 
trapezium stretches=true, % A later addition
trapezium left angle=70, 
trapezium right angle=110, 
minimum width=3cm, 
minimum height=1cm, text centered, 
draw=black, fill=blue!30]

\tikzstyle{process} = [rectangle, 
minimum width=3cm, 
minimum height=1cm, 
text centered, 
text width=3cm, 
draw=black, 
fill=orange!30]

\tikzstyle{decision} = [diamond, 
minimum width=3cm, 
minimum height=1cm, 
text centered, 
draw=black, 
fill=green!30]
\tikzstyle{arrow} = [thick,->,>=stealth]


%\newcommand{\void}{\varnothing}
\let\void\varnothing

\renewcommand{\phi}{\varphi}
\renewcommand{\epsilon}{\varepsilon}
\newcommand{\scr}[1]{\mathscr{#1}}



\title{}
\author{Козырнов Александр Дмитриевич, ИУ7-32Б}
\date{\today}

\begin{document}
\section{Классификации грамматик}

\begin{enumerate}
    \item[1)] Грамматики типа 0
    \item[2)] Неукорачивающие грамматики (НК-)
    \item[3)] Контекстно зависимые грамматики (КЗ-)
    \item[4)] ОКЗ-грамматики (ограниченно КЗ)
    \item[5)] Контекстно свободные (КС-)
    \item[6)] Линейные грамматики
    \item[7)] Праволинейные грамматики
    \item[8)] Леволинейные грамматики
    \item[9)] Регулярные (автоматные) грамматики
\end{enumerate}

\begin{definition}
Грамматики называются эквивалентными, если они порождают один и тот же язык
\[
G_1 \simeq G_2 \leftrightharpoons L(G_1) = L(G_2)
\] 
\end{definition}

\begin{definition}
Грамматики называют почти эквивалентными, если порождаемые ими языки совпадают
с точностью до пустого слова, то есть
\[
G_1 \approx G_2 \leftrightharpoons L(G_1)\nabla L(G_2) \subseteq \{\lambda\} 
\] 
\end{definition}

\begin{theorem} ${}$\newline
1) Для каждой грамматики типа 0 может быть построена эквивалентная ей ОКЗ-грамматика

2) Для каждой неукорачивающей грамматики может быть построена эквивалентная ей КЗ-грамматика

3) Для каждой КС-грамматики может быть построена почти эквивалентная ей КС-грамматика, не
содержащая правил с пустой правой частью (т.н. лямбда-правил)

4) Для каждой леволинейной грамматики может быть построена эквивалентная ей
праволинейная грамматика и наоборот.

5) Для каждой праволинейной грамматики может быть построена жквивалентная ей регулярная
грамматика
\end{theorem}

\begin{theorem}
Язык перечислим тогда и только тогда, когда он порождается грамматикой типа 0.

Всякий КС-язык разрешим, но обратное неверно.
\end{theorem}

\section{МП-автоматы (Pushdown machine)}
рис1

$qaZ \to r\gamma$, где $q,r \in Q$, $Z \in \Gamma, \gamma \in \Gamma^{*}$, $a \in V \cup \{\lambda\} $


рис2

\paragraph*{Пример} ${}$ \newline

$$
\begin{align*}
    &q_0aZ \to q_0\text{  }aZ\\
    &q_0aa \to q_0\text{  }aa\\
    &q_0ba \to q_1\lambda\\
    &q_1ba \to q_1\lambda\\
    &q_1\lambda Z \to q_2\lambda
\end{align*}
$$

Машинный автомат может быть описан тоже в виде конфигураций.
Начальное:
\[
    (q, ay, Z\alpha) \quad \alpha \in \Gamma^{*}\text{, то есть может быть пустой}
\]
Z - все, что есть в магазине.

\medskip

\[
\begin{align*}
    &(q_0, aabb, Z) \step (q_0, abb, aZ) \step (q_0,bb,aaZ) \step (q_1,b,aZ) \step (q_1,\lambda,Z)
    \step (q_1, \lambda, \lambda)
\end{align*}
\] 

\begin{definition}
    $\scr{M} = (Q, V, \Gamma, q_0, F, Z_0 \text{(нач. маг. симв.)}, \delta\text{(сист. перех.)})$ -
    магазинный автомат
\end{definition}

\begin{definition}
Конфигурация МП-авт: $(Q, ay, Z\alpha)$,
где $q \in Q$, $a \in V \cup \{\lambda\}$, $y \in V^{*}$, $z \in \Gamma, \alpha \in \Gamma^{*}$ 

\[
    (q, ay, Z\alpha) \underset{\scr{M}}{\step} (r, y, \gamma\alpha) \leftrightharpoons
    qaZ \to r\gamma
\] 
\end{definition}
Далее отношение непосредственной выводимости на мн-стве конфигурации рефлексивно-транзитивно
замыкается подобно тому, как это было сделано на конфигурации машины Тьюринга.

\begin{definition}
Язык, допускаемый магазинным автоматом, - это
\[
L(\scr{M}) \leftrightharpoons \{x: (q_0,x,Z_0)\} \step^{*} (q_f, \lambda, \alpha),
\]
где $q_f \in F$.

\medskip

Мы можем немного переопределить наш язык так:
\[
    L(\scr{M}) = \{x: (q_0,x,Z_0) \step^{*} (q_f, \lambda, \lambda); x\in V^{*} \}
\] 
\end{definition}

\end{document}
