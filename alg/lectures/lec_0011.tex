\documentclass{report}

\usepackage{amssymb, amsmath, amsthm, amscd}
\usepackage[utf8]{inputenc}
\usepackage[russian]{babel}
\usepackage{bookmark}

\usepackage{tikz}
\usetikzlibrary{shapes,arrows,positioning,decorations.pathmorphing}

\usepackage{listings}

\usepackage{hyperref}
\hypersetup{
       colorlinks=true,
       linkcolor=blue,
}

\usepackage{geometry}
\geometry{papersize={15cm, 11in}, left=1.5cm, lmargin=1.5cm,right=2cm, top=2cm, bottom=3cm}

\usepackage{graphicx}

\usepackage{titlesec}
\titleformat{\chapter}[display]{\fontsize{17pt}{0pt}\bfseries}{}{-20pt}{}
\titleformat{\subsubsection}[display]{\fontsize{12pt}{0pt}\bfseries}{}{0pt}{}

\newcounter{mylabelcounter}

\makeatletter
\newcommand{\labelText}[2]{%
#1\refstepcounter{mylabelcounter}%
\immediate\write\@auxout{%
    \string\newlabel{#2}{{1}{\thepage}{{\unexpanded{#1}}}{mylabelcounter.\number\value{mylabelcounter}}{}}%
}%
}
\makeatother

\newcommand{\bslash}{\mbox{ } \backslash \mbox{ }}
\newcommand{\band}{\mbox{ } \& \mbox{ }}

%\newcommand*\pathto[1]{ \Rightarrow^{*}_{#1} }

\newcommand{\pathto}[1][1]{ \Rightarrow^{*}_{#1} }

\renewcommand{\phi}{\varphi}


\title{}
\author{Козырнов Александр Дмитриевич, ИУ7-32Б}
\date{\today}

\begin{document}

\section{Дизъюнктивная и конъюнктивная нормальные формы (ДНФ и КНФ)}

\begin{definition}
Литерал - это формула, в которой есть либо переменная, либо отрицание переменной.
\[
x^{\sigma} \leftrightharpoons \begin{cases}
    x_{i}\text{, если } \sigma = 1\\
    \overline{x_{i}}\text{, если } \sigma = 0
\end{cases}
\]

Обозначение $\widetilde{x_{i}}$ - это возможное отрицание.
\end{definition}

\begin{definition}
Элементарная конъюнкция - это конъюнкция каких-то литералов.
\[
\widetilde{x_{i_1}}\widetilde{x_{i_2}}\ldots \widetilde{x_{ik}}
\] 
\end{definition}

\begin{definition}
ДНФ - это $k_1 \lor k_2 \lor \ldots \lor k_{m}$ от $x_1,x_2,\ldots,x_3$, где 
$k_{i}$ - элементарная конъюнкция.
\end{definition}

\begin{definition}
В СДНФ в каждую элементарную конъюнкцию входит каждый из $x_1,x_2,\ldots,x_{n}$ либо сам,
либо как отрицание.
\end{definition}

\medskip

ДНФ: $\{x_1,x_2,x_3\}:\quad \overline{x_1}x_2 \lor x_2 \lor x_1\overline{x_2}\overline{x_3} $ 

СДНФ: $\{x_1,x_2,x_3\}:\quad x_1x_2x_3 \lor x_1\overline{x_2}x_3 \lor \overline{x_1}x_2x_3$

\begin{definition}
Элементарная дизъюнкция - это дизъюнкция каких-то литералов.
\end{definition}

\begin{definition}
    КНФ от $x_1,x_2,\ldots, x_{n}$: $D_1 * D_2 * \ldots * D_m, m\ge 1$
\end{definition}

\begin{definition}
В СКНФ в каждую элементарную дизъюнкцию входит каждый из $x_1,x_2,\ldots,x_{n}$ либо сам,
либо как отрицание.
\end{definition}

\begin{theorem}
Любая функция, отличная от константы 0, может быть представлена в виде ДНФ.
Любая функция, отличная от константы 1, может быть представлена в виде КНФ.
\end{theorem}
\begin{myproof}
    1) Так как $ f \not \equiv 0$, то  $\exists \widetilde{\alpha} \in \{0,1\}^{n}:
    f(\widetilde{\alpha})$ = 1 - называется это конституента 1 функции $f$.

    Тогда
    $$
    C^{1}_{f} \leftrightharpoons \{\widetilde{\alpha}:
    f(\widetilde{\alpha}) = 1\} \neq \void, \widetilde{\alpha} = 
    (\alpha_1, \ldots, \alpha_n)
    .$$
    $K_{\widetilde{\alpha}} = x_1^{\alpha_1}
    x_2^{\alpha_2}\ldots x_{n}^{\alpha_n}$ 

    Заметим, что
    \[
        K_{\widetilde{\alpha}}(\widetilde{\beta}) = 1 \Longleftrightarrow \widetilde{\beta} =
        \widetilde{\alpha}
    \] 
    Отсюда получаем:
    $$
    \boxed{
    f(x_1,\ldots,x_n) = \underset{{\cal Z} \in C^{1}_f}{\bigvee}K_{\widetilde{\alpha}}
    }
    $$

    Заметим, что если
    \[
        f(x_1,\ldots,x_m) = 1 \implies (\exists \widetilde{\alpha} \in C^{1}_f)(f(\widetilde{\alpha})
        = 1) \implies k_{\widetilde{\alpha}} = 1 \implies \underset{{\cal Z} \in C^{1}_f}{\biglor}
        k_{\widetilde{\alpha}} = 1,
    \]
    то есть $f(\widetilde{\alpha}) = 1$. Аналогично для КНФ.
\end{myproof}
\paragraph*{Следствие.} Любая булевая функция может быть представлена некоторой формулой над
стандартным базисом. То есть стандартным базисом является полным множеством булевых функций.

\section{Полином Жегалкина}

${\cal F}_1 = \{\oplus, *, 1\} $ 

Отсюда $\overline{x} = x\oplus 1$ и $x_1 \lor x_2 = x_1x_2 \oplus x_1\oplus x_2$.



\begin{definition}
Полиномом Жегалкина является
\[
P(x_1,x_2,\ldots,x_{n}) = \sum\limits (mod 2) a_{i_1i_2\ldots i_k}x_{i_1}x_{i_2}\ldots x_{ik},
\quad \{i_1,i_2,\ldots,i_k\} \subseteq \{1,2,\ldots,n\}  
.\] Здесь $2^{n}$ слагаемых. $a_{i_1i_2\ldots i_k} \in \{0,1\}$ 
\end{definition}

Общий вид полинома Жегалкина от двух переменных:
\[
P(x_1,x_2) = a_{12}x_1x_2 \oplus a_1x_1 \oplus a_2x_2 \oplus a_0
\]

Общий вид от трех:
\[
P(x_1,x_2,x_3) = a_{123}x_1x_2x_3 \oplus a_{12}x_1x_2 \oplus a_{13}x_1x_3 \oplus a_{23}x_2x_3 \oplus
a_1x_1 \oplus x_2x_2 \oplus a_3x_3 \oplus a_0
\] 

\begin{theorem}
Каждая булева функция однозначно представима в виде полинома Жегалкина.
\end{theorem}

\paragraph*{Метод неопределенных коэффициентов.} ${}$ \newline

$f = (00010111)$

$f(0,0,0) = a_0 = 0$

$f(1,0,0) = a_1\oplus a_0 = 0 \implies a_1 = 0$

$f(0,1,0) = a_2 \oplus a_0 = 0 \implies a_2 = 0$

$f(0,0,1) = a_3 \oplus a_0 = 0 \implies a_3 = 0$

$f(1,1,0) = a_{12} \oplus a_2 \oplus a_1 \oplus a_0 = 1 \implies a_{12} = 1$

$f(1,0,1) = a_{13} \oplus a_1 \oplus a_3 \oplus a_0 \implies a_{13} = 1$

$f(0,1,1) = a_{23} \oplus a_2 \oplus a_3 \oplus 3 \implies a_{23} = 1$

$f(1,1,1) = a_{123} \oplus a_{12} \oplus a_{13} \oplus a_{23} \oplus a_1 \oplus a_2 \oplus a_3
\oplus a_0 = 1 \implies a_{123} \oplus 1 = 1 \implies a_{123} = 0$

\begin{definition}
Булева функция называется линейной, если она может быть представлена полиномом Жегалкина
первой степени.
\[
f \in L \leftrightharpoons f(x_1,\ldots,x_{n}) = \sum\limits_{i=1}^{n}(mod2)a_{i}x_{i}\oplus a_0
\] 
\end{definition}

\end{document}
