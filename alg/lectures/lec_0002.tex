\documentclass{report}

\usepackage{amssymb, amsmath, amsthm, amscd}
\usepackage[utf8]{inputenc}
\usepackage[russian]{babel}
\usepackage{bookmark}
\usepackage{wasysym}
\usepackage{import}
\usepackage{caption}
\usepackage{mathrsfs}

\usepackage[makeroom]{cancel}

\usepackage{tikz}
\usetikzlibrary{shapes,shapes.geometric,arrows,positioning,decorations.pathmorphing}

\usepackage{listings}

\usepackage{hyperref}
\hypersetup{
       colorlinks=true,
       linkcolor=blue,
}

\usepackage{geometry}
%\geometry{papersize={15cm, 11in}, left=1.5cm, lmargin=1.5cm,right=2cm, top=2cm, bottom=3cm}
\geometry{a4paper, left=1.5cm, lmargin=1.5cm,right=2cm, top=2cm, bottom=3cm}

\tolerance=1
\emergencystretch=\maxdimen
\hyphenpenalty=10000
\hbadness=10000

\usepackage{graphicx}

\usepackage{titlesec}
\titleformat{\chapter}[display]{\fontsize{17pt}{0pt}\bfseries}{}{-20pt}{}
\titleformat{\subsubsection}[display]{\fontsize{12pt}{0pt}\bfseries}{}{0pt}{}

\newcounter{mylabelcounter}

\makeatletter
\newcommand{\labelText}[2]{%
#1\refstepcounter{mylabelcounter}%
\immediate\write\@auxout{%
    \string\newlabel{#2}{{1}{\thepage}{{\unexpanded{#1}}}{mylabelcounter.\number\value{mylabelcounter}}{}}%
}%
}
\makeatother

\newcommand{\circlesign}[1]{
    \mathbin{
        \mathchoice
        {\buildcircledesign{\displaystyle}{#1}}
        {\buildcircledesign{\textstyle}{#1}}
        {\buildcircledesign{\scriptstyle}{#1}}
        {\buildcircledesign{\scriptscriptstyle}{#1}}
    }
}

\newcommand\buildcircledesign[2]{%
    \begin{tikzpicture}[baseline=(X.base), inner sep=0, outer sep=0]
        \node[draw,circle] (X) {\ensuremath{#1 #2}};
    \end{tikzpicture}%
}

\newcommand{\ozv}{\circlesign{*}}

\theoremstyle{plain}
\newtheorem{theorem}{Теорема}[chapter]
\theoremstyle{definition}
\newtheorem{definition}{Определение}
\newtheorem*{pruf}{Доказательство}

\newenvironment{myproof}[1][\textbf{\textup{Доказательство}}]
{\begin{proof}[#1]}
	%\renewcommand*{\qedsymbol}{\(\blacksquare\)}}
{\end{proof}}

\DeclareMathOperator{\band}{\&}
\DeclareMathOperator{\trim}{\triangle}
\DeclareMathOperator{\step}{\vdash}

\makeatletter
\newcommand*\bigcdot{\mathpalette\bigcdot@{.5}}
\newcommand*\bigcdot@[2]{\mathbin{\vcenter{\hbox{\scalebox{#2}{$\m@th#1\bullet$}}}}}
\makeatother


%\newcommand*\pathto[1]{ \Rightarrow^{*}_{#1} }

\newcommand{\pathto}[1][1]{ \Rightarrow^{*}_{#1} }

\newcommand{\dx}[2]{\frac{d#1}{d#2}}
\newcommand{\pdx}[2]{\frac{\partial#1}{\partial#2}}
\newcommand{\zap}[1]{\reflectbox{$3$}#1 3}

\newcommand{\incfig}[2]{%
    \def\svgwidth{#2\columnwidth}
    \import{./images/}{#1.pdf_tex}
}

\tikzstyle{startstop} = [rectangle, rounded corners, 
minimum width=3cm, 
minimum height=1cm,
text centered, 
draw=black, 
fill=red!30]

\tikzstyle{io} = [trapezium, 
trapezium stretches=true, % A later addition
trapezium left angle=70, 
trapezium right angle=110, 
minimum width=3cm, 
minimum height=1cm, text centered, 
draw=black, fill=blue!30]

\tikzstyle{process} = [rectangle, 
minimum width=3cm, 
minimum height=1cm, 
text centered, 
text width=3cm, 
draw=black, 
fill=orange!30]

\tikzstyle{decision} = [diamond, 
minimum width=3cm, 
minimum height=1cm, 
text centered, 
draw=black, 
fill=green!30]
\tikzstyle{arrow} = [thick,->,>=stealth]


%\newcommand{\void}{\varnothing}
\let\void\varnothing

\renewcommand{\phi}{\varphi}
\renewcommand{\epsilon}{\varepsilon}
\newcommand{\scr}[1]{\mathscr{#1}}



\title{}
\author{Козырнов Александр Дмитриевич, ИУ7-32Б}
\date{\today}

\begin{document}

Дано:

Начальная конфигурация $C_0=(q_0,\lambda, \ozv x\square)$, где $x \in V^{*}$

Конечная конфигурация $C_{f} = (q_{f}, \lambda, \ozv y\square)$, где $y \in V^{*}$


\begin{definition}
Машина Тьюринга применима к слову х, то есть
\begin{align*}
    !{\cal T}(x) \leftrightharpoons &\\
                                    \leftrightharpoons& C_0 = (q_0,\lambda,\ozv x \square)
\step^{*} C_{f} = (q_{f}, \lambda, \ozv y \square);
\end{align*}
при этом $y \leftrightharpoons {\cal T}(x)

\medskip

При этом если не применимо к машине тьюринга данное слово, то $$\lnot! {\cal T}(x)$$
\end{definition}

\begin{definition}
Конфигурация машины Тьюринга называется тупиковой, если она не является заключительной и при этом из
нее не выводится ни одна конфигурация.
\end{definition}

\paragraph*{Пример.}
\[
f(x) = \begin{cases}
    \#\text{, если } x = \lambda \\
    \lambda\text{, если } cab \sqsubseteq x\\
    x\text{, если } x\neq \lambda\text{ и } cab \not\sqsubseteq x
\end{cases}
\]
$\lambda$ - Пустое слово.

Тогда программа записывается так:
\begin{align*}
    &q_0\ozv \to q_0\ozv, R\\
    &q_0\square \to q_{f}\#, L \\
    &q_0a \to q_0'a, R \\
    &q_0b \to q_0'b, R \\
    &q_0c \to q_1c, R \\
    &q_0'a \to q_0'a, R\\
    &q_0'b \to q_0'b, R\\
    &q_0'c \to q_1c, R\\
    &q_1a\to q_2a,R\\
    &q_1b \to q_0'b, R\\
    &q_1c\to q_1c,R\\
    &q_2a\to a_0'a, R \text{ //} caa\\
    &q_2b\to q_3b, R \text{ //} cab\\
    &q_2c\to q_1c, R\text{ //} cac\\
    &q_3\alpha \to q_3\alpha, R\text{ //}\alpha \in \{a,b,c\}\\ 
    &q_3\square\to q_4\square,L\\
    &q_4\alpha\to q_4\square,L\\
    &q_4\ozv\to q_{f}\ozv,S \\
    &r\square\to q_5\square, L\text{ //} r \in \{q_0',q_1,q_2\} \\
    &q_5\alpha\to q_5\alpha,L\\
    &q_5\ozv\to q_{f}\ozv, S\\
\end{align*}

Для ошибочного решения ($q_{0}'$ не вводится): \[
\begin{align*}
    &(a_1,\lambda, \ozv ab\square) \step (q_0,\ozv,ab\square)
    &\step (q_0,\ozv a,b\square) \step (q_0,\ozv ab, \square)
    &\step \underline{(q_{f}, \ozv a, b\#\square)}
\end{align*}
\] 

\begin{definition}
Машина Тьюринга называется детерминированной, если из каждой ее конфигурации
\underline{непосредственно} выводится не более одной конфигурации.
\end{definition}

\begin{theorem}
Машина Тьюринга называется детерминированной тогда и только тогда, когда в ее программе
(системе команд) нет двух (более) различных комманд с одинаковыми левыми частями.
\end{theorem}

\medskip

\paragraph*{Соглашение.}
Во всех дальнейших суждениях машина Тьюринга будет считаться детерминированной. ДМТ - 
детерминированная машина Тьюринга.

\medskip

Допустим машина Тьюринга с алфавитом V, то мы говорим, что это машина Тьюринга в
алфавите V. Но если $V \supset V'$, то мы говорим, что Машина Тьюринга над алфавитом
V.

\begin{definition}
    Вербальная функция $f: V^{*} \underset{\bullet}{\to} V^{*}$ называется вычисломой по Тьюрингу,
если может быть построена МТ  $ {\cal T}_{f}$ над алфавитом $V$ такая, что  \[
        (\forall x \in V^{*})
        (! {\cal T}(x) \Longleftrightarrow x \in D(f) \band {\cal T}_{f}(x) = f(x))
    \] 
\end{definition}

\paragraph*{Тезис Тьюринга.}
Он гласит, что любая вербальная функция, вычислимая в интуитивном смысле слова, вычислима
по Тьюрингу.

\medskip

Общие разделы:
\begin{itemize}
    \item[1.] Основная модель.
    \item[2.] Понятие вычислимой функциию. Основная гипотеза.
    \item[3.] Эквивалентный алгоритм.
    \item[4.] Теорема сочетания.
    \item[5.] Универсальный алгоритм.
    \item[6.] Разрешимые перечислимые множества (языки).
    \item[7.] Анализ алгоритмически неразрешимых задач.
\end{itemize}

\section{Нормальные алгорифмы Маркова}

Предположим, что есть
\[
    V; x,y \in V^{*}; x\sqsubseteq y \leftrightharpoons (\exists y_1,y_2)(y=y_1xy_2)
\] 
причем тройка слов $(y1,x,y2)$ - \underline{вхождение} слова х в слово у.

\medskip

Некоторые свойства:
\begin{itemize}
    \item $(\forall x)(\lambda \sqsubseteq x)$
    \item $(\forall x)(x \sqsubseteq x)$
    \item $(\forall x)(\forall y)(\forall z)
        (x \sqsubseteq y, y\sqsubseteq z \implies x \sqsubseteq z)$
\end{itemize}


\medskip

Записывается иногда так: $y_1*x*y_2$  $(x \not\in V)$

Пример: $y = \underbrace{\text{вход}}_x$ит;  $*$вход$*$ит - корень

Еще один: $\underbrace{\text{абра}}_x$кад$\underbrace{\text{абра}}_x$

\medskip

Среди всех вхождений х в у выделяется первое, или главное, вхождение, а именно
имеющую наименьшую длину левого крыла (самое левое вхождение).

\medskip

\begin{definition}
Подстановка: \[
    u,v \in V^{*}\text{    } \underbrace{u}_{\text{л.ч.}} \to \underbrace{v}_{\text{п.ч.}}; \to \not\in V
\] 
\end{definition}

\begin{definition}
Омега применима, или подходит, если ее левая часть входит в слово х. \[
\omega : u \to v
\] 
Тогда вхождение:
\[
    x=x_1 u x_2; \text{    } x_1*u*x_2\text{ - 1-е вхождение u в х }
\]
Отсюда
\[
y \leftrightharpoons \omega x \leftrightharpoons x_1 v x_2
\] 
Это можно представить так:

\begin{matrix}
    x &= \fbox{x_{1}}\fbox{u_{}}\fbox{x_{2}}\\
      &\quad\downarrow\\
    y=\omega x&= \fbox{x_{1}}\fbox{v_{}}\fbox{x_{2}}
\end{matrix}

\end{definition}

\paragraph*{Пример.}
Пусть дана замена: \[
\omega: B \to y
\] 
Тогда слово Входит превратится в слово уходит. $\omega x$ = уходит

\begin{definition}
Нормальный алгорифм $ {\cal A} = (V,S, {\cal P})$
\end{definition}

\paragraph*{Пример.}
\[
{\cal A}: \begin{cases}
    \#a\to a# (1)\\
    \#b \to b\#\\
    \# \to \cdot aba\\
    \text{  } \to \#
\end{cases}
\] 


Рассматриваем систему сверху вниз и ищем первую подходящую формулу. Пусть \[
x = bbab
\] 
Отсюда получаем:
\[
    x = bbab \step \#bbab \step b\#bab \step bb\#ab \step bba\#b \step bbab\#
    \step\bigcdot bbab\underline{aba}
\] 

\medskip

Общий вид:\[
{\cal A}: \begin{cases}
    u_1 \to [\bigcdot]v_1\\
    u_2 \to [\bigcdot]v_2\\
    \vdots\\
    u_{n} \to [\bigcdot]v_{n}
\end{cases}
\]

Можно записать это в виде блок-схемы неформально:


\begin{tikzpicture}[node distance=2cm]
\node (start) [process] {Входное слово х};
\node (search) [process, below=of start] {Поиск в схеме 1-й подходящей для х формулы};
\node (decision1) [decision, below of=search, yscale=.5, xscale=0.6, yshift=-1.5cm]
    {Такая формула $\omega$ нашлась?};
\node (getword) [process, below=of decision1] {Получить слово $\omega x$};
\node (decision2) [decision, below=of getword, xscale=.6,yscale=.5]
    {Формула $\omega$ заключительная?};
\node (end2) [process,left of=decision2,scale=.5, xshift=-2cm] {END};
\node (xxx) [process, right of=decision2, scale=.5, xshift=5cm] {$x :=\omega x$};
\node (end1) [process, right of=decision1, scale=.5, xshift=2cm] {END};
\node (getspace) [left of=decision1, xshift=-1.5cm]


\draw [arrow] (start) -- (search);
\draw [arrow] (search) -- (decision1);
\draw [arrow] (decision1) -- node[anchor=east, above=3mm] {Нет} (end1);
\draw [arrow] (decision1) -- node[anchor=west, above=3mm] {Да} (getspace);
\draw [arrow] (getspace) |- (getword);
\draw [arrow] (getword) -- (decision2);
\draw [arrow] (decision2) -- node[above=3mm] {Да} (end2);
\draw [arrow] (decision2) -- node[above=3mm] {Нет} (xxx);
\draw [arrow] (xxx) |- (start);

\end{tikzpicture}

\medskip

Теперь формально опишем его.
Распишем 5 разных ситуаций.

\begin{itemize}
    \item[1)] $ {\cal A}: x\step y \leftrightharpoons$
        непосредственно просто переводит
        слово х в слово у $\leftrightharpoons y = \omega x$, где $\omega$ - 1-я в схеме  $ {\cal A}$ 
        формула, которая оказывается простой
    \item[2)] $ {\cal A} \step\bigcdot y \leftrightharpoons$
        Алгорифм А непосредственно заключительно переводит
        слово х в слово у $\leftrightharpoons y = \omega x$, где $\omega$ - 1-я в схеме
         $ {\cal A}$, которая оказывается заключительной
     \item[3)] $ {\cal A} x\models y \leftrightharpoons$
         Алгорифм А переводит слово х в слово у, когда
         существует последовательность $x=x_0,x_1,\ldots,x_{n}=y$, где
         $(\forall i=\overline{0,n-1})( {\cal A}:x_{i}\step x_{i+1})$ 
     \item[4)] $ {\cal A}:x \models\bigcdot y \leftrightharpoons$
         Алгорифм А заключительно переводит слово х в слово у
         $\leftrightharpoons {\cal A}: x\step\bigcdot y \lor
         (\exists z)( {\cal A}:x \models z\step\bigcdot y )$ 
     \item[5)] $\sim {\cal A}(x) \leftrightharpoons$ в схеме А нет ни одной подходящей формулы для х.
\end{itemize}
\end{document}
