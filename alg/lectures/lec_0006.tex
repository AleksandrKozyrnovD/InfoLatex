\documentclass{report}

\usepackage{amssymb, amsmath, amsthm, amscd}
\usepackage[utf8]{inputenc}
\usepackage[russian]{babel}
\usepackage{bookmark}
\usepackage{wasysym}
\usepackage{import}
\usepackage{caption}
\usepackage{mathrsfs}

\usepackage[makeroom]{cancel}

\usepackage{tikz}
\usetikzlibrary{shapes,shapes.geometric,arrows,positioning,decorations.pathmorphing}

\usepackage{listings}

\usepackage{hyperref}
\hypersetup{
       colorlinks=true,
       linkcolor=blue,
}

\usepackage{geometry}
%\geometry{papersize={15cm, 11in}, left=1.5cm, lmargin=1.5cm,right=2cm, top=2cm, bottom=3cm}
\geometry{a4paper, left=1.5cm, lmargin=1.5cm,right=2cm, top=2cm, bottom=3cm}

\tolerance=1
\emergencystretch=\maxdimen
\hyphenpenalty=10000
\hbadness=10000

\usepackage{graphicx}

\usepackage{titlesec}
\titleformat{\chapter}[display]{\fontsize{17pt}{0pt}\bfseries}{}{-20pt}{}
\titleformat{\subsubsection}[display]{\fontsize{12pt}{0pt}\bfseries}{}{0pt}{}

\newcounter{mylabelcounter}

\makeatletter
\newcommand{\labelText}[2]{%
#1\refstepcounter{mylabelcounter}%
\immediate\write\@auxout{%
    \string\newlabel{#2}{{1}{\thepage}{{\unexpanded{#1}}}{mylabelcounter.\number\value{mylabelcounter}}{}}%
}%
}
\makeatother

\newcommand{\circlesign}[1]{
    \mathbin{
        \mathchoice
        {\buildcircledesign{\displaystyle}{#1}}
        {\buildcircledesign{\textstyle}{#1}}
        {\buildcircledesign{\scriptstyle}{#1}}
        {\buildcircledesign{\scriptscriptstyle}{#1}}
    }
}

\newcommand\buildcircledesign[2]{%
    \begin{tikzpicture}[baseline=(X.base), inner sep=0, outer sep=0]
        \node[draw,circle] (X) {\ensuremath{#1 #2}};
    \end{tikzpicture}%
}

\newcommand{\ozv}{\circlesign{*}}

\theoremstyle{plain}
\newtheorem{theorem}{Теорема}[chapter]
\theoremstyle{definition}
\newtheorem{definition}{Определение}
\newtheorem*{pruf}{Доказательство}

\newenvironment{myproof}[1][\textbf{\textup{Доказательство}}]
{\begin{proof}[#1]}
	%\renewcommand*{\qedsymbol}{\(\blacksquare\)}}
{\end{proof}}

\DeclareMathOperator{\band}{\&}
\DeclareMathOperator{\trim}{\triangle}
\DeclareMathOperator{\step}{\vdash}

\makeatletter
\newcommand*\bigcdot{\mathpalette\bigcdot@{.5}}
\newcommand*\bigcdot@[2]{\mathbin{\vcenter{\hbox{\scalebox{#2}{$\m@th#1\bullet$}}}}}
\makeatother


%\newcommand*\pathto[1]{ \Rightarrow^{*}_{#1} }

\newcommand{\pathto}[1][1]{ \Rightarrow^{*}_{#1} }

\newcommand{\dx}[2]{\frac{d#1}{d#2}}
\newcommand{\pdx}[2]{\frac{\partial#1}{\partial#2}}
\newcommand{\zap}[1]{\reflectbox{$3$}#1 3}

\newcommand{\incfig}[2]{%
    \def\svgwidth{#2\columnwidth}
    \import{./images/}{#1.pdf_tex}
}

\tikzstyle{startstop} = [rectangle, rounded corners, 
minimum width=3cm, 
minimum height=1cm,
text centered, 
draw=black, 
fill=red!30]

\tikzstyle{io} = [trapezium, 
trapezium stretches=true, % A later addition
trapezium left angle=70, 
trapezium right angle=110, 
minimum width=3cm, 
minimum height=1cm, text centered, 
draw=black, fill=blue!30]

\tikzstyle{process} = [rectangle, 
minimum width=3cm, 
minimum height=1cm, 
text centered, 
text width=3cm, 
draw=black, 
fill=orange!30]

\tikzstyle{decision} = [diamond, 
minimum width=3cm, 
minimum height=1cm, 
text centered, 
draw=black, 
fill=green!30]
\tikzstyle{arrow} = [thick,->,>=stealth]


%\newcommand{\void}{\varnothing}
\let\void\varnothing

\renewcommand{\phi}{\varphi}
\renewcommand{\epsilon}{\varepsilon}
\newcommand{\scr}[1]{\mathscr{#1}}



\title{}
\author{Козырнов Александр Дмитриевич, ИУ7-32Б}
\date{\today}

\begin{document}

\begin{myproof}
От противного. Предполагаем, что для языка L невозможен полуразрешающий, то возможен разрешающий НА.

Пусть $\scr{A}_L$ - разрешающий НА для $L \subseteq V^{*}$ 

По теореме о разветвлении строим
\[
    \scr{B}_L = {}_{\scr{A}_L}(\scr{A}_L \lor Null),
\] 
где
\[
Null: \begin{cases}
    \to 
\end{cases}
\]

\medskip

Если $\scr{A}_L(x) = \lambda$, то есть $x \in L$, то $\scr{B}_L(x) = \scr{A}_L(x) = \lambda$.

Если $\scr{A}_L(x) \neq \lambda$, то есть $x \not\in L$, отсюда $\lnot !\scr{B}_L(x)$, так как
$\lnot ! Null(x)$ 

Итак, $!\scr{B}_L(x) \Longleftrightarrow x \in L$, то есть $\scr{B}_L$ - полуразрешающий НА для
$L$ вопреки условию теоремы.
\end{myproof}

\begin{theorem}
Если язык $L$ разрешим, то и разрешимо его дополнение.
 \[
     \scr{A}_L(x) = \lambda \Longleftrightarrow x \in L\text{, то есть } \scr{A}_L \neq \lambda
     \Longleftrightarrow x \not\in L\text{ при } (\forall x)!\scr{A}_L(x)
\] 
\end{theorem}

\medskip

Для универсального языка:
\[
L = V^{*} \quad \scr{A}_{V^{*}}: \begin{cases}
    \xi \to \text{ //}\xi \in V\\
    \to \bigcdot
\end{cases}
\] 
Отсюда следует, что и пустой язык тоже разрешим, потому что он - дополнение универсального.

\begin{definition}
Конструктивное натуральное число (КНЧ) - это слово вида $0\underbrace{11\ldots 1}_{n\ge 0}$. Ноль
кодирует ноль, 01 кодирует 1 и так далее. КНЧ $x \in V_0^{*}$
\[
0 \to 0; \quad 01 \to 1; \quad 011 \to 2; \quad \ldots
\] 
\end{definition}

\begin{definition}
    Конструктивное целое число (КЦЧ) - это слово вида $[-]n$, где  $n$ - КНЧ.
\end{definition}

\begin{definition}
Конструктивное рациональное число (КРЧ): $m / n$, где $m,n$ - КЦЧ, то есть слово в
 $\{0, 1, -, /\} $ и $n \neq 0$
\end{definition}

\begin{definition}
Язык $L \subseteq V^{*}$ называется алгорифмически перечислимым, если может быть построен НА 
$N_L$ такой, что для любого КНЧ  $n$ $!N_L(n)$ и $N_L(n) \in L$, и $(\forall x \in L)$
\underline{осуществимо} КНЧ $n$ такое, что  $x = N_L(n)$
\end{definition}

\medskip

\begin{definition}
$A, \quad \nu: \mathbb{N}_0 \to A$ сюръективно, то есть\newline
$(\forall x \in A)(\exists n \in \mathbb{N}_0)(x = \nu(n))$.
Это называется нумерацией множества $A$.
\end{definition}

Далее будем предполагать, что отображение $\nu$ будет биективной.

\medskip

Проведем нумерацию целых чисел:

\begin{figure}[h]
    \centering
    \incfig{numbering}
    \caption{Порядок чисел}
    \label{fig:}
\end{figure}

Можно записать в виде формулы:
\[
\gamma(n) = \begin{cases}
    -\frac{n}{2}\text{, если $n$ четное}\\
    \frac{n+1}{2}\text{, если $n$ нечетное}
\end{cases}
\] 

\medskip

Сначала сделаем 3 алгорифма, нужных для следующей задачи (?)

\[
\scr{C}: \begin{cases}
    11 \to \\
    0 \to \bigcdot
\end{cases}
\] 
Можем заметить, что $\scr{C}(n) = \lambda \Longleftrightarrow n$ четное

\[
    N_L = {}_{\scr{C}}(\scr{A} \lor \scr{B})
\]

Схема $\scr{A}$:
 \[
\scr{A}: \begin{cases}
    \alpha 11 \to 1\alpha\\
    \alpha \to \bigcdot\\
    01 \to -0\alpha 1\\
    0 \to \bigcdot 0\\
\end{cases}
\]
Причем $\alpha \not\in V_0$

Схема $\scr{B}$
\[
\scr{B}: \begin{cases}
    \alpha 11 \to 1\alpha\\
    \alpha \to \bigcdot\\
    01 \to  0\alpha 11\\
    \to \bigcdot\\
\end{cases}
\]

\medskip

Нужно пронумеровать рациональные числа. Это по факту пары двух целых.
Значит, учимся упорядочивать пары.

\medskip

\medskip

\medskip

\medskip

\medskip

\medskip

\medskip

\medskip

\medskip


\medskip

\medskip

\medskip

\begin{figure}[h]
    \centering
    \incfig{numbering2}
    \caption{Предлагаемый Порядок рациональных чисел}
\end{figure}

\begin{definition}
Область применимости НА $\scr{A}$ относительно алфавита $V$: пусть
 $\scr{A} = (V' \supset V, S, P)$ - НА над $V$; Тогда область применимости НА
 относительно алфавита  $V$ есть множество  $\scr{M}^{V}_{\scr{A}} \leftrightharpoons 
 \{x: x \in V^{*} \text{ и } !\scr{A}(x)\}$, причем $\scr{A}: V^{*} \underset{\bigcdot}{\to } V^{*}$.
 $\scr{M}^{V}_{\scr{A}}$ и есть область применимости.
\end{definition}

\begin{theorem}
Язык $L \subseteq V^{*}$ перечислим тогда и только тогда, когда он является областью применимости
относительно алфавита $V$ некоторого НА.
\end{theorem}

\paragraph*{Следствие.} Всякий разрешимый язык перечислим.
\begin{myproof}
    (следствия). Пусть $L$ - разрешимый язык и  $\scr{A}_L$ - разрешающий НА.

    Строим такой НА $\scr{B}_L = Empty \circ \scr{A}_L$, где $Empty$ применим только к пустому
    слову.
     \[
    Empty: \begin{cases}
        \xi \to \xi\text{ //}\xi \in V\\
        \to \bigcdot\\
    \end{cases}
    \]

    Отсюда получаем
    \[
        !\scr{B}_L(x) \Longleftrightarrow !\scr{A}_L(x)\text{ и } \scr{A}_L(x) = \lambda,
    \] 
    то есть $L = \scr{M}^{V}_{\scr{B}_L}$ 

    Однако обратное неверно!
\end{myproof}

\section{Проблема применимости нормальных алгорифмов Маркова}

\paragraph*{Частная проблема применимости.}
Дан НА $\scr{A}$ в алфавите $V$. Можно ли построить НА  $\scr{B}$ над
алфавитом $V$ такой, что  $(\forall x \in V^{*})!\scr{B}(x)\text{ и } \scr{B}(x) = \lambda
\Longleftrightarrow \lnot ! \scr{A}(x)$. Алгорифм Б задуман для того, чтобы расширить область
применимости алгорфима А.

\paragraph*{Общая проблема применимости.}
Дан алфавит $V$,  $\$ \not\in V \cup V_0$. Можно ли построить НА $\scr{B}$ над
алфавитом $V \cup V_0$ так, что для любых НА $\scr{A}$ в алфавите $V$ и слова $x \in V^{*}$
\[
    ! \scr{B}(\zap{\scr{A}}\$ x)\text{ и } \scr{B}(\zap{\scr{A}}\$ x) = \lambda \Longleftrightarrow
    \lnot ! \scr{A}(x)
\] 

\subsection{Проблема самоприменимости.}

Рассмотрим проблему самоприменимости. Мы хотим, чтобы алгорифм работал со своей
собственной записью.

\paragraph*{Соглашение.}
В дальнейшем, не оговаривая это особо, мы считаем, что алгорифм в алфавите $V$ заменяем его в
алфавит  $V \cup V_0$ 
\[
V \to V \cup V_0
\]
\[
V_1 = V \cup V_0 \cup \{\alpha,\beta\};\alpha, \beta \not\in V \cup V_0 
\]
\[
    \scr{A}: (V \cup V_0)^{*} \subset{\bigcdot}{\to } (V \cup V_0)^{*}
\]
\[
V_2 = V_0 \cup \{\alpha,\beta\} 
\] 


Дан алфавит $V$. Можно ли построить НА  $\scr{B}$ над алфавитом $V_0$ такой, что
для любого НА $\scr{A}$ в $V \cup V_0$ будет верно
\[
!\scr{B}(\zap{\scr{A}})\text{ и } \scr{B}(\zap{\scr{A}}) = \lambda \Longleftrightarrow
\lnot ! \scr{A}(\zap{\scr{A}})
\] 

\paragraph*{Примеры.} 
Построим как самоприменимые, так и несамоприменимые НА.

\[
\scr{A}_0: \begin{cases}
    \# a \to a \#\\
    \# b \to b\# \\
    \# \to \bigcdot aba\\
    \to \# \\
\end{cases}
\] 
Дадим ему на вход свою же запись:
\[
    \scr{A}_0: \zap{\scr{A}_0} \step \# \zap{\scr{A}_0}\step\bigcdot aba\zap{\scr{A}_0}
\]
Причем $V_0 \cap \{\#, a, b\} = \void$. Этот алгорифм самоприменим.


\[
\scr{A}_{0}^{f}: \begin{cases}
    0 \to 0 \\
    1 \to 1\\
    \text{Схема $\scr{A}_0$}
\end{cases}
\]
 
Дадим ему на вход свою же запись:
\[
    \scr{A}_{0}^{f}: \zap{\scr{A}^{f}_0} \step \zap{\scr{A}^{f}_0} \step \ldots
\]
То есть $\lnot! \scr{A}^{f}_0(\zap{\scr{A}^{f}_0})$


\paragraph*{Лемма.}
Невозможен НА $\scr{B}$ в алфавите $V \cup V_0$ такой, что для любого
НА $\scr{A}$ в алфавите $V \cup V_0$ имело бы место
\[
    !\scr{B}(\zap{\scr{A}}) \Longleftrightarrow \lnot ! \scr{A}(\zap{\scr{A}})
\] 
\begin{myproof}
Пусть алгорифм $\scr{B}$ построен. Тогда при $\scr{A} = \scr{B}$ имеем:
\[
    !\scr{B}(\zap{\scr{B}}) \Longleftrightarrow \lnot ! \scr{B}(\zap{\scr{B}})
\]
что является противоречием. То есть он применим тогда, когда не применим?)
\end{myproof}

\begin{theorem}
Невозможен НА $\scr{B}$ над алфавитом $V_0$ так, что для любого НА $\scr{A}$ в алфавите
$V_1$ имело бы место
\[
    !\scr{B}(\zap{\scr{A}}) \Longleftrightarrow \lnot ! \scr{A}(\zap{\scr{A}})
\] 
\end{theorem}

\begin{myproof}
По теореме о переводе может быть построен НА $\scr{B}_1$ в алфавите $V_0 \cup \{\alpha,\beta\}$ так,
что $(\forall x \in V^{*}_0)\scr{B}_1(x) \simeq \scr{B}(x)$.

Строим НА $\scr{B}_2$ как естественное распространение НА $\scr{B}_1$ на алфавит $V_1$.

\medskip

Пусть
\[
    !\scr{B}(\zap{\scr{A}}) \Longleftrightarrow \lnot !\scr{A}(\zap{\scr{A}}),
\]
но тогда $!\scr{B}(\zap{\scr{A}}) \Longleftrightarrow !\scr{B}_1(\zap{\scr{A}})
\Longleftrightarrow !\scr{B}_2(\zap{\scr{A}}) \Longleftrightarrow \lnot !\scr{A}(\zap{\scr{A}})$,
что невозможно в силу самой леммы.
\end{myproof}

Итак, мы доказали невозможность полуразрешимость самоприменимости.

Проблема самоприменимости для алгорифмов алгорифмически неразрешима.

\end{document}
