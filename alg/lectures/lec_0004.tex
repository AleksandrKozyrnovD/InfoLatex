\documentclass{report}

\usepackage{amssymb, amsmath, amsthm, amscd}
\usepackage[utf8]{inputenc}
\usepackage[russian]{babel}
\usepackage{bookmark}
\usepackage{wasysym}
\usepackage{import}
\usepackage{caption}
\usepackage{mathrsfs}

\usepackage[makeroom]{cancel}

\usepackage{tikz}
\usetikzlibrary{shapes,shapes.geometric,arrows,positioning,decorations.pathmorphing}

\usepackage{listings}

\usepackage{hyperref}
\hypersetup{
       colorlinks=true,
       linkcolor=blue,
}

\usepackage{geometry}
%\geometry{papersize={15cm, 11in}, left=1.5cm, lmargin=1.5cm,right=2cm, top=2cm, bottom=3cm}
\geometry{a4paper, left=1.5cm, lmargin=1.5cm,right=2cm, top=2cm, bottom=3cm}

\tolerance=1
\emergencystretch=\maxdimen
\hyphenpenalty=10000
\hbadness=10000

\usepackage{graphicx}

\usepackage{titlesec}
\titleformat{\chapter}[display]{\fontsize{17pt}{0pt}\bfseries}{}{-20pt}{}
\titleformat{\subsubsection}[display]{\fontsize{12pt}{0pt}\bfseries}{}{0pt}{}

\newcounter{mylabelcounter}

\makeatletter
\newcommand{\labelText}[2]{%
#1\refstepcounter{mylabelcounter}%
\immediate\write\@auxout{%
    \string\newlabel{#2}{{1}{\thepage}{{\unexpanded{#1}}}{mylabelcounter.\number\value{mylabelcounter}}{}}%
}%
}
\makeatother

\newcommand{\circlesign}[1]{
    \mathbin{
        \mathchoice
        {\buildcircledesign{\displaystyle}{#1}}
        {\buildcircledesign{\textstyle}{#1}}
        {\buildcircledesign{\scriptstyle}{#1}}
        {\buildcircledesign{\scriptscriptstyle}{#1}}
    }
}

\newcommand\buildcircledesign[2]{%
    \begin{tikzpicture}[baseline=(X.base), inner sep=0, outer sep=0]
        \node[draw,circle] (X) {\ensuremath{#1 #2}};
    \end{tikzpicture}%
}

\newcommand{\ozv}{\circlesign{*}}

\theoremstyle{plain}
\newtheorem{theorem}{Теорема}[chapter]
\theoremstyle{definition}
\newtheorem{definition}{Определение}
\newtheorem*{pruf}{Доказательство}

\newenvironment{myproof}[1][\textbf{\textup{Доказательство}}]
{\begin{proof}[#1]}
	%\renewcommand*{\qedsymbol}{\(\blacksquare\)}}
{\end{proof}}

\DeclareMathOperator{\band}{\&}
\DeclareMathOperator{\trim}{\triangle}
\DeclareMathOperator{\step}{\vdash}

\makeatletter
\newcommand*\bigcdot{\mathpalette\bigcdot@{.5}}
\newcommand*\bigcdot@[2]{\mathbin{\vcenter{\hbox{\scalebox{#2}{$\m@th#1\bullet$}}}}}
\makeatother


%\newcommand*\pathto[1]{ \Rightarrow^{*}_{#1} }

\newcommand{\pathto}[1][1]{ \Rightarrow^{*}_{#1} }

\newcommand{\dx}[2]{\frac{d#1}{d#2}}
\newcommand{\pdx}[2]{\frac{\partial#1}{\partial#2}}
\newcommand{\zap}[1]{\reflectbox{$3$}#1 3}

\newcommand{\incfig}[2]{%
    \def\svgwidth{#2\columnwidth}
    \import{./images/}{#1.pdf_tex}
}

\tikzstyle{startstop} = [rectangle, rounded corners, 
minimum width=3cm, 
minimum height=1cm,
text centered, 
draw=black, 
fill=red!30]

\tikzstyle{io} = [trapezium, 
trapezium stretches=true, % A later addition
trapezium left angle=70, 
trapezium right angle=110, 
minimum width=3cm, 
minimum height=1cm, text centered, 
draw=black, fill=blue!30]

\tikzstyle{process} = [rectangle, 
minimum width=3cm, 
minimum height=1cm, 
text centered, 
text width=3cm, 
draw=black, 
fill=orange!30]

\tikzstyle{decision} = [diamond, 
minimum width=3cm, 
minimum height=1cm, 
text centered, 
draw=black, 
fill=green!30]
\tikzstyle{arrow} = [thick,->,>=stealth]


%\newcommand{\void}{\varnothing}
\let\void\varnothing

\renewcommand{\phi}{\varphi}
\renewcommand{\epsilon}{\varepsilon}
\newcommand{\scr}[1]{\mathscr{#1}}



\title{}
\author{Козырнов Александр Дмитриевич, ИУ7-32Б}
\date{\today}

\begin{document}
$V_{\alpha} = \{\alpha,\beta\} $ 

Чаще всего будет рассматривать такой алфавит: $V_0=\{0,1\} $ 

\medskip

\begin{theorem}
    (О переводе). Каков бы ни был нормальный алгорифм $ {\cal A} = (V',S,P)$
    над алфавитом $V \subset V'$, может быть построен НА $ {\cal B}$ в алфавите $V \cup V_{\alpha}$
    так, что $(\forall x \in V^{*})( {\cal B}(x) \simeq {\cal A}(x))$
\end{theorem}

\section{Теорема сочетания}
\subsection{Композиция}

\begin{tikzpicture}[node distance=2cm]
\node (A) [process, xscale=.5, yscale=.6] {$\mathscr{A}$};
\node (leftA) [left of=A, xshift=-1cm];
\node (B) [process, right of=A, xscale=.5, yscale=.6, xshift=1cm] {$\mathscr{B}$};
\node (rightB) [right of=B];

\draw [arrow] (leftA) -- node[above=2mm, anchor=west] {$X$} (A);
\draw [arrow] (A) -- node[above=2mm, anchor=west, xshift=-0.5cm] {$\mathscr{A}(x)$} (B);
\draw [arrow] (B) -- node[above=2mm, anchor=west, xshift=-0.5cm]
    {$\mathscr{B}(\mathscr{A}(x))$} (rightB);

\end{tikzpicture}

\begin{theorem}
    (О композиции). Каковы бы ни были НА $ {\cal A, B}$ в алфавите $V$ может быть построен
    НА алгорифм  $ {\cal C}$ над алфавитом $V$ такой, что 
    \[
        (\forall x \in V^{*})( {\cal C}(x) \simeq {\cal B}( {\cal A}(x) ) )
    \] 
\end{theorem}

\begin{myproof}
Вводится алфавит двойников.

$V = \{a_1,a_2,\ldots,a_n\} $ 
$\overline{V} = \{\overline{a_1}, \overline{a_2},\ldots,\overline{a_{n}}\} $ 

Вводятся две буквы $\alpha,\beta$ такие, что  $\alpha,\beta \not\in V \cup \overline{V}$ 

\[
{\cal C}: \begin{cases}
    \xi\alpha \to \alpha\xi\text{ //}\xi \in V\\
    \alpha\xi \to \alpha \overline{\xi}\\
    \overline{\xi}\eta \to \overline{\xi}\overline{\eta}\text{ //} \xi,\eta \in V\\
    \overline{\xi}\beta \to \beta \overline{\xi}\\
    \beta \overline{\xi} \to \beta\xi\\
    \xi \overline{\eta} \to \xi\eta\\ 
    \alpha\beta \to \bigcdot\\
    \overline{ {\cal B}^{\beta}_{\alpha} }\\
    {\cal A}^{\alpha}\\
\end{cases}
\] 

\[
    \begin{tabular}{|c|c|}
        \hline
        {\cal A}^{\bigcdot} & {\cal A}^{\alpha}\\
        \hline
        u \to v & u\to v\\
        \hline
        u\to \bigcdot v & u\to \alpha v\\
        \hline\hline
\end{tabular}
\] 

\medskip

\[
    \begin{tabular}{|c|c|}
        \hline
        {\cal B}^{\bigcdot} & \overline{ {\cal B}^{\beta}_{\alpha} }\\
        \hline
        u\to v & \overline{u} \to \overline{v}\\
        u \neq \lambda &\\
        \to v & \alpha \to \alpha \overline{v}\\
        u\to \bigcdot v & \overline{u} \to \beta \overline{v}\\
        \to \bigcdot v & \alpha \to \alpha\beta \overline{v}\\
        \hline\hline
\end{tabular}
\] 

\paragraph*{Примерно идея доказательства.}
$x \in V^{*}$ 
\[
    {\cal C}: x \models_{(9)}^{! {\cal A}^{\bigcdot}(x)} y_1\alpha y_2\text{, где } y_1y_2= {\cal A}
    ^{\bigcdot}(x)
\]
Если $\lnot! {\cal A}^{\bigcdot}(x)$, то и $\lnot ! {\cal C}(x)$, заметим. Отсюда
\[
y_1\alpha y_2 \models_{(1)} \alpha y_1 y_2 = \alpha y = \alpha y(1)y(2)\ldots y(m),
\]
где $y_1 y_2 = y$.
Далее получаем
\[
\alpha y(1)y(2)\ldots y(m) \step_{(2)} \alpha \overline{y(1)}y(2)\ldots y(m)\models_{(3)}
\alpha \overline{y(1)} \overline{y(2)} \ldots \overline{y(m)} = \alpha \overline{y}
\] 

Следующий, третий шаг
\[
    \alpha \overline{y} \models_{(8)} \alpha \overline{z_1}, \beta \overline{z_2}_{z}\text{, где }
    z_1,z_1 = z = {\cal B}^{\bigcdot}(y)\text{, если } ! {\cal B}(y)
\]
Заметим, что если $\lnot ! {\cal B}^{\bigcdot}(y) \implies 
\lnot ! {\cal C}(y) \implies \lnot ! {\cal C}(x)$. Получаем
\[
\alpha \overline{z_1}\beta \overline{z_2} \models_{(4)} \alpha\beta \overline{z_1} \overline{z_2}=
\alpha\beta \overline{z}\models_{(5),(6)} \alpha\beta z 
\step\bigcdot z = {\cal B}^{\bigcdot}(y) = {\cal B}^{\bigcdot}( {\cal A}^{\bigcdot}(x)) = 
{\cal B}( {\cal A}(x) )
\] 
\end{myproof}

\paragraph*{Пример.}
\[
{\cal A}^{\bigcdot}: \begin{cases}
    \#\alpha \to \alpha\#\\
    \#\beta \to \beta\#\\
    \# \to \bigcdot aba\\
    \to \#\\
    \to \bigcdot\\
\end{cases}
\]
\[
{\cal B}^{\bigcdot}: \begin{cases}
    \to \bigcdot babb\\
    \to \bigcdot\\
\end{cases}
\] 

Строим систему:
\[
    {\cal A}^{\alpha}: \left[
        \begin{array}
            \# a \to a\#\\
            \# b \to  b \#\\
            \# \to \alpha aba\\
            \to \# \\
            \to \alpha\\
        \end{array}
\] 
\[
    \overline{B}^{\beta}_{\alpha}: \left[
        \begin{array}
            \alpha\alpha \to \alpha\beta \overline{babb}\\
            \alpha \to \alpha\beta\\
        \end{array}
\] 


\[
\begin{align*}
 &   x=bab \step \#bab\models bab\#\step bab\alpha aba\models \alpha bababa \step \\
 &\step\alpha\overline{b}ababa\models \alpha \overline{bababa} \step \\
 &\step \alpha\beta\overline{babbbababa} \step \alpha\beta \alpha\beta
 b \overline{abbbababa}\models\\
 &\models \alpha\beta babbbababa\step\bigcdot babbbababa\\
\end{align*}
\] 

Отсюда видно:
\[
{\cal C} \leftrightharpoons {\cal B} \circ {\cal A};
\]
\[
{\cal B}\circ {\cal A}(x) \simeq {\cal B}( {\cal A}(x) );
\] 
\[
{\cal A}_n \circ {\cal A}_{n-1} \circ \ldots \circ {\cal A}_{1} \leftrightharpoons
{\cal A}_{n} \circ ( {\cal A}_{n-1} \circ \ldots \circ {\cal A}_{1} ), n \ge 1;
\]

\begin{definition}
    Степень алгорифма:
\[
    {\cal A}^{n} \leftrightharpoons {\cal A} \circ {\cal A}^{n-1}, n\ge 1\text{, где }
    {\cal A}^{0} \leftrightharpoons {\cal J}\alpha
\] 
\end{definition}

\subsection{Объединение}

\begin{tikzpicture}[node distance=2cm]
\node (x) {$x$};
\node (A) [process, above right of=x, xscale=.6, yscale=.7, xshift=2.5cm] {$\mathscr{A}$};
\node (B) [process, below right of=x, xscale=.6, yscale=.7, xshift=2.5cm] {$\mathscr{B}$};
\node (rightA) [right of=A, xshift=.5cm] {$\mathscr{A}(x)$};
\node (rightB) [right of=B, xshift=.5cm] {$\mathscr{B}(x)$};
\node (middle) [right of=x, xshift=6cm] {$\mathscr{A}(x)\mathscr{B}(x)$};

\draw [arrow] (x) |- (A);
\draw [arrow] (x) |- (B);
\draw [arrow] (A) -- (rightA);
\draw [arrow] (B) -- (rightB);
\draw [arrow] (rightA) |- (middle);
\draw [arrow] (rightB) |- (middle);
\end{tikzpicture}

\begin{theorem}
    (Объединения). Каковы бы ни были НА $ {\cal A}, {\cal B}$ в алфавите $V$, может быть
    построен НА  $ {\cal A}$ над алфавитом $V$ так, что 
     \[
         (\forall x \in V^{*})( {\cal C}(x) \simeq {\cal A}(x) {\cal B}(x) )
    \] 
\end{theorem}

Можно представить это так:
\begin{gather*}
    \overline{ {\cal C}(x\$y) } \simeq {\cal A}(x)\$ {\cal B}(y)\\
    \$ \not\in V\\
\end{gather*}

\subsection{Разветвление}

\begin{tikzpicture}[node distance=2cm]
\node (C) [startstop, xscale=.6, yscale=.7] {$\scr{C}(x) = \lambda$};
\node (entry) [left of=C];
\node (A) [process, above right of=C, xscale=.6, yscale=.7, xshift=2.5cm] {$\scr{A}(x)$};
\node (B) [process, below right of=C, xscale=.6, yscale=.7, xshift=2.5cm] {$\scr{B}(x)$};
\node (rightA) [right of=A, xshift=1cm];
\node (rightB) [right of=B, xshift=1cm];

\draw [arrow] (entry) -- node[above=2.5mm] {$x$} (C);
\draw [arrow] (C) |- node [above=2mm] {Да} (A);
\draw [arrow] (C) |- node [below=2mm] {Нет} (B);
\draw [arrow] (A) -- node[above=2mm] {$\scr{A}(x)$} (rightA);
\draw [arrow] (B) -- node[above=2mm] {$\scr{B}(x)$} (rightB);
\end{tikzpicture}


Записать в виде псевдокода можно так:
$$if ( {\cal C}(x) = \lambda ) \text{  }\underline{then}\text{  } y:= {\cal A}(x)
\text{  }\underline{else}\text{  }
y:= {\cal B}(x);$$ 

\begin{theorem}
    (О разветвлении). Каковы бы ни были НА $ {\cal A,B,C}$ в алфавите $V$,
    может быть построен НА  $D$ над алфавитом  $V$ так, что
    \[
        (\forall x \in V^{*})(D(x) = {\cal A}(x)\text{, если } {\cal C}(x)=\lambda)
        \text{ и }
        (D(x)= {\cal B}(x)\text{, если } {\cal C}(x) \neq \lambda)
    \] 
\end{theorem}
$D \leftrightharpoons {\cal C}( {\cal A} \lor {\cal B} )$ 

\subsection{Повторение}

\begin{tikzpicture}[node distance=2cm]
\node (B) [startstop, xscale=.6, yscale=.7] {$\scr{B}(x) = \lambda$};
\node (entry) (left of=B);
\node (A) [process, xscale=.6, yscale=.7, right of=B, xshift=5cm] {$\scr{A}$};
\node (belowA) [below of=A, xshift=2cm];
\node (belowB) [below of=B];
\node (rightA) [right of=A];
\node (belowrightA) [below of=rightA];
\node (end)[above of=B, process, xscale=.6, yscale=.7] {END};

\draw [arrow] (entry) -- node[above=2mm] {$x$} (B);
\draw [arrow] (B) -- node[above=2mm] {Да} (A);
\draw (A) -- node [above=3mm, xshift=.5cm] {$x:=\scr{A}(x)$}(rightA);
\draw (rightA) -- (belowrightA);
\draw (belowrightA) |-  (belowB);
\draw [arrow] (belowB) -- (B);
\draw [arrow] (B) -- node[left=3mm] {Нет} (end);

\end{tikzpicture}


В виде псевдокода:
\begin{itemize}
    \item Для цикла с условием, пока правда:
$$\underline{while}\text{  } {\cal B}(x)=\lambda \text{  }\underline{do}\text{  }x:= {\cal A}(x)
\text{  }\underline{end};\text{ Записывается так: } {}_{\beta}\{{\cal A}\}  $$
\item Для цикла с условием, пока неправда:
$$\underline{while}\text{  } {\cal B}(x)!=\lambda \text{  }\underline{do}\text{  }x:= {\cal A}(x)
\text{  }\underline{end}; \text{Записывается так: } {}_{\beta}\langle{\cal A}\rangle $$
\end{itemize}


\begin{theorem}
    (Повторения). Каковы бы ни были НА $ {\cal A,B}$ в алфавите $V$, может быть
    построен НА  $ {\cal C}$ над алфавитом  $V$ такой, что
    $! {\cal C}(x) \leftrightharpoons ( {\cal B}(x) \neq \lambda)$ и тогда
    $ {\cal C}(x) = x$ или существует последовательность $x=x_0,x_1,\ldots,x_{n}$, где
    $(\forall i=\overline{0,n-1})$ $( {\cal B}(x_{i}) = \lambda )$ и
    $x_{i+1} = {\cal A}(x_{i});$ $ {\cal B}(x_{n}) \neq \lambda$ и $ {\cal C}(x)=x_{n}$
\end{theorem}

\end{document}
