\documentclass{report}

\usepackage{amssymb, amsmath, amsthm, amscd}
\usepackage[utf8]{inputenc}
\usepackage[russian]{babel}
\usepackage{bookmark}

\usepackage{tikz}
\usetikzlibrary{shapes,arrows,positioning,decorations.pathmorphing}

\usepackage{listings}

\usepackage{hyperref}
\hypersetup{
       colorlinks=true,
       linkcolor=blue,
}

\usepackage{geometry}
\geometry{papersize={15cm, 11in}, left=1.5cm, lmargin=1.5cm,right=2cm, top=2cm, bottom=3cm}

\usepackage{graphicx}

\usepackage{titlesec}
\titleformat{\chapter}[display]{\fontsize{17pt}{0pt}\bfseries}{}{-20pt}{}
\titleformat{\subsubsection}[display]{\fontsize{12pt}{0pt}\bfseries}{}{0pt}{}

\newcounter{mylabelcounter}

\makeatletter
\newcommand{\labelText}[2]{%
#1\refstepcounter{mylabelcounter}%
\immediate\write\@auxout{%
    \string\newlabel{#2}{{1}{\thepage}{{\unexpanded{#1}}}{mylabelcounter.\number\value{mylabelcounter}}{}}%
}%
}
\makeatother

\newcommand{\bslash}{\mbox{ } \backslash \mbox{ }}
\newcommand{\band}{\mbox{ } \& \mbox{ }}

%\newcommand*\pathto[1]{ \Rightarrow^{*}_{#1} }

\newcommand{\pathto}[1][1]{ \Rightarrow^{*}_{#1} }

\renewcommand{\phi}{\varphi}


\title{}
\author{Козырнов Александр Дмитриевич, ИУ7-32Б}
\date{\today}

\begin{document}

\begin{theorem}
Язык записей несамоприменимых НА неперечислим.
\end{theorem}

\begin{myproof}
    Пусть указанный язык $L = \{\zap{\scr{A}}: \lnot!\scr{A}(\zap{\scr{A}})\} $ перечислим.
    Тогда $L$ есть область применимости относительно алфавита  $V_0$ некоторого НА $\scr{B}$,
    то есть
    \[
        !\scr{B}(\zap{\scr{A}}) \Longleftrightarrow \lnot !\scr{A}(\zap{\scr{A}}),
    \]
    что невозможно!
\end{myproof}

\paragraph*{Один вспомогательный НА.}
Нам нужен такой НА:
\[
Double^{\$}(x) = x\$ x,\quad x \in V^{*},\quad \$ \not\in V
\]

Его схема:
\[
Double^{\$}: \begin{cases}
    \alpha\xi \to \xi\beta\xi\alpha\\
    \beta\xi\eta \to \eta\beta\xi\\
    \alpha \to \$\\
    \beta\xi\$ \to \$ \xi\\
    \$ \to \bigcdot \$\\
    \to \alpha\\
\end{cases}
\]
причем $\alpha,\beta,\# \not\in V; \quad \xi,\eta \in V$

\paragraph*{Пример его работы.} Несколько примеров.

\circlesign{1} $\lambda \step \alpha \step \$ \step\bigcdot \$$

\circlesign{2}  $a \step \alpha a\step a\beta a\alpha \step a\beta a\$ \step a\$ a \step\bigcdot a\$ a$

\begin{align*}
    abc &\step\\
              &\step \alpha abc \step a\beta a\alpha bc \step a\beta ab\beta b\alpha c \step\\
              & \step \ldots \step abc\$ abc \\
              &\step\bigcdot abc\$ abc
\end{align*}


\begin{theorem}
Может быть построен НА $\scr{A}$ в алфавите $V_2$ так, что невозможен НА $\scr{B}$ над
алфавитом $V_2$, для которого выполнялось бы $$!\scr{B}(y) \Longleftrightarrow
\lnot!\scr{A}(y), y \in V_2^{*}$$
\end{theorem}
\begin{myproof}
По теореме об универсальном НА построим НА $U$ над алфавитом  $V_2$ так, что для любых
НА $D$ в алфавите  $V_2$ и слово $y \in V_2^{*}$ выполняется
\[
    U(\zap{D}\$ y) \simeq D(y).
\]

Определим НА $U_1$ так, что
$$(\forall y \in V_2^{*})(U_1(y) \simeq U(y\$ y)),$$ то есть $U_1 = U \circ Double^{\$}$.

Тонкий момент здесь! Алгорифм $U_1$ будучи НА над алфавитом $V_2$ тем самым является и НА
над алфавитом $V_0$ ($V_2$ - расширение $V_0$). По теореме о переводе он может быть заменен
вполне эквивалентным ему относительно алфавита $V_0$
НА $U_2$ в алфавите $V_2$ (то есть в двухбуквенном расширении $V_0$).
\[
    U_2(x) \simeq U_1(x)\text{, где } x \in V_0^{*},
    U_2\text{ - НА в } V_2 = V_0 \cup \{\alpha,\beta\} 
\]

Предположим, что такой НА ${\cal B}$ нашелся.
\[
    !{\cal B}(\zap{D}) \Longleftrightarrow \lnot!U_2(\zap{D}) \Longleftrightarrow
    \lnot!U_1(\zap{D}) \Longleftrightarrow \lnot!U(\zap{D}\$\zap{D}) \Longleftrightarrow
    \lnot !D(\zap{D})
\] 

Он будет полуразразрешающим НА для несамоприменимых НА в языке $V_2$, что невозможно.
\end{myproof}

\paragraph*{Следствие.} Может быть построен НА с неразрешимой частной
проблемой применимости, следовательно его область применимости будет перечислимая, но
неразрешимая (множество?).


\paragraph*{Примеры неразрешимых проблем.} Проблема соответствия Поста.

$\rho = \{(x_1,y_1), (x_2,y_2), \ldots, (x_{n},y_{n})\} \subseteq V^{+^{2}}$

Существует ли $$(x_{i1}, y_{i1}), (x_{i 2}, y_{i 2}), \ldots, (x_{im}, y_{im}): 
x_{i 1}x_{i 2}\ldots x_{im} = y_{i1}y_{i 2}\ldots y_{im}?$$

\section{Порождающие грамматики}

\begin{definition}
${\cal J} = (V, N, S \in N, \Phi), V \cap N = \void$
\end{definition}

Правило вывода: $\alpha \to \beta, \quad \to \not\in V \cup N$

Левая часть $\alpha \in (V \cup N)^{*}N(V \cup N)^{*}$, N - детерминал.

\medskip

Пусть $\gamma, \delta \in (V \cup N)^{*}$. Тогда
\[
    \gamma \underset{{\cal J}}{\step} \delta \leftrightharpoons\text{сущ правило вывода $\alpha
    \to \beta$ в системе $\Phi$ и } \gamma=\gamma_1\alpha\gamma_2, \delta=\gamma_1\beta\gamma_2  
\] 

\begin{definition}
Вывод в порождающей грамматике ${\cal J}$ - это последовательность $\alpha_0, \alpha_1, \ldots,
\alpha_{n}, \ldots$, где $(\forall i \ge 0)(\alpha_{i} \in (V \cup N)^{*})$ и $(\forall i \ge 0)
(\alpha_i \underset{{\cal J}}{\step} \alpha_{i+1})$, если $\alpha_{i+1}$ определен в
последовательности.
\end{definition}

\begin{definition}
$\gamma \step_{{\cal J}}^{*} \delta \leftrightharpoons$ существует вывод \newline$\gamma = 
\alpha_0 \step \alpha_1 \step \ldots \step \alpha_{n} = \delta, n \ge 0$ - длина вывода (к-рая
конечна). 
\end{definition}

\begin{definition}
$L({\cal J}) \leftrightharpoons \{x: x \in V^{*}, S \step_{{\cal J}}^{*}x\} $
\end{definition}

\paragraph*{Примеры грамматик.} ${}$

\begin{itemize}
    \item[1)] $S \to  aSb\big|\lambda$

        $S \step aSb \step aaSbb \step \ldots \step a^{n}Sb^{n} \step a^{n}b^{n}$

        ${\cal J}_1 = (\{a,b\}, \{S\}, S, \Phi_1)$ 

        Тогда язык, порожденный такой грамматикой \newline $L({\cal J}_1) = \{a^{n}b^{n}:n \ge 0\} $ 
    \item[2)] $\Phi_2: S \to aSa \big| bSb \big| aa \big| bb \big| a \big| b \big| \lambda$

        $S \step aSa \step aba$

        $S \step aSa \step abSba \step abbSbba \step abbbba$

        $L({\cal J}_2) = \{x: x = x^{R}, x \in \{a,b\}^{*} \}$ - палиндром
    \item[3)] $S \to ()\big|(S)\big|SS$ - правильная скобочная структура
    \item[4)] ${\cal J}_4 = (\{a,b\}, \{S,A,B,C,D\}, S, \Phi_4)$ 
        \[
        \Phi_4: \begin{cases}
            S \to CD\\
            c \to aCA\big|bcD\big|\lambda\\
            AD \to aD\\
            BD \to bD\\
            Aa \to aA\\
            Ab \to bA\\
            Ba \to aB\\
            Bb \to bB\\
            D \to \lamda
        \end{cases}
        \]

        $S \step CD \step \lambda D \step \lambda\lambda = \lambda$

         $S \step CD \step aCAD \step abcBAD \step abbCBBAD \step abbBBAD \step \newline
         \step abbBBaD \step abbBaBD \step abbaBBD \step abbaBbD \step abbabBD \step \newline
         \step abbabbD \step abbabb$

          $L({\cal J}_4) \supseteq \{\omega\omega: \omega \in \{a,b\}^{*} \} $. Можно доказать,
          что такой язык будет состоять только из двойных слов.

          $L({\cal J}_4) = \{\omega\omega: \omega \in \{a,b\}^{*} \} $
\end{itemize}

\end{document}
