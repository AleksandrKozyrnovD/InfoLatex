\documentclass{report}

\usepackage{amssymb, amsmath, amsthm, amscd}
\usepackage[utf8]{inputenc}
\usepackage[russian]{babel}
\usepackage{bookmark}
\usepackage{wasysym}
\usepackage{import}
\usepackage{caption}
\usepackage{mathrsfs}

\usepackage[makeroom]{cancel}

\usepackage{tikz}
\usetikzlibrary{shapes,shapes.geometric,arrows,positioning,decorations.pathmorphing}

\usepackage{listings}

\usepackage{hyperref}
\hypersetup{
       colorlinks=true,
       linkcolor=blue,
}

\usepackage{geometry}
%\geometry{papersize={15cm, 11in}, left=1.5cm, lmargin=1.5cm,right=2cm, top=2cm, bottom=3cm}
\geometry{a4paper, left=1.5cm, lmargin=1.5cm,right=2cm, top=2cm, bottom=3cm}

\tolerance=1
\emergencystretch=\maxdimen
\hyphenpenalty=10000
\hbadness=10000

\usepackage{graphicx}

\usepackage{titlesec}
\titleformat{\chapter}[display]{\fontsize{17pt}{0pt}\bfseries}{}{-20pt}{}
\titleformat{\subsubsection}[display]{\fontsize{12pt}{0pt}\bfseries}{}{0pt}{}

\newcounter{mylabelcounter}

\makeatletter
\newcommand{\labelText}[2]{%
#1\refstepcounter{mylabelcounter}%
\immediate\write\@auxout{%
    \string\newlabel{#2}{{1}{\thepage}{{\unexpanded{#1}}}{mylabelcounter.\number\value{mylabelcounter}}{}}%
}%
}
\makeatother

\newcommand{\circlesign}[1]{
    \mathbin{
        \mathchoice
        {\buildcircledesign{\displaystyle}{#1}}
        {\buildcircledesign{\textstyle}{#1}}
        {\buildcircledesign{\scriptstyle}{#1}}
        {\buildcircledesign{\scriptscriptstyle}{#1}}
    }
}

\newcommand\buildcircledesign[2]{%
    \begin{tikzpicture}[baseline=(X.base), inner sep=0, outer sep=0]
        \node[draw,circle] (X) {\ensuremath{#1 #2}};
    \end{tikzpicture}%
}

\newcommand{\ozv}{\circlesign{*}}

\theoremstyle{plain}
\newtheorem{theorem}{Теорема}[chapter]
\theoremstyle{definition}
\newtheorem{definition}{Определение}
\newtheorem*{pruf}{Доказательство}

\newenvironment{myproof}[1][\textbf{\textup{Доказательство}}]
{\begin{proof}[#1]}
	%\renewcommand*{\qedsymbol}{\(\blacksquare\)}}
{\end{proof}}

\DeclareMathOperator{\band}{\&}
\DeclareMathOperator{\trim}{\triangle}
\DeclareMathOperator{\step}{\vdash}

\makeatletter
\newcommand*\bigcdot{\mathpalette\bigcdot@{.5}}
\newcommand*\bigcdot@[2]{\mathbin{\vcenter{\hbox{\scalebox{#2}{$\m@th#1\bullet$}}}}}
\makeatother


%\newcommand*\pathto[1]{ \Rightarrow^{*}_{#1} }

\newcommand{\pathto}[1][1]{ \Rightarrow^{*}_{#1} }

\newcommand{\dx}[2]{\frac{d#1}{d#2}}
\newcommand{\pdx}[2]{\frac{\partial#1}{\partial#2}}
\newcommand{\zap}[1]{\reflectbox{$3$}#1 3}

\newcommand{\incfig}[2]{%
    \def\svgwidth{#2\columnwidth}
    \import{./images/}{#1.pdf_tex}
}

\tikzstyle{startstop} = [rectangle, rounded corners, 
minimum width=3cm, 
minimum height=1cm,
text centered, 
draw=black, 
fill=red!30]

\tikzstyle{io} = [trapezium, 
trapezium stretches=true, % A later addition
trapezium left angle=70, 
trapezium right angle=110, 
minimum width=3cm, 
minimum height=1cm, text centered, 
draw=black, fill=blue!30]

\tikzstyle{process} = [rectangle, 
minimum width=3cm, 
minimum height=1cm, 
text centered, 
text width=3cm, 
draw=black, 
fill=orange!30]

\tikzstyle{decision} = [diamond, 
minimum width=3cm, 
minimum height=1cm, 
text centered, 
draw=black, 
fill=green!30]
\tikzstyle{arrow} = [thick,->,>=stealth]


%\newcommand{\void}{\varnothing}
\let\void\varnothing

\renewcommand{\phi}{\varphi}
\renewcommand{\epsilon}{\varepsilon}
\newcommand{\scr}[1]{\mathscr{#1}}



\title{}
\author{Козырнов Александр Дмитриевич, ИУ7-32Б}
\date{\today}

\begin{document}

\subsubsection{Теория первого порядка с равенством}

(НЛ) - это нелогическая теория.

\paragraph*{(НЛ1)} $(\forall x)(x = x)$

\paragraph*{(НЛ2)} $(x = y) \to (A(x, x) \to A(x,y))$, $x \in F \lor (A)$


\begin{theorem} ${}$\newline
\begin{enumerate}
    \item $\step (x = y) \to (y = x)$
    \item $\step (x = y) \to ((y = z) \to (x = z))$
\end{enumerate}
\end{theorem}

\begin{myproof}
    ${}$\newline

1.1 $(x = y) \to ((x = x) \to (y = x))$ - (НЛ2) при $A(x,x) := (x = x), A(x,y) := (y = x)$

1.2  $(\forall x)(x = x)$ - (НЛ1)

1.3 $x = x$ - A4, (2)

1.4  $(x = y) \to (y = x)$ - R2, (1) и (3)

\medskip

2.1 $x = y$ - гипотеза

2.2  $y = z$ - гипотеза

2.3  $(y = x) \to ((y = z) \to (x = z))$ - (НЛ2) при $A(y,y) := (y = z), A(y,x) := (x = z)$

2.4  $(x = y) \to (y = x)$ - теорема (п. 1)

2.5 $y = x$ - MP, (1) и (4)

2.6  $(y = z) \to (x = z)$ - MP, (3) и (5)

2.7 $x = z$ - MP, (2) и (6)
\end{myproof}

\begin{theorem}
Для произвольных термов $s,t$ и  '$u$' имеет место:
 \begin{enumerate}
     \item $\step (t = t)$
     \item  $\step ((s = t) \to (t = s))$
     \item $\step(s = t) \to ((t = u) \to (s = u))$
\end{enumerate}
\end{theorem}

\begin{myproof}
    ${}$\newline

1.1  $(\forall x)(x = x) \to (t = t)$ - схема (4)

1.2 $(\forall x)(x = x)$ - (НЛ1)

1.3 $t = t$ - MP, (1) и (2)

\medskip

2.1 $(x = y) \to (y = x)$ - теорема

2.2 $(\forall y)((x = y) \to (y = x))$ - Gen, (1)

2.3 $(\forall x)(\forall y)((x = y) \to (y = x))$ - Gen, (2)

2.4 $(\forall y)((s = y) \to (y = s))$ - A4, (3), \ $x,y \not\in Var(s)$

2.5 $(s = t) \to (t = s)$ - A4, (4), \ $x,y \not\in Var(t)$ 
\end{myproof}

\section{Метод резолюций (МР)}

МР - это формализованное доказательство от противного.

\subsection{МР в ИВ}

Идея метода резолюции состоит в том, что надо доказать $\step \Phi$. Для этого
мы подвергаем формулу  $\Phi$ в отрицание и преобразуем в конъюнктивной нормальной форме (КНФ).  

Мы понимаем под
$\lnot \Phi = \text{ дизъюнкты } = (A_1 \lor \ldots \lor A_n)(B_1 \lor \ldots\lor B_n)\ldots(C_1 \lor \ldots
\lor C_n)$

\paragraph*{Правило R(esolution)}: \[
\frac{A \lor L_1, \lnot A \lor L_2}{L_1 \lor L_2}
\] 

$L_1 \lor L_2$ называется резолвентой.

\medskip

$\square$ - пустой дизъюнкт.

Например,
 \[
\frac{A, \lnot A}{\square}
\] 

При нахождении пустого дизъюнкта доказательство завершается, доказываемая формула считается тавтологией.

\medskip

Пусть $\Phi = \Theta \to \Psi$. Тогда $\Theta$ преобразуется к КНФ и  $\lnot\Psi$ преобразуется в КНФ.

\paragraph*{Пример.} ${}$ \newline

$\step \underbrace{((A \to B) \to A)}_{\Theta} \to \underbrace{A}_{\Psi}$

$\Theta =
(A \to B) \to A = (\lnot A \lor B) \to A = \lnot (\lnot A \lor B) \lor A = (A \band \lnot B) \lor A = A$

1. A - из посылки

2. $\lnot$A - из отрицания заключения

3.  $\square$ - (1) и (2)

\medskip

Докажем в теории $L$, что резолвента в правиле является логическим следствием своих посылок, то есть
\[
A \lor L_1, \lnot A \lor L_2 \step L_1 \lor L_2
\] 
\[
\lnot A \to  L_1, \lnot\lnot A \to  L_2 \step \lnot L_1 \to L_2
\] 

1. $\lnot A \to L_1$ - гипотеза

2. $\lnot\lnot A \to L_2$ - гипотеза

3. $\lnot L_1$ - гипотеза

4. $\lnot L_1 \to \lnot\lnot A$ - R7, (1)

5. $\lnot L_1 \to L_2$ - R1, (4) и (2)

6. $L_2$ - MP, (3) и (5)

\end{document}
