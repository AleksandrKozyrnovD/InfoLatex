\documentclass{report}

\usepackage{amssymb, amsmath, amsthm, amscd}
\usepackage[utf8]{inputenc}
\usepackage[russian]{babel}
\usepackage{bookmark}
\usepackage{wasysym}
\usepackage{import}
\usepackage{caption}
\usepackage{mathrsfs}

\usepackage[makeroom]{cancel}

\usepackage{tikz}
\usetikzlibrary{shapes,shapes.geometric,arrows,positioning,decorations.pathmorphing}

\usepackage{listings}

\usepackage{hyperref}
\hypersetup{
       colorlinks=true,
       linkcolor=blue,
}

\usepackage{geometry}
%\geometry{papersize={15cm, 11in}, left=1.5cm, lmargin=1.5cm,right=2cm, top=2cm, bottom=3cm}
\geometry{a4paper, left=1.5cm, lmargin=1.5cm,right=2cm, top=2cm, bottom=3cm}

\tolerance=1
\emergencystretch=\maxdimen
\hyphenpenalty=10000
\hbadness=10000

\usepackage{graphicx}

\usepackage{titlesec}
\titleformat{\chapter}[display]{\fontsize{17pt}{0pt}\bfseries}{}{-20pt}{}
\titleformat{\subsubsection}[display]{\fontsize{12pt}{0pt}\bfseries}{}{0pt}{}

\newcounter{mylabelcounter}

\makeatletter
\newcommand{\labelText}[2]{%
#1\refstepcounter{mylabelcounter}%
\immediate\write\@auxout{%
    \string\newlabel{#2}{{1}{\thepage}{{\unexpanded{#1}}}{mylabelcounter.\number\value{mylabelcounter}}{}}%
}%
}
\makeatother

\newcommand{\circlesign}[1]{
    \mathbin{
        \mathchoice
        {\buildcircledesign{\displaystyle}{#1}}
        {\buildcircledesign{\textstyle}{#1}}
        {\buildcircledesign{\scriptstyle}{#1}}
        {\buildcircledesign{\scriptscriptstyle}{#1}}
    }
}

\newcommand\buildcircledesign[2]{%
    \begin{tikzpicture}[baseline=(X.base), inner sep=0, outer sep=0]
        \node[draw,circle] (X) {\ensuremath{#1 #2}};
    \end{tikzpicture}%
}

\newcommand{\ozv}{\circlesign{*}}

\theoremstyle{plain}
\newtheorem{theorem}{Теорема}[chapter]
\theoremstyle{definition}
\newtheorem{definition}{Определение}
\newtheorem*{pruf}{Доказательство}

\newenvironment{myproof}[1][\textbf{\textup{Доказательство}}]
{\begin{proof}[#1]}
	%\renewcommand*{\qedsymbol}{\(\blacksquare\)}}
{\end{proof}}

\DeclareMathOperator{\band}{\&}
\DeclareMathOperator{\trim}{\triangle}
\DeclareMathOperator{\step}{\vdash}

\makeatletter
\newcommand*\bigcdot{\mathpalette\bigcdot@{.5}}
\newcommand*\bigcdot@[2]{\mathbin{\vcenter{\hbox{\scalebox{#2}{$\m@th#1\bullet$}}}}}
\makeatother


%\newcommand*\pathto[1]{ \Rightarrow^{*}_{#1} }

\newcommand{\pathto}[1][1]{ \Rightarrow^{*}_{#1} }

\newcommand{\dx}[2]{\frac{d#1}{d#2}}
\newcommand{\pdx}[2]{\frac{\partial#1}{\partial#2}}
\newcommand{\zap}[1]{\reflectbox{$3$}#1 3}

\newcommand{\incfig}[2]{%
    \def\svgwidth{#2\columnwidth}
    \import{./images/}{#1.pdf_tex}
}

\tikzstyle{startstop} = [rectangle, rounded corners, 
minimum width=3cm, 
minimum height=1cm,
text centered, 
draw=black, 
fill=red!30]

\tikzstyle{io} = [trapezium, 
trapezium stretches=true, % A later addition
trapezium left angle=70, 
trapezium right angle=110, 
minimum width=3cm, 
minimum height=1cm, text centered, 
draw=black, fill=blue!30]

\tikzstyle{process} = [rectangle, 
minimum width=3cm, 
minimum height=1cm, 
text centered, 
text width=3cm, 
draw=black, 
fill=orange!30]

\tikzstyle{decision} = [diamond, 
minimum width=3cm, 
minimum height=1cm, 
text centered, 
draw=black, 
fill=green!30]
\tikzstyle{arrow} = [thick,->,>=stealth]


%\newcommand{\void}{\varnothing}
\let\void\varnothing

\renewcommand{\phi}{\varphi}
\renewcommand{\epsilon}{\varepsilon}
\newcommand{\scr}[1]{\mathscr{#1}}


\title{}
\author{Козырнов Александр Дмитриевич, ИУ7-42Б}
\date{\today}

\begin{document}

\chapter{Элементы Теории Алгоритмов}
\section{Понятие алгоритма в интуитивном смысле слова}


\begin{figure}[ht]
    \centering
    \incfig{command}{.5}
    \caption{Команда}
\end{figure}

$ {\cal A}:X \to Y$

\medskip

Признаки алгоритма:
\begin{itemize}
	\item Признак детерминизированности (нет выбора в алгоритме)
	\item Признак массовости (работает для всех входных данных одного типа~, например,
		квадратных уравнений)
	\item Признак результативности (ожидается какой-то результат)
\end{itemize}


\begin{definition}
алгоритм $A$ применим к элементу $x$. (То есть останавливается за n шагов) \[
    (x \in X)(!A(x))
\] 
\end{definition}

\begin{definition}
$\lnot!A(x)$ - алгоритм $A$ не применим к $x$.
\end{definition}

\begin{definition}
Конструктивный объект - слово в конечном алфавите.
\end{definition}

\begin{figure}[h]
    \centering
    \incfig{automat}{.5}
    \caption{Автомат}
    \label{automat}
\end{figure}

\medskip

\begin{definition}
Вербальная, или словарная, функция - это \[
	f: V^{*} \underset{\bullet}{\to} W^{*}
\] 
Вербальная функция $(V,W)$.

\end{definition}

\begin{definition}
Алгоритм можно записать так: \[
	{\cal A}: V^{*} \underset{\bullet}{\to } W^{*}
\] 
\end{definition}

\begin{definition}
Функция $f: V^{*} \underset{\bullet}{\to } W^{*}$ называется вычислимой в интуитивном смысле слова,
если существует алгоритм $ {\cal A}_f: V^{*} \to  W^{*}$ такой, что
$$(\forall x \in V^{*})
((! {\cal A}_f(x) \Longleftrightarrow x \in D(f))\band ( {\cal A}_f (x) = f(x)))$$
\end{definition}

\section{Машина Тьюринга.}

\begin{figure}[h]
    \centering
    \incfig{turing1}{.5}
    \caption{Машина Тьюринга}
    \label{turing1}
\end{figure}


Команды следующего формата: \[
qa \to rb,
\begin{Bmatrix}
	S \\ L \\ R
\end{Bmatrix}; q,r \in Q; a,b \in V \cup \{\ozv, \square  \} 
\] 

\begin{figure}[h]
    \centering
    \incfig{turing2}{0.7}
    \caption{Что к чему}
    \label{}
\end{figure}

\paragraph*{Заметка.}
Мы считаем, что у нас не может быть команд с одинаковыми левыми частями.

\medskip

\newpage

Начальная конфигурация:

\begin{figure}[h]
    \centering
    \incfig{config1}{.55}
\end{figure}

Заключительная конфигурация:

\begin{figure}[h]
    \centering
    \incfig{config2}{.55}
\end{figure}

\medskip

Пример программы:
\begin{align*}
	&q_0\ozv \to q_0\ozv, R \\
	&q_0a \to q_0a, R \\
	&q_0b \to q_0b, R \\
	&q_0c \to q_1c, R \\
	&q_1a \to q_2a, R \\
	&q_1b \to q_0b, R \\
	&q_1c \to  q_1c, R\\
	&q_2a \to q_0a, R\\
	&q_2b \to q_3b, R\\
	&q_2c \to q_1c, R \\
	&q_3\alpha\to q_3\alpha, R \text{ //} \alpha \in \{a,b,c\} \\
	&q_3\square \to q_4\square, R \\
	&q_{i}\square \to q_5\square, L\text{ //} i = 0,1,2\\
	&q_4\ozv\to q_{f}1,L\\
	&q_5\alpha \to q_5\square, L\\
	&q_5\ozv \to q_5{*}, R\\
	&q_5\square\to q_{f}0, L\\
\end{align*}

\[
f(x) = \begin{cases}
	1\text{, если } cab \sqsubseteq x \in \{a,b,c\} \\
	0\text{ иначе}
\end{cases}
\]

\medskip

\begin{definition}
	Машина Тьюринга (МТ): \[
	{\cal J} = (V, Q, q_0, q_{f},{*}, \square, S, L, R, \delta)
	\] 

\medskip

	Конфигурация МТ: \[
	C = (q,x,ay),
	\]
	где $q \in Q$, а $x,y\in (V \cup \{{*}, \square\})^{*}, a
	\in V \cup \{{*}, \square\} $ 
\end{definition}

\medskip

Мы полагаем, что \[
	(q,x,ay) \underset{ {\cal J} }{\step} \begin{cases}
		(r,x,by)\text{, если } qa\to rb, S \in \delta \\
		(r,x',cby)\text{, где } x'c=x\text{, если} qa\to rb, L \in \delta \\
		(r,xb,dy')\text{, где } y=dy'\text{, если } qa\to rb, R \in \delta \\
	\end{cases}
\] 
\begin{definition}
Вывод на множестве конфигураций:
\[
K_0,K_1,\ldots,K_{n}\text{, где } (\forall i \ge 0)(K_{i}\step K_{i+1}\text{, если } K_{i+1}
	\text{ определен в последовательности})
\] 

\[
    K\underset{ {\cal J} }{\step^{*}}K'
        \text{, если существует вывод } K = K_0 \step K_1 \step \ldots \step K_{n} = K'
\] 
\end{definition}

\end{document}
