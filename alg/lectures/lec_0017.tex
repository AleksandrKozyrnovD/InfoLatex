\documentclass{report}

\usepackage{amssymb, amsmath, amsthm, amscd}
\usepackage[utf8]{inputenc}
\usepackage[russian]{babel}
\usepackage{bookmark}
\usepackage{wasysym}
\usepackage{import}
\usepackage{caption}
\usepackage{mathrsfs}

\usepackage[makeroom]{cancel}

\usepackage{tikz}
\usetikzlibrary{shapes,shapes.geometric,arrows,positioning,decorations.pathmorphing}

\usepackage{listings}

\usepackage{hyperref}
\hypersetup{
       colorlinks=true,
       linkcolor=blue,
}

\usepackage{geometry}
%\geometry{papersize={15cm, 11in}, left=1.5cm, lmargin=1.5cm,right=2cm, top=2cm, bottom=3cm}
\geometry{a4paper, left=1.5cm, lmargin=1.5cm,right=2cm, top=2cm, bottom=3cm}

\tolerance=1
\emergencystretch=\maxdimen
\hyphenpenalty=10000
\hbadness=10000

\usepackage{graphicx}

\usepackage{titlesec}
\titleformat{\chapter}[display]{\fontsize{17pt}{0pt}\bfseries}{}{-20pt}{}
\titleformat{\subsubsection}[display]{\fontsize{12pt}{0pt}\bfseries}{}{0pt}{}

\newcounter{mylabelcounter}

\makeatletter
\newcommand{\labelText}[2]{%
#1\refstepcounter{mylabelcounter}%
\immediate\write\@auxout{%
    \string\newlabel{#2}{{1}{\thepage}{{\unexpanded{#1}}}{mylabelcounter.\number\value{mylabelcounter}}{}}%
}%
}
\makeatother

\newcommand{\circlesign}[1]{
    \mathbin{
        \mathchoice
        {\buildcircledesign{\displaystyle}{#1}}
        {\buildcircledesign{\textstyle}{#1}}
        {\buildcircledesign{\scriptstyle}{#1}}
        {\buildcircledesign{\scriptscriptstyle}{#1}}
    }
}

\newcommand\buildcircledesign[2]{%
    \begin{tikzpicture}[baseline=(X.base), inner sep=0, outer sep=0]
        \node[draw,circle] (X) {\ensuremath{#1 #2}};
    \end{tikzpicture}%
}

\newcommand{\ozv}{\circlesign{*}}

\theoremstyle{plain}
\newtheorem{theorem}{Теорема}[chapter]
\theoremstyle{definition}
\newtheorem{definition}{Определение}
\newtheorem*{pruf}{Доказательство}

\newenvironment{myproof}[1][\textbf{\textup{Доказательство}}]
{\begin{proof}[#1]}
	%\renewcommand*{\qedsymbol}{\(\blacksquare\)}}
{\end{proof}}

\DeclareMathOperator{\band}{\&}
\DeclareMathOperator{\trim}{\triangle}
\DeclareMathOperator{\step}{\vdash}

\makeatletter
\newcommand*\bigcdot{\mathpalette\bigcdot@{.5}}
\newcommand*\bigcdot@[2]{\mathbin{\vcenter{\hbox{\scalebox{#2}{$\m@th#1\bullet$}}}}}
\makeatother


%\newcommand*\pathto[1]{ \Rightarrow^{*}_{#1} }

\newcommand{\pathto}[1][1]{ \Rightarrow^{*}_{#1} }

\newcommand{\dx}[2]{\frac{d#1}{d#2}}
\newcommand{\pdx}[2]{\frac{\partial#1}{\partial#2}}
\newcommand{\zap}[1]{\reflectbox{$3$}#1 3}

\newcommand{\incfig}[2]{%
    \def\svgwidth{#2\columnwidth}
    \import{./images/}{#1.pdf_tex}
}

\tikzstyle{startstop} = [rectangle, rounded corners, 
minimum width=3cm, 
minimum height=1cm,
text centered, 
draw=black, 
fill=red!30]

\tikzstyle{io} = [trapezium, 
trapezium stretches=true, % A later addition
trapezium left angle=70, 
trapezium right angle=110, 
minimum width=3cm, 
minimum height=1cm, text centered, 
draw=black, fill=blue!30]

\tikzstyle{process} = [rectangle, 
minimum width=3cm, 
minimum height=1cm, 
text centered, 
text width=3cm, 
draw=black, 
fill=orange!30]

\tikzstyle{decision} = [diamond, 
minimum width=3cm, 
minimum height=1cm, 
text centered, 
draw=black, 
fill=green!30]
\tikzstyle{arrow} = [thick,->,>=stealth]


%\newcommand{\void}{\varnothing}
\let\void\varnothing

\renewcommand{\phi}{\varphi}
\renewcommand{\epsilon}{\varepsilon}
\newcommand{\scr}[1]{\mathscr{#1}}



\title{}
\author{Козырнов Александр Дмитриевич, ИУ7-32Б}
\date{\today}

\begin{document}

4) $A \lor A \equiv A$

5)  $A \to (B \to C) \equiv (A\band B) \to C$

6) $\lnot(A \to B) \equiv A\band \lnot B$

\begin{definition}
Подформула - это часть формулы, которая сама является формулой. Формула Фи содержит
Тета в виде подформулы - $\Phi[\Theta]$.  $\Phi[\Theta' /\ \Theta]$ - формула, получаемая заменой
$\Theta$ на формулу  $\Theta'$
\end{definition}

\begin{theorem}
    Пусть $\Phi[\Theta](x_1,\ldots,x_n)$. Тогда, если $\Theta' \equiv \Theta$,  то
    $(\forall \widetilde{\alpha} = (\alpha_1,\ldots,\alpha_n))
    \Phi(\Theta' /\ \Theta)(\widetilde{\alpha}) = \Phi[\Theta](\widetilde{\alpha})$
\end{theorem}

\paragraph*{Следствие.}
Если $\step \Phi[\Theta]$, то при  $\Theta' \equiv \Theta \step \Phi[\Theta' /\ \Theta]$

\section{Исчисление предикатов первого порядка}

\subsection{Понятие алгебраической системы}

 \begin{definition}
 $\scr{A} = (A, \Omega, \prod)$ - алгебраическая система. A - множество, далее сигнатура
 операций, сигнатура предикатов.

 $\omega: A^{n} \to A, \quad n\ge 0, \omega \in \Omega$ - операция

 $p: A^{n} \to \{T,F\},\quad n \ge 1 $ - предикат
 \end{definition}

$p(x_1) = T \leftrightharpoons x_1$ есть четное число

$p(x_1,x_2) = T \leftrightharpoons x_1 + x_2 \ge x_1*x_2$

\medskip

Если множество предикатов $\prod = \void$, то получаем алгебру  $\scr{A} = (A, \Omega)$ 

Если множество операций $\Omega = \void$, то получаем модель  $\scr{A} = (A, \prod)$ 

\medskip

Модель - это, например, граф ${\cal J} = (V, \rho)$.

\subsection{ИП1: алфавит, понятие формулы}

\begin{definition}
    Алфавит состоит из таких частей:
\begin{itemize}
    \item[1)] $X = \{x_1,x_2,\ldots,x_{n}\} $ - множество предметных элементов
    \item[2)] ${\cal F} = {\cal F}^{(0)} \cup {\cal F}^{(1)} \cup \ldots \cup F^{(n)} \cup \ldots$ -
        множество функциональных символов
    \item[3)] $\scr{P} = \scr{P}^{(1)} \cup \scr{P}^{(2)} \cup \ldots \cup \scr{P}^{(n)} \cup \ldots$ 
        - множество предикатных символов
    \item[4)] $C = {\cal F}^{(0)}$ - множество предметных констант
    \item[5)] Множество логических символов:  $\to , \lnot, \forall $. $\forall $ - квантор
        общности.
    \item[6)] Множество вспомогательных символов $Aux$
\end{itemize}
\end{definition}

\begin{definition}
Термы - это
\begin{itemize}
    \item[1)] Любая предметная переменная и любая переменная константа есть терм
    \item[2)] Если $t_1,\ldots,t_n$ - термы, а $f^{(n)} \in {\cal F}^{(n)}$, то
        $f^{(n)}(t_1,\ldots,t_n)$ - терм
    \item[3)] Других термов нет
\end{itemize}

\end{definition}

Вместо $f^{(2)}(t_1,t_2)$ пишем $t_1f^{(2)}t_2$

\medskip

$t = (x_1 + x_2)\cdot ((-x_3) + x_1)$

$+, \cdot \in {\cal F}^{(2)},\quad - \in F^{(1)}$


\begin{definition}
Атомарная формула - это выражение вида $p^{(n)}(t_1,\ldots,t_n)$, где $p^{(n)}$ - $n$-арный
предикатный символ, а $t_1,\ldots,t_n$ - термы.
\end{definition}

$ \underbrace{\ge}_{p^{(2)}} ( \underbrace{x_1 + x_1}_{t_1}, \underbrace{x_1*x_2}_{t_2})$ 

\begin{definition}
    Формула - это
\begin{itemize}
    \item[1)] Атомарная формула есть формула.
    \item[2)] Если $\Phi, \Psi$ - формулы, то  $(\Phi \to \Psi)$ - формула
    \item[3)] Если $\Phi$ - формула, то  $(\overline{\Phi})$ - формула
    \item[4)] Если $\Phi$ - формула, а  $x_i \in X$, то $(\forall x_{i})\Phi$ - формула
    \item[5)] Других формул нет
\end{itemize}
\end{definition}


\begin{definition} ${}$\newline
\begin{itemize}
    \item[1)] $\Phi \lor \Psi = \lnot \Phi \to \Psi$
    \item[2)] $\Phi\band\Psi = \lnot(\Phi \to \lnot\Psi)$
    \item[3)] $(\exists x_{i})\Phi = \lnot(\forall x_{i})\lnot\Phi$
\end{itemize}

\end{definition}

\medskip

$F \lor (\Phi)$ - множество свободных переменных в формуле $\Phi$

\end{document}
