\documentclass{report}

\usepackage{amssymb, amsmath, amsthm, amscd}
\usepackage[utf8]{inputenc}
\usepackage[russian]{babel}
\usepackage{bookmark}
\usepackage{wasysym}
\usepackage{import}
\usepackage{caption}
\usepackage{mathrsfs}

\usepackage[makeroom]{cancel}

\usepackage{tikz}
\usetikzlibrary{shapes,shapes.geometric,arrows,positioning,decorations.pathmorphing}

\usepackage{listings}

\usepackage{hyperref}
\hypersetup{
       colorlinks=true,
       linkcolor=blue,
}

\usepackage{geometry}
%\geometry{papersize={15cm, 11in}, left=1.5cm, lmargin=1.5cm,right=2cm, top=2cm, bottom=3cm}
\geometry{a4paper, left=1.5cm, lmargin=1.5cm,right=2cm, top=2cm, bottom=3cm}

\tolerance=1
\emergencystretch=\maxdimen
\hyphenpenalty=10000
\hbadness=10000

\usepackage{graphicx}

\usepackage{titlesec}
\titleformat{\chapter}[display]{\fontsize{17pt}{0pt}\bfseries}{}{-20pt}{}
\titleformat{\subsubsection}[display]{\fontsize{12pt}{0pt}\bfseries}{}{0pt}{}

\newcounter{mylabelcounter}

\makeatletter
\newcommand{\labelText}[2]{%
#1\refstepcounter{mylabelcounter}%
\immediate\write\@auxout{%
    \string\newlabel{#2}{{1}{\thepage}{{\unexpanded{#1}}}{mylabelcounter.\number\value{mylabelcounter}}{}}%
}%
}
\makeatother

\newcommand{\circlesign}[1]{
    \mathbin{
        \mathchoice
        {\buildcircledesign{\displaystyle}{#1}}
        {\buildcircledesign{\textstyle}{#1}}
        {\buildcircledesign{\scriptstyle}{#1}}
        {\buildcircledesign{\scriptscriptstyle}{#1}}
    }
}

\newcommand\buildcircledesign[2]{%
    \begin{tikzpicture}[baseline=(X.base), inner sep=0, outer sep=0]
        \node[draw,circle] (X) {\ensuremath{#1 #2}};
    \end{tikzpicture}%
}

\newcommand{\ozv}{\circlesign{*}}

\theoremstyle{plain}
\newtheorem{theorem}{Теорема}[chapter]
\theoremstyle{definition}
\newtheorem{definition}{Определение}
\newtheorem*{pruf}{Доказательство}

\newenvironment{myproof}[1][\textbf{\textup{Доказательство}}]
{\begin{proof}[#1]}
	%\renewcommand*{\qedsymbol}{\(\blacksquare\)}}
{\end{proof}}

\DeclareMathOperator{\band}{\&}
\DeclareMathOperator{\trim}{\triangle}
\DeclareMathOperator{\step}{\vdash}

\makeatletter
\newcommand*\bigcdot{\mathpalette\bigcdot@{.5}}
\newcommand*\bigcdot@[2]{\mathbin{\vcenter{\hbox{\scalebox{#2}{$\m@th#1\bullet$}}}}}
\makeatother


%\newcommand*\pathto[1]{ \Rightarrow^{*}_{#1} }

\newcommand{\pathto}[1][1]{ \Rightarrow^{*}_{#1} }

\newcommand{\dx}[2]{\frac{d#1}{d#2}}
\newcommand{\pdx}[2]{\frac{\partial#1}{\partial#2}}
\newcommand{\zap}[1]{\reflectbox{$3$}#1 3}

\newcommand{\incfig}[2]{%
    \def\svgwidth{#2\columnwidth}
    \import{./images/}{#1.pdf_tex}
}

\tikzstyle{startstop} = [rectangle, rounded corners, 
minimum width=3cm, 
minimum height=1cm,
text centered, 
draw=black, 
fill=red!30]

\tikzstyle{io} = [trapezium, 
trapezium stretches=true, % A later addition
trapezium left angle=70, 
trapezium right angle=110, 
minimum width=3cm, 
minimum height=1cm, text centered, 
draw=black, fill=blue!30]

\tikzstyle{process} = [rectangle, 
minimum width=3cm, 
minimum height=1cm, 
text centered, 
text width=3cm, 
draw=black, 
fill=orange!30]

\tikzstyle{decision} = [diamond, 
minimum width=3cm, 
minimum height=1cm, 
text centered, 
draw=black, 
fill=green!30]
\tikzstyle{arrow} = [thick,->,>=stealth]


%\newcommand{\void}{\varnothing}
\let\void\varnothing

\renewcommand{\phi}{\varphi}
\renewcommand{\epsilon}{\varepsilon}
\newcommand{\scr}[1]{\mathscr{#1}}



\title{}
\author{Козырнов Александр Дмитриевич, ИУ7-32Б}
\date{\today}

\begin{document}
\section{Классы Поста}
Всего 5 классов.

\begin{itemize}
    \item[1)] ${\cal T}_0 \leftrightharpoons \{f: f(0,\ldots,0) = 0\} $ 
    \item[2)] ${\cal T}_1 \leftrightharpoons \{f: f(1,\ldots,1) = 1\}$
    \item[3)] ${\cal S} \leftrightharpoons \{f: (\forall \widetilde{\alpha})
        (f(\overline{\widetilde{\alpha}}) = \overline{f(\widetilde{\alpha})})\} $

$f \not\in S \Longleftrightarrow (\exists \tildaa)(f(\tildaa) = f(\overline{\tildaa}))$ 

$f, \quad f^{*}(\tildaa) = \overline{f(\overline{\tildaa})} \Longleftrightarrow
\overline{f^{*}(\tildaa)} = f(\overline{\tildaa})$

    \item[4)] ${\cal M} \leftrightharpoons \{f: (\forall \tildaa, \tildab)(\tildaa \le \tildab
            \implies f(\tildaa) \le f(\tildab))$

            $\tildaa = (\alpha_1, \ldots, \alpha_n) \le \tildab =
            (\beta_1, \ldots, \beta_n) \Longleftrightarrow
            (\forall i = \overline{1,n})(\alpha_i \le \beta_i)$
            
        $\overline{{\cal T}_0} \cap \overline{{\cal T}_1} \subseteq \overline{{\cal M}}$ 

    \item[5)] ${\cal L} \leftrightharpoons \{f: 
        f = \sum\limits_{i=1}^{n} (mod2)a_ix_i \oplus a_0\} $ 

        $x_1 \sim x_2 = x_1 \oplus x_2 \oplus a_0 \in {\cal L}$
\end{itemize}

Есть функции, которые принадлежат всем классам Поста, и есть такие, которые не принадлежат никакому.

\medskip

\paragraph*{Лемма 1} (О несамодвойственной функции).
Пусть $f_S \not\in {\cal S}$. Тогда обе константы (0 и 1) представимы формулами
над множеством $\{f_S, \overline{\phantom{A}}\} $ 

\begin{myproof}
    Так как $f_S \not\in {\cal S}$, то $(\exists \tildaa = (\alpha_1, \ldots, \alpha_n))
    (f(\tildaa) = f(\overline{\tildaa}))$ 

    Определим
    $$
    h(x) \leftrightharpoons f_S(x^{\alpha_1}, \ldots, x^{\alpha_n});
    \quad
    h(x) = const \in \{0,1\} 
    .$$ 

    Подставим 1 или 0:
    \begin{align*}
        &h(0) = f_S(0^{\alpha_1}, \ldots, 0^{\alpha_n}) = f_S(\overline{\tildaa})\\
        &h(1) = f_S(1^{\alpha_1}, \ldots, 1^{\alpha_n}) = f_S(\tildaa)
    \end{align*}

    То есть
    \[
    h(0) = h(1) = f_S(\tildaa) = f(\overline{\tildaa}) \in \{0,1\} 
    .\]

    Представим ее как отрицание: $\overline{h(x)} \in \{0,1\} $ - и получим вторую константу.
\end{myproof}


\paragraph*{Лемма 2} (О немонотонной функции).
Если функция $f_M \not\in {\cal M}$, то существует два набора (вектора)  $\tildaa = (\alpha_1, \ldots,
\alpha_{i-1}, 0, \alpha_{i+1}, \ldots, \alpha_n)$ и $\tildab = (\alpha_1, \ldots,
\alpha_{i-1}, 1, \alpha_{i+1}, \ldots, \alpha_n)$, и $f(\tildaa) = 1, f(\tildab) = 0$

\medskip

Рассмотрим такую функцию: $f_M =
                            (1000\text{ }0011\text{ }1111\text{ }1100) \in 
                            \overline{{\cal T}_0} \cap \overline{{\cal T}_1} \implies f_M \not\in 
                            {\cal M}$


\paragraph*{Лемма 3} (О немонотонной функции).
Отрицание может быть представлено формулой над множеством $\{f_M, 0, 1\}$, где
$f_M \not\in M$

\begin{myproof}
В силу леммы 2 берем два набора $\tildaa$ и  $\tildab$. Тогда очевидно отрицание
представимо формулой
 \[
\overline{x} = f_M(\alpha_1, \ldots, \alpha_{i-1}, x, \alpha_{i+1}, \ldots, \alpha_n)
\]

$f_M(\alpha_1, \ldots, \alpha_{i-1}, 0, \alpha_{i+1}, \ldots, \alpha_n) = 1$ и 0 иначе.
\end{myproof}


\paragraph*{Лемма 4} (О нелинейной функции).
Пусть $f_L \not\in {\cal L}$. Тогда конъюнкция может быть представлена формулой
над множеством $\{f_L, 0, \overline{\phantom{A}}\} $ 

\begin{myproof}
Поскольку $f_L$ нелинейная функция, в ее полиноме Жегалкина обязательно будет нелинейное
слагаемое. Среди всех нелинейных слагаемых функции  $f_L$ выбираем самое короткое. Пусть
это самое короткое слагаемое будет  $x_i_1, x_i_2, \ldots, x_i_k. \quad (k \ge 2)$

Строим новую функцию
\begin{align*}
    &f_L' = f_L\bigg|_{x_j = 0 \text{ при } j \neq \{i_1,i_2,\ldots,i_k\} } = x_i_1x_i_2\ldots
    x_i_k \oplus a_i_1x_i_1 \oplus a_i_2x_i_2 \oplus \ldots \oplus a_i_kx_i_k \oplus a_0
\end{align*}

Произвольно делим переменные на две части. Мы строим функцию от двух переменных. Первая
часть переменных есть $x$, вторая -  $y$. 
\[
    \chi(x,y) = f_L'\bigg|_{
        \begin{matrix}
            x_i_1 = \ldots = x_{i_s} = x\\
            x_{i_{s+1}} = \ldots = x_{i_k} = y\\
            1 \le s < k
        \end{matrix}
    } = xy \oplus ax \oplus by \oplus c
,\]
где $a = \sum\limits_{j=1}^{s}(mod 2)a_i_k,\quad b = \sum\limits_{l=s+1}^{k}(mod 2)a_i_l,
\quad c = a_0$

Утверждается, что конъюнкция $xy = \chi(x \oplus b, y \oplus a) \oplus ab \oplus c$.

Посмотрим:
 \begin{align*}
    &(x\oplus b)(y \oplus a) \oplus a(x \oplus b) \oplus b(y \oplus a) \oplus c \oplus ab \oplus c =\\
    & xy \oplus ax \oplus by \oplus ab \oplus ax \oplus ab \oplus by \oplus ab \oplus c \oplus
    ab \oplus c = xy
\end{align*}
Что и требовалось доказать.

\end{myproof}


\end{document}
