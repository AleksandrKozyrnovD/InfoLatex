\documentclass{report}

\usepackage{amssymb, amsmath, amsthm, amscd}
\usepackage[utf8]{inputenc}
\usepackage[russian]{babel}
\usepackage{bookmark}

\usepackage{tikz}
\usetikzlibrary{shapes,arrows,positioning,decorations.pathmorphing}

\usepackage{listings}

\usepackage{hyperref}
\hypersetup{
       colorlinks=true,
       linkcolor=blue,
}

\usepackage{geometry}
\geometry{papersize={15cm, 11in}, left=1.5cm, lmargin=1.5cm,right=2cm, top=2cm, bottom=3cm}

\usepackage{graphicx}

\usepackage{titlesec}
\titleformat{\chapter}[display]{\fontsize{17pt}{0pt}\bfseries}{}{-20pt}{}
\titleformat{\subsubsection}[display]{\fontsize{12pt}{0pt}\bfseries}{}{0pt}{}

\newcounter{mylabelcounter}

\makeatletter
\newcommand{\labelText}[2]{%
#1\refstepcounter{mylabelcounter}%
\immediate\write\@auxout{%
    \string\newlabel{#2}{{1}{\thepage}{{\unexpanded{#1}}}{mylabelcounter.\number\value{mylabelcounter}}{}}%
}%
}
\makeatother

\newcommand{\bslash}{\mbox{ } \backslash \mbox{ }}
\newcommand{\band}{\mbox{ } \& \mbox{ }}

%\newcommand*\pathto[1]{ \Rightarrow^{*}_{#1} }

\newcommand{\pathto}[1][1]{ \Rightarrow^{*}_{#1} }

\renewcommand{\phi}{\varphi}


\title{}
\author{Козырнов Александр Дмитриевич, ИУ7-32Б}
\date{\today}

\begin{document}
\section{Алгебра высказываний. Тавтологии}

У нас есть высказывания $p,q,r,\ldots$ и они могут принимать
значения Ложь (Л) или Истина (И). Указываются по-русски, однако
для упрощения разметки буду использовать F, T (False, True).

$$
\begin{matrix}
    p & q & \lor & \band & \to & \oplus\\
    F & F & F & F & T & F\\
    F & T & T & F & T & T \\
    T & F & T & F & F & T\\
    T & T & T & T &T & F
\end{matrix}
$$

Функции также аналогичны тем, что описаны в математической логике:
\begin{align*}
    &G: \{F, T\}^{n} \to \{F,T\}\\
    &f: \{0,1\}^{n} \to \{0,1\}  
\end{align*}

\begin{definition}
Тавтология - это то, что говорит само за себя
\end{definition}

\section{Исчисление высказываний}

Мы строим ее на основе Теории $L$.

\begin{definition}
Теория $L = (V_L, {\cal F}_L, {\cal A}_L, {\cal P}_L)$. Причем 
$V_L = Var \cup \{\lnot, \to \} \cup Aux, \quad$ ${\cal F}_L: $ 1) Каждая переменная есть формула,
2) Если $\Phi$ - формула, то  $(\lnot \Phi)$ - формула, 3) если  $\Phi$ и $\Psi$ - формулы,
то  $(\Phi \to \Psi)$ - формула, 4) Никаких других формул нет.


Наше "подсахаривание" формул:
    1) $\Phi \lor \Psi = \lnot\Phi \to \Psi$,
    2) \Phi \& \Psi = \lnot(\Phi \to \lnot \Psi)


Схем аксиом всего три:
\[
{\cal A}_L:
\begin{matrix}
    (1) & A \to (B \to A)\\
    (2) & (A \to (B \to C)) \to ((A \to B) \to (A \to C))\\
    (3) & (\lnot B \to \lnot A) \to ((\lnot B \to A) \to B)
\end{matrix}
\] 

И наши правила вывода:
\[
    {\cal P}_L: \frac{A, A\to B}{B} \quad modus \text{ } ponens \text{ } (MP)
\] 
\end{definition}

\paragraph*{Пример Тавтологии.} ${}$ \newline

$\step (A \to A)$ 

\begin{myproof} ${}$\newline 

    1. $A \to ((A \to A) \to A) \to ((A \to (A\to A)) \to (A \to A))$ - схема (2)
    при $B := A \to A,\quad C := A$

    2. $A \to ((A \to A) \to A)$ - схема (1) при $B := A \to A$

    3. $(A \to (A \to A)) \to (A \to A)$ - Modus ponens к шагам (1) и (2)

    4. $A \to (A \to A)$ - схема (1) при $B := A$

    5.  $A \to A$ - modus ponens шагов (3) и (4)
\end{myproof}

\section{Теорема дедукции}

\begin{theorem}
    (Эрбрам). Пусть дано некоторое множество формул, $A$ - произвольная формула,
    тогда если из  $\Gamma, A$ выводится формула  $B$  $(\Gamma, A \step B)$, то
     $\Gamma \step (A \to B)$.

     \[
     \frac{\Gamma, A \step B}{\Gamma \step (A \to B)}
     \] 
\end{theorem}


\paragraph*{Пример применения.} ${}$ \newline

$\step (\lnot B \to \lnot A) \to (A \to B)$

1. $\lnot B \to \lnot A$ - гипотеза

2. $A$ - гипотеза

3. $(\lnot B \to  \lnot A) \to ((\lnot B \to A) \to B)$ - схема 3

4. $(\lnot B \to A)$ - MP, (1) и (3)

5. $A \to (\lnot B \to A)$ - схема 1 при $B := \lnot B$

6.  $\lnot B \to A$ - MP, (2) и (5)

7. $B$ - MP, (4) и (6)

То есть $\lnot B \to \lnot A, A \step B$ по теореме дедукции
$\lnot B \to \lnot A \step A \to B$ по теореме дедекции 
$\step (\lnot B \to \lnot A) \to (A \to B)$
\end{document}
