\documentclass{report}

\usepackage{amssymb, amsmath, amsthm, amscd}
\usepackage[utf8]{inputenc}
\usepackage[russian]{babel}
\usepackage{bookmark}
\usepackage{wasysym}
\usepackage{import}
\usepackage{caption}
\usepackage{mathrsfs}

\usepackage[makeroom]{cancel}

\usepackage{tikz}
\usetikzlibrary{shapes,shapes.geometric,arrows,positioning,decorations.pathmorphing}

\usepackage{listings}

\usepackage{hyperref}
\hypersetup{
       colorlinks=true,
       linkcolor=blue,
}

\usepackage{geometry}
%\geometry{papersize={15cm, 11in}, left=1.5cm, lmargin=1.5cm,right=2cm, top=2cm, bottom=3cm}
\geometry{a4paper, left=1.5cm, lmargin=1.5cm,right=2cm, top=2cm, bottom=3cm}

\tolerance=1
\emergencystretch=\maxdimen
\hyphenpenalty=10000
\hbadness=10000

\usepackage{graphicx}

\usepackage{titlesec}
\titleformat{\chapter}[display]{\fontsize{17pt}{0pt}\bfseries}{}{-20pt}{}
\titleformat{\subsubsection}[display]{\fontsize{12pt}{0pt}\bfseries}{}{0pt}{}

\newcounter{mylabelcounter}

\makeatletter
\newcommand{\labelText}[2]{%
#1\refstepcounter{mylabelcounter}%
\immediate\write\@auxout{%
    \string\newlabel{#2}{{1}{\thepage}{{\unexpanded{#1}}}{mylabelcounter.\number\value{mylabelcounter}}{}}%
}%
}
\makeatother

\newcommand{\circlesign}[1]{
    \mathbin{
        \mathchoice
        {\buildcircledesign{\displaystyle}{#1}}
        {\buildcircledesign{\textstyle}{#1}}
        {\buildcircledesign{\scriptstyle}{#1}}
        {\buildcircledesign{\scriptscriptstyle}{#1}}
    }
}

\newcommand\buildcircledesign[2]{%
    \begin{tikzpicture}[baseline=(X.base), inner sep=0, outer sep=0]
        \node[draw,circle] (X) {\ensuremath{#1 #2}};
    \end{tikzpicture}%
}

\newcommand{\ozv}{\circlesign{*}}

\theoremstyle{plain}
\newtheorem{theorem}{Теорема}[chapter]
\theoremstyle{definition}
\newtheorem{definition}{Определение}
\newtheorem*{pruf}{Доказательство}

\newenvironment{myproof}[1][\textbf{\textup{Доказательство}}]
{\begin{proof}[#1]}
	%\renewcommand*{\qedsymbol}{\(\blacksquare\)}}
{\end{proof}}

\DeclareMathOperator{\band}{\&}
\DeclareMathOperator{\trim}{\triangle}
\DeclareMathOperator{\step}{\vdash}

\makeatletter
\newcommand*\bigcdot{\mathpalette\bigcdot@{.5}}
\newcommand*\bigcdot@[2]{\mathbin{\vcenter{\hbox{\scalebox{#2}{$\m@th#1\bullet$}}}}}
\makeatother


%\newcommand*\pathto[1]{ \Rightarrow^{*}_{#1} }

\newcommand{\pathto}[1][1]{ \Rightarrow^{*}_{#1} }

\newcommand{\dx}[2]{\frac{d#1}{d#2}}
\newcommand{\pdx}[2]{\frac{\partial#1}{\partial#2}}
\newcommand{\zap}[1]{\reflectbox{$3$}#1 3}

\newcommand{\incfig}[2]{%
    \def\svgwidth{#2\columnwidth}
    \import{./images/}{#1.pdf_tex}
}

\tikzstyle{startstop} = [rectangle, rounded corners, 
minimum width=3cm, 
minimum height=1cm,
text centered, 
draw=black, 
fill=red!30]

\tikzstyle{io} = [trapezium, 
trapezium stretches=true, % A later addition
trapezium left angle=70, 
trapezium right angle=110, 
minimum width=3cm, 
minimum height=1cm, text centered, 
draw=black, fill=blue!30]

\tikzstyle{process} = [rectangle, 
minimum width=3cm, 
minimum height=1cm, 
text centered, 
text width=3cm, 
draw=black, 
fill=orange!30]

\tikzstyle{decision} = [diamond, 
minimum width=3cm, 
minimum height=1cm, 
text centered, 
draw=black, 
fill=green!30]
\tikzstyle{arrow} = [thick,->,>=stealth]


%\newcommand{\void}{\varnothing}
\let\void\varnothing

\renewcommand{\phi}{\varphi}
\renewcommand{\epsilon}{\varepsilon}
\newcommand{\scr}[1]{\mathscr{#1}}



\title{}
\author{Козырнов Александр Дмитриевич, ИУ7-32Б}
\date{\today}

\begin{document}
Пусть дана булева алгебра $\scr{B} = (B, \lor,\land, \Theta, I, \overline{\phantom{A}})$ 

$\scr{B}^{n} = (B^{n},\lor,\land, \widetilde{\Theta}, \widetilde{I})$ 

Тогда пусть $\widetilde{\alpha}, \widetilde{\beta} \in \scr{B}^{n}$;
$\bigtilde{\alpha} = (\alpha_1,\ldots,\alpha_{n})$ 

$\bigtilde{\beta} = (\beta_1,\ldots,\beta_{n})$ 

Отсюда 
\begin{align*}
    &\widetilde{\alpha} \lor \widetilde{\beta} \leftrightharpoons
    (\alpha_1 \lor \beta_1, \ldots, \alpha_{n} \lor \beta_{n})
\end{align*}

Аналогично и для $\widetilde{\alpha} \land \widetilde{\beta}$.

\medskip

Также  $\widetilde{\Theta} = (\Theta, \ldots, \Theta)$ и
$\widetilde{I} = (I, \ldots, I)$

 \begin{definition}
 Булев куб размерности n: ${\cal B}^{n} = (\{0,1\}^{n}, \lor, \land,
 \widetilde{0}, \widetilde{1})$
 \end{definition}


\medskip

Рассмотрим всевозможные отображения Х в носитель булевой алгебры
\[
f: X \to B
\]

Тогда можно сказать такое:
\begin{itemize}
    \item[1)] $(f \lor g)(x) \leftrightharpoons f(x) \lor g(x)$
    \item[2)] $(f \land g)(x) \leftrightharpoons f(x) \land g(x)$
    \item[3)] $\overline{f}(x) \leftrightharpoons \overline{f(x)}$ 
    \item(4) $\sigma(x) \leftrightharpoons \Theta \quad (\forall x)$ 
    \item[5)] $\xi(x) = I(\forall x)$
\end{itemize}


\medskip

\begin{definition}
Так обозначается булева алгебра функций:
\[
\scr{B}^{X} = (B^{X}, \lor,\land, \sigma, \xi)
\] 
\end{definition}

Булево кольцо, соответствующее булевой алгебре $\scr{B}$ 
\[
    \scr{R}_{B} = (B, \oplus, \cdot,\Theta, I)
\] 

Отсюда
\begin{align*}
    &a \oplus b \leftrightharpoons a \overline{b} \lor \overline{a}b\\
    &a\cdot b \leftrightharpoons a \land b
\end{align*}   


$\scr{S}_M = (2^{M}, \cup, \cap, \void, M)$ 

$\scr{R}_M = (2^{M}, \triangle, \cap, \void, M)$

\section{Булевые функции. Основные понятия}

\begin{definition}
Булева функция от n переменных:
\[
f: \{0,1\}^{n} \to \{0,1\}  
\] 
\end{definition}

Булева переменная - это $x_1,x_2,\ldots,x_{n}$. Функция выглядит обычно:
$y = f(x_1,\ldots,x_{n})$

\medskip

Множество всех булевых функций:
\[
\scr{P}_2 = \scr{P}_2^{(0)} \cup \scr{P}_2^{(1)} \cup \ldots \cup \scr{P}_2^{(n)} \cup \ldots
\] 

Нам известно определение н-арной операции: $\omega: A^{n} \to A$. То есть булевы функции
своего рода н-арные операции.

$$
\begin{matrix}
    & f_1 & f_2 & f_3 & f_4\\
    0 & 0 & 1 & 0 & 1\\
    1 & 0 & 1 & 1 & 0
\end{matrix}
$$

Можно заметить, что $\overline{x} = x \oplus 1 = x \sim 0$

\medskip

$h = (0011 1110 1010 1110) \Longleftrightarrow h = \{2,3,4,5,6,8,10,12,13,14\} $


\section{Равенство булевых функций. Фиктивные переменные}

\begin{definition}
Пусть есть $f,g: \{0,1\}^{n} \to \{0,1\}$. Тогда функции равны, если

\[
    f = g \leftrightharpoons (\forall \widetilde{\alpha} \in \{0,1\}^{n})(f(\widetilde{\alpha})
    = g (\widetilde{\alpha}))
\] 
\end{definition}

$f(x_1,x_2) = x_1 \lor x_2$

$g(x_1,x_2,x_3) = x_1x_3 \lor x_1\overline{x_3} \lor x_2x_3 \lor x_2\overline{x_3} = 
x_1(x_2 \lor \overline{x_3}) \lor x_2(x_3 \lor \overline{x_3}) = x_1 \lor x_2$

\begin{definition}
Булевы функции считаются равными, если они отличаются друг от друга, может быть, только
фиктивными переменными.
\end{definition}


Можно переформулировать так предыдущее определение.

\begin{definition}
Булевы функции равны, если они существенно зависят от одних и тех же переменных
и на каждом наборе значений этих переменных принимают одинаковые значения
\end{definition}

\medskip

Пусть дан набор значений $X = \{x_1,\ldots,x_{n}\} $.
Тогда селектор $pr_i(x_1,\ldots,x_{i},\ldots,x_{n}) = x_{n}$ и иногда называется
$i$-селектором.

\medskip

Так можно добавит фиктивные переменные:
\[
    y = f(x_1,\ldots,x_{n}) \quad \widetilde{y} = (x_{n+1} \lor \overline{x_{n+1}})
    f(x_1,\ldots,x_{n}) = y
\] 

\section{Суперпозиции и формулы}

\begin{definition}
Пусть у нас есть $f \in \scr{P}_2^{(n)}, \quad g_1,\ldots,g_n \in \scr{P}_2^{(m)}$
\[
    f(g_1,\ldots,g_n)(\widetilde{\alpha}) = f(g_1(\widetilde{\alpha}), \ldots,
    g_n(\widetilde{\alpha})), \quad \widetilde{\alpha} \in  \{0,1\}^{m} 
\]
и это называется суперпозицией.
\end{definition}


\end{document}
