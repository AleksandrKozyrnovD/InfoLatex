\documentclass{report}

\usepackage{amssymb, amsmath, amsthm, amscd}
\usepackage[utf8]{inputenc}
\usepackage[russian]{babel}
\usepackage{bookmark}

\usepackage{tikz}
\usetikzlibrary{shapes,arrows,positioning,decorations.pathmorphing}

\usepackage{listings}

\usepackage{hyperref}
\hypersetup{
       colorlinks=true,
       linkcolor=blue,
}

\usepackage{geometry}
\geometry{papersize={15cm, 11in}, left=1.5cm, lmargin=1.5cm,right=2cm, top=2cm, bottom=3cm}

\usepackage{graphicx}

\usepackage{titlesec}
\titleformat{\chapter}[display]{\fontsize{17pt}{0pt}\bfseries}{}{-20pt}{}
\titleformat{\subsubsection}[display]{\fontsize{12pt}{0pt}\bfseries}{}{0pt}{}

\newcounter{mylabelcounter}

\makeatletter
\newcommand{\labelText}[2]{%
#1\refstepcounter{mylabelcounter}%
\immediate\write\@auxout{%
    \string\newlabel{#2}{{1}{\thepage}{{\unexpanded{#1}}}{mylabelcounter.\number\value{mylabelcounter}}{}}%
}%
}
\makeatother

\newcommand{\bslash}{\mbox{ } \backslash \mbox{ }}
\newcommand{\band}{\mbox{ } \& \mbox{ }}

%\newcommand*\pathto[1]{ \Rightarrow^{*}_{#1} }

\newcommand{\pathto}[1][1]{ \Rightarrow^{*}_{#1} }

\renewcommand{\phi}{\varphi}


\title{}
\author{Козырнов Александр Дмитриевич, ИУ7-32Б}
\date{\today}

\begin{document}

\begin{definition}
Терм $t$ свободен для переменной  $X_i$ в формуле  $\Phi(x_i)$, если никакое свободное
вхождение переменной  $X_i$ в формулу  $\Phi(x_i)$ не находится в области действия квантора
по переменной, входящей в терм.
\end{definition}

\subsection{Понятие интерпретации. Выполнимость, истинность, логическая общезначность.}

\begin{definition}
Интерпретация - это ${\cal J} = (\underbrace{\scr{A}=(A,\Omega,\prod)}_{\text{Область интерпретации}},
i_F,i_P)$

\[
    i_F: {\cal F} \to \Omega\text{, причем } (\forall n \ge 0)i_F(f^{(n)}) \in \Omega^{(n)}
\] 

\[
    I_P: \scr{P} \to \prod\text{, причем } (\forall n\ge 1)i_P(P^{(n)}) \in \text{П}^{(n)}
\] 
\end{definition}


\begin{definition}
Состояние - это
$
\sigma: X \to A
$
\end{definition}


\begin{definition}
    $\sigma \underset{i}{=} \tau \leftrightharpoons$ для всех $i\neq j$ верно
    $\sigma(x_{j}) = \tau(x_{j})$ 
\end{definition}

\begin{definition}
Значение $t ^{\sigma}_{{\cal J}}$ терма $t$ в состоянии  $\sigma$ при интерпретации  ${\cal J}$ 

\begin{itemize}
    \item[1)] Если $t=x_{i}\in X$, то $t ^{\sigma} \leftrightharpoons \sigma(x_{i})$ 
    \item[2)] Если $t = c \in C = {\cal F}^{(0)}$, то $t ^{\sigma} = i_F(c) \in A$ 
    \item[3)] Если $t = f^{(n)}(s_1,\ldots,s_n)$, то $t ^{\sigma} \leftrightharpoons
        i_F (f^{(n)})(s_1^{\sigma},\ldots,s_n^{\sigma})$ 

        Пусть $t = (x_1 + x_2)((-x_3) + x_1x_2)$. Состояние $\sigma = 
    \{1|x_1,2|x_2,3|x_3,\ldots\}  = \{x_1:=1, x_2:=2, x_3:=3,\ldots\}$ 

    То есть $t ^{\sigma} = (3)(-1) = -3$ (просто подставили в формулу и посчитали)

    \item[4)] (Истинностное) значение $\Phi^{\sigma}$ формулы $\Phi$ в состоянии  $\sigma$
        (при заданной интерпретации)
\end{itemize}
\end{definition}


\begin{definition}
Значение формулы с квантором
\begin{itemize}
    \item[1)] Если $\Phi = p^{(n)}(t_1,\ldots,t_n)$, то $\Phi^{\sigma} \leftrightharpoons
        i_P(p^{(n)})(t_1^{\sigma},\ldots,t_n^{\sigma})$ 
    \item[2)] Если $\Phi = \lnot\Psi$, то  $\Phi^{\sigma} = \lnot(\Psi^{\sigma})$ 
    \item[3)] Если $\Phi = \Theta \to \Psi$, то $\Phi^{\sigma} \leftrightharpoons
        \Theta^{\sigma} \to \Psi^{\sigma}$ 
    \item[4)] Если $\Phi = (\forall x_{i})\Psi$, то $\Phi^{\sigma} = T \leftrightharpoons$ 
        Для любого состояния $\tau \underset{i}{=}\sigma :\ \Psi^{\tau} = T$
\end{itemize}
\end{definition}


\begin{definition} ${}$ \newline

$\underset{{\cal J}}{\models} \Phi \leftrightharpoons$ существует состояние $\sigma$, для
которого  $\Phi^{\sigma} = T$

$\underset{{\cal J}}{\step} \Phi \leftrightharpoons$ Для всех состояний $\sigma \ \Phi^{\sigma} = T$

Формула называется логически общезначной, если она истинна в любой интерпретации.
\end{definition}


\subsection{Аксиомы и правила вывода ИП1}


\[
\begin{matrix}
    & (1) & A \to (B \to A)\\
    & (2) & (A \to (B \to C)) \to ((A \to B) \to (A\to C))\\
    & (3) & (\lnot B \to \lnot A) \to ((\lnot B \to A) \to B)\\
    & (4) & (\forall x_{i})A(x_{i}) \to A(t|x_{i})\text{ при } Free(t,x_{i},A)\\
    & (5) & (\forall x_{i})(A \to B) \to (A \to (\forall x_{i})B)\text{ при } x_{i} \not\in F \lor (A)
\end{matrix}
\] 

\paragraph*{Правило А4:} $\frac{(\forall x_{i})A(x_{i})}{A(t)}$, где $Free(t,x_{i},A)$.


\begin{theorem}
Всякая теорема исчисления предикатов первого порядка логически общезначима.
\end{theorem}


По определению в исчислении предикатов первого порядка
считается, что тавтологией считается любая формула, выводимая исключительно
из первых трех схем с применением только правила $modus \ ponens$

Исчисление предикатов первого порядка не противоречиво.

\begin{theorem}
Исчисление предикатов первого порядка полно, то есть любая логически общезначимая формула
доказуема в этом исчислении.
\end{theorem}

\paragraph*{Следствие.}
Формула логически общезначимая тогда и только тогда, когда она доказуема в исчислении предикатов
первого порядка. (Теорема Бёдаля о Полноте).


\end{document}
