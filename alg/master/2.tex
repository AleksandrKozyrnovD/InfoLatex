
Процесс работы НА $ {\cal A} = (S,S,P)$ со словом $x \in V^{*}:$ это последовательность слов
$x=x_0,x_1,\ldots,x_{n}, \ldots$ такая, что $(\forall i\ge 0)( {\cal A}:x_{i} \step x_{i+1} 
\underline{\text{ или }} {\cal A}:x_{i} \step\bigcdot x_{i+1})$, если $x_{i+1}$
определено в последовательности.

Слово $x_{i+1}$ и каждое слово $x_{n} n > i+1$ считается неопределенным, если
$ {\cal A}: x_{i-1} \step\bigcdot x_{i} \underline{\text{ или  }} \sim {\cal A}(x_{i})$

\medskip

Если процесс работы НА $ {\cal A}$ со словом конечный, то есть $x=x_0,x_1,\ldots,x_{n}, n\ge 0$, то
$! {\cal A}(x)$ и $x_{n} \leftrightharpoons {\cal A}(x)$. В противном случае пишем
$\lnot! {\cal A}(x)$, то есть алгоритм со словом х будет бесконечный, или не останавливается.

\medskip

\paragraph*{Об алфавитах в НА.}
Пусть НА алгорифм $ {\cal A} = (V,S,P)$. Тогда мы говорим, что это НА в алфавите $V$.
 Пусть ${\cal A}_{1} = (V_1 \subset V, S_1,P_1)$ - нормальный алгорифм над алфавитом $V$.


\begin{definition}
    Вербальная функция $f: V^{*} \underset{\bullet}{\to } V^{*}$ называется вычислимой по Маркову,
    если может быть построен нормальный алгорифм $ {\cal A}_{f}$ над алфавитом $V$ такой, что
     $$(\forall x \in V^{*})(! {\cal A}_{f}(x) \Longleftrightarrow x \in D(f))
     \band ( {\cal A}_{f}(x) = f(x) )$ $
\end{definition}

\paragraph*{Гипотеза НА (Принцип нормализации).}
Любая вербальная функция, вычислимая в интуитивном смысле слова, вычислима по Маркову.

\paragraph*{Примеры НА.}
Первый пример.
\[
{\cal J}\alpha: \begin{cases}
    \to \bigcdot
\end{cases}
\] 
Получаем вот что: $(\forall x)( {\cal J}\alpha(x) = x )$, то есть вычисляет тождественную функцию
в любом алфавите.

\medskip

Второй пример.
\[
Null: \begin{cases}
    \to 
\end{cases}
\] 
Для любого слова будет работать бесконечно: $(\forall x)\lnot!Null(x)$

\medskip

Третий пример.
\[
Lc: \begin{cases}
    \to \bigcdot x_0\text{, где } x_0 \in V^{*}\text{ - \underline{фиксированное} слово}
\end{cases}
\] 
Получим: $x \in V^{*}: x\step\bigcdot x_0 x$, то есть $Lc(x)=x_0 x$

\medskip

Четвертый пример.
\[
Rc: \begin{cases}
    \#\xi \to  \xi\#\\
    \# \to \bigcdot x_0 (x_0 \in V^{*}\text{ - фиксированное слово }) \\
    \to \#\\
\end{cases}
\] 
$$x \in V^{*}, x = x(1)x(2)\ldots x(k) \step \#x(1)x(2)\ldots x(k) \step 
x(1)\#x(2)\ldots x(k) \models^{k-1} x\#\step\bigcdot x x_0$ $

\medskip

Пятый пример.
\[
Double: \begin{cases}
    \alpha\xi \to  \xi\beta\xi\alpha\\
    \beta\xi\eta \to  \eta\beta\xi\\
    \beta \to \\
    \alpha \to \bigcdot\\
    \to \alpha\\
\end{cases}
\] 
Причем $\alpha,\beta \not\in V; \xi,\eta \in V$.

\medskip

Первый тест:
$\lambda \step \alpha \step\bigcdot \lambda$. 

\medskip

Второй тест:
$a \step \alpha a \step a\beta a\alpha \step aa\alpha \step\bigcdot aa$

\medskip

Третий тест:
\begin{align*}
    abca &\step \alpha abca \step a\beta a \alpha bca \step a\beta a b\beta b\alpha ca \step\\
    &\step a\beta ab\beta bc\beta c\alpha a\step a\beta a b\beta b c\beta c a\beta a\alpha \step\\
    &\step  ab\beta a\beta bc\beta c a\beta a\alpha \step ab\beta ac\beta b\beta ca\beta a\alpha \step\\
    &\step abc\beta a\beta b\beta ca\beta a\alpha \step
    abc\beta a\beta ba\beta c\beta a\alpha \step\\
    &\step abc\beta a a\beta b\beta c\beta a\alpha \step
    abca\beta a\beta b\beta c\beta a\alpha \models^{4}\\
    &\models^{4} abcaabca\alpha \step\bigcdot abcaabca
\end{align*}

Можно строго доказать, что \[
    (\forall x \in V^{*})(Double(x) =x x = x^2)
\]

\section{Эквивалентность нормальных алгоритмов. Теорема о переводе.}
Пусть даны $ {\cal A,B}: V^{*} \underset{\bullet}{\to } V^{*}$ над алфавитом $V$.

 \begin{definition}
Алогрифмы $ {\cal A,B}$ называются эквивалентными относительно алфавита $V$, если
 \[
     (\forall x \in V^{*})
     (! {\cal A}(x) \Longleftrightarrow ! {\cal B}(x) \band ( {\cal A}(x)= {\cal B}(x) ))
\]

Это называется условным равенством:
\[
{\cal A}(x) \simeq {\cal B}(x)
\] 
\end{definition}

\medskip

\paragraph*{
Рассмотрим такую конструкцию, называемую замыканием НА.}
\[
 {\cal A}: \begin{cases}
     u_1 \to [\bigcdot]v_1\\
     \vdots\\
     u_{n} \to [\bigcdor]v_{n}\\
 \end{cases}
\]

\[
{\cal A}^{\bigcdot}: \begin{cases}
    \text{Схема } {\cal A}\\
    \to \bigcdot\\
\end{cases}
\] 
То есть 
\[
    (\forall x \in V^{*}) {\cal A}^{\bigcdot}(x) \simeq {\cal A}(x)
\]

Рассмотрим преобразования:

$ {\cal A}: x\models\bigcdot y$,
то есть $ {\cal A}(x) = y; {\cal A}^{\bigcdot}: x \models y = {\cal A}(x)$.

$ {\cal A}: x\models y$, то есть
$y = {\cal A}(x); {\cal A}^{\bigcdot}:x \models y\step\bigcdot y = {\cal A}(x)$

\paragraph*{Заметка.}
Переход к замыканию НА позволяет без ограничения общности не рассматривать ситуацию естественного
обрыва процесса работы.

\medskip

Если $! {\cal A}(x)$, то $x \models\bigcdot {\cal A}(x)$ (система $ {\cal A}$ замкнутая)

\medskip

\paragraph*{Естественное распространение НА на более широкий алгорифм.}

$ {\cal A} = (V,S,P)$ и пусть $V' \supset V$. Тогда  $ {\cal A}' = (V',S,P)$. То есть просто
означает, что рассматриваем тот же алгоритм в более широком алфавите. Из этого следует, что 
\[
    (\forall x \in V^{*})( {\cal A}(x) \simeq {\cal A}(x) )
\] 

\medskip

\paragraph*{
Формальное распространение НА на более широкий алфавит.}
$ {\cal A} = (V,S,P)$ в алфавите $V$.

 \[
{\cal A}^{f}: \begin{cases}
    \eta \to \eta\text{ //}\eta \in V'\setminus V\\
    \text{Схема } {\cal A}\\
\end{cases}
\] 
Получаем:
\[
    (\forall x \in V^{*})( {\cal A}^{f}(x) = {\cal A}(x) )\text{, но если } x \not\in V^{*}
    \text{, то }\lnot! {\cal A}^{f}(x)
\] 

\medskip

Нам нужно расширить алфавит. Как это делается?

Рассмотрим алфавиты $V = \{a_1,a_2,\ldots,a_{n}\}, V_{\alpha} = \{\alpha,\beta\}$
и $V \cap V_{\alpha} = \void$

Тогда считается
\[
\begin{matrix}
    [a_{i} \leftrightharpoons \alpha\beta^{i}\alpha; & [\lambda = \lambda; &
    [x = [x(1)x(2)\ldots x(k) \leftrightharpoons [x(1)[x(2)\ldots[x(k)\\
\end{matrix}
\] 

\paragraph*{Пример.}
\[
    [ \underbrace{abca}_{V_0} = \underbrace{010}_{a} \underbrace{0110}_{b}
    \underbrace{0111}_{c} \underbrace{010}_{a}
\] 
