\documentclass{report}

\usepackage{amssymb, amsmath, amsthm, amscd}
\usepackage[utf8]{inputenc}
\usepackage[russian]{babel}
\usepackage{bookmark}
\usepackage{wasysym}
\usepackage{import}
\usepackage{caption}
\usepackage{mathrsfs}

\usepackage[makeroom]{cancel}

\usepackage{tikz}
\usetikzlibrary{shapes,shapes.geometric,arrows,positioning,decorations.pathmorphing}

\usepackage{listings}

\usepackage{hyperref}
\hypersetup{
       colorlinks=true,
       linkcolor=blue,
}

\usepackage{geometry}
%\geometry{papersize={15cm, 11in}, left=1.5cm, lmargin=1.5cm,right=2cm, top=2cm, bottom=3cm}
\geometry{a4paper, left=1.5cm, lmargin=1.5cm,right=2cm, top=2cm, bottom=3cm}

\tolerance=1
\emergencystretch=\maxdimen
\hyphenpenalty=10000
\hbadness=10000

\usepackage{graphicx}

\usepackage{titlesec}
\titleformat{\chapter}[display]{\fontsize{17pt}{0pt}\bfseries}{}{-20pt}{}
\titleformat{\subsubsection}[display]{\fontsize{12pt}{0pt}\bfseries}{}{0pt}{}

\newcounter{mylabelcounter}

\makeatletter
\newcommand{\labelText}[2]{%
#1\refstepcounter{mylabelcounter}%
\immediate\write\@auxout{%
    \string\newlabel{#2}{{1}{\thepage}{{\unexpanded{#1}}}{mylabelcounter.\number\value{mylabelcounter}}{}}%
}%
}
\makeatother

\newcommand{\circlesign}[1]{
    \mathbin{
        \mathchoice
        {\buildcircledesign{\displaystyle}{#1}}
        {\buildcircledesign{\textstyle}{#1}}
        {\buildcircledesign{\scriptstyle}{#1}}
        {\buildcircledesign{\scriptscriptstyle}{#1}}
    }
}

\newcommand\buildcircledesign[2]{%
    \begin{tikzpicture}[baseline=(X.base), inner sep=0, outer sep=0]
        \node[draw,circle] (X) {\ensuremath{#1 #2}};
    \end{tikzpicture}%
}

\newcommand{\ozv}{\circlesign{*}}

\theoremstyle{plain}
\newtheorem{theorem}{Теорема}[chapter]
\theoremstyle{definition}
\newtheorem{definition}{Определение}
\newtheorem*{pruf}{Доказательство}

\newenvironment{myproof}[1][\textbf{\textup{Доказательство}}]
{\begin{proof}[#1]}
	%\renewcommand*{\qedsymbol}{\(\blacksquare\)}}
{\end{proof}}

\DeclareMathOperator{\band}{\&}
\DeclareMathOperator{\trim}{\triangle}
\DeclareMathOperator{\step}{\vdash}

\makeatletter
\newcommand*\bigcdot{\mathpalette\bigcdot@{.5}}
\newcommand*\bigcdot@[2]{\mathbin{\vcenter{\hbox{\scalebox{#2}{$\m@th#1\bullet$}}}}}
\makeatother


%\newcommand*\pathto[1]{ \Rightarrow^{*}_{#1} }

\newcommand{\pathto}[1][1]{ \Rightarrow^{*}_{#1} }

\newcommand{\dx}[2]{\frac{d#1}{d#2}}
\newcommand{\pdx}[2]{\frac{\partial#1}{\partial#2}}
\newcommand{\zap}[1]{\reflectbox{$3$}#1 3}

\newcommand{\incfig}[2]{%
    \def\svgwidth{#2\columnwidth}
    \import{./images/}{#1.pdf_tex}
}

\tikzstyle{startstop} = [rectangle, rounded corners, 
minimum width=3cm, 
minimum height=1cm,
text centered, 
draw=black, 
fill=red!30]

\tikzstyle{io} = [trapezium, 
trapezium stretches=true, % A later addition
trapezium left angle=70, 
trapezium right angle=110, 
minimum width=3cm, 
minimum height=1cm, text centered, 
draw=black, fill=blue!30]

\tikzstyle{process} = [rectangle, 
minimum width=3cm, 
minimum height=1cm, 
text centered, 
text width=3cm, 
draw=black, 
fill=orange!30]

\tikzstyle{decision} = [diamond, 
minimum width=3cm, 
minimum height=1cm, 
text centered, 
draw=black, 
fill=green!30]
\tikzstyle{arrow} = [thick,->,>=stealth]


%\newcommand{\void}{\varnothing}
\let\void\varnothing

\renewcommand{\phi}{\varphi}
\renewcommand{\epsilon}{\varepsilon}
\newcommand{\scr}[1]{\mathscr{#1}}



\title{}
\author{Козырнов Александр Дмитриевич, ИУ7-42Б}
\date{\today}

\begin{document}
\subsection{Сплайны}


Остановимся на кубическом сплайне (3-я степень).

\begin{definition}
    (Кубический) Сплайн - полином третьей степени, непрерывный на заданном интервале
    аргумента вместе со своими первыми и вторыми производными.
\end{definition}

\medskip

\paragraph*{Построение сплайна.}
Нужно N интервалов. Зададим на i-м интервале полином третьей степени:
\begin{equation}
\phi_{i}(x) = a_{i} + b_{i}(x-x_{i-1}) + c_{i}(x-x_{i-1})^2 + d_{i}(x-x_{i-1})^3
\end{equation}
Прошу заметить, что в каждом интервале a, b, c, d - различные коэффициенты. То есть их нужно
определить для каждого i-го интервала. Если N интервалов, то всего коэффициентов 4N.

Сначала по всей таблице находим коэффициенты, а потом их используем исходя из х.

\medskip

Получим из
\[
x_{i}-x_{i-1} = h_{i}\\
\] 
Систему уравнений
\begin{gather}
\phi(x_{i-1}) = y_{i-1} = a_{i},\quad i = \overline{1,N}\\
\phi(x_{i}) = y_{i}= a_{i} + b_ih_i + c_ih_i^2 + d_ih_i^3,\quad i = \overline{1,N}
\end{gather}

Заметим, что на стыке интервалов производные равны, отчего мы можем решить систему.

Из (1) получаем уравнения (4):
\begin{gather*}
\phi_{i}'(x) = b_{i} + 2c_{i}(x-x_{i-1}) + 3d_{i}(x-x_{i-1})\\
\phi''_{i}(x) = 2c_{i} + 6d_{i}(x-x_{i-1})(4)\\
\end{gather*}

Получаем такое
\begin{equation}
    b_{i+1} = b_{i} + 2c_ih_i + 3d_ih_i^2
\end{equation}

и такое
\begin{equation}
    c_{i+1} = c_i + 3d_{i}h_i
\end{equation}

\[
\phi_{i+1}(x) = b_{i+1} + 2c_{i+1}(x-x_{i}) + 3d_{i+1}(x-x_{i})^2
\] 
Недостающие два уравнения (на границах интервала от $x_0$ до $x_{N}$) представляют из себя
\[
    \phi''(x_0) = 0, \quad \phi''(x_{N})= 0\text{ - Естественные краевые условия}
\] 
Отсюда получаются моментально два условия:
\begin{equation}
    c_1 = 0
\end{equation}

\begin{equation}
    C_{N} + 3d_{N}h_{N} = 0
\end{equation}

\medskip

Для решения задачи удобно решать ее так: свести эту систему к системе уравнений с
одной переменной ($c_{i}$).

\paragraph*{Пример.}
Из уравнения (5) находится $d_i$.  $d_i$ подставляется в (3). И из полученного уравнения
выражается  $b_i$. Далее найденный  $b_i$ и  $d_i$ подставляется в (4). Получается уравнение
относительно  $c_{i}, c_{i-1}, c_{i+1}$ 

Сделав переименование индексов, получим систему уравнений
\[
    \begin{cases}
    c_1 = 0\\
    h_{i-1}c_{i-1} + 2(h_{i-1} + h_i)c_i + h_ic_{i+1} =
    3\left( \frac{y_{i} - y_{i-1}}{h_i} - \frac{y_{i-1} - y_{i-2}}{h_{i-1}} \right) \\
    c_{N+1} = 0\\
    \end{cases}
\]
Заметим, что $i = \overline{2,N}$ 

\medskip

Решается она методом прогонки (тк получаем трехдиагональную матрицу!)

\paragraph*{Метод прогонки.}
Запишем СУ с трехдиагональной матрицей в каноническом виде
\[
A_iu_{i-1} - B_iu_i + D_iu_{i+1} = -F_i,\quad i = \overline{2,N}
\]

$A_i$ - Это  $h_{i-1}$; $B_i = -2(h_{i-1} + h_i)$. А $F_i$ - это  $3(\frac{\ldots}{\ldots})$.
И $c = u$

\medskip

Решение ищется в виде
\begin{equation}
u_i = \xi_{i+1}u_{i+1} + \eta_{i+1}
\end{equation}
$\xi, \eta$ - прогоночные переменные. Отсюда
 \[
u_{i-1} = \xi_{i}u_i + \eta_i
\]
Подставим в СУ
\[
A_i\xi_iu_i + A_i\eta_i - B_iu_i + D_iu_{i+1} = -F_i
\]
Выразим $u_i$:

\begin{equation}
u_i = \frac{D_i}{B_i - A_i\xi_i}u_{i+1} + \frac{F_i + A_i\eta_i}{B_i - A_i\xi_i}
\end{equation}

Отсюда находим прогоночные коэффициенты:
\begin{equation}
    \xi_{i+1} = \frac{D_i}{B_i - A_i\xi_i}, \quad \eta_{i+1} = 
    \frac{F_i + A_i\eta_i}{B_i - A_i\xi_i}
\end{equation}

Для начала прогона из граничных условий:
\begin{gather*}
    u_1=0\\
    u_1=\xi_2u_2 + \eta_2 = 0 \implies \xi_2 = 0, \eta_2 = 0
\end{gather*}

Она состоит из двух этапов (прогонка):
\begin{itemize}
    \item По (10) ищем $\xi, \eta$
    \item Определение  $u_i$, то есть  $c_i$, по (8) от $i=N$ до  $i = 1$
\end{itemize}
Также известно, что $C_{N+1} = 0$


\medskip

После того, как все коэффициенты $c_i$ найдены, остальные коэффициенты находятся по формулам:
 \begin{align*}
    &a_i = y_{i-1},\quad i=\overline{1,N}\\
    &d_i = \frac{c_{i+1} - c_i}{3h_i},\quad i = \overline{1,N-1}\\
    &d_{N} = - \frac{c_{N}}{3h_{N}}\\
    &b_i = \frac{y_{i} - y_{i-1}}{h_i} - \frac{1}{3}h_i(c_{i+1} + 2c_i), \quad i = \overline{1,N-1}\\
    &b_{N} = \frac{y_{N} - y_{N-1}}{h_{N}} - \frac{2}{3}h_{N}C_{N}\\
\end{align*}

\section{Нелинейная интерполяция\newline (Выравнивающие переменные)}
Если график ''крутой'' , то можно его перевернуть и превратить в линию. Идея состоит в том,
что мы делаем новые переменные, выравнивающие график в линию, проводим в них интерполяцию
(на практически линии) и конвертируем обратно!
\paragraph*{Пример 1.}
\begin{gather*}
    y=ax^{n}\\
    \ln(y) = \ln(a) + n\ln(x)\\
    \ln(y) = \eta, \quad \ln(x) = \xi\\
    \eta = b + n\xi\\
    y = e^{\eta}\\
\end{gather*}

\paragraph*{Пример 2.}
\begin{gather*}
    y = ae^{cx}\\
    \ln(y) = \ln(a) + cx\\
    \eta = \ln(y), \quad \xi \equiv x\\
    \eta = b + cx\\
\end{gather*}

\paragraph*{Пример 3.}
\begin{gather*}
    y = \frac{a}{c + bx}\\
    \ldots \\
\end{gather*}
\end{document}
